%%%%%%%%%%%%%%%%%%%%%%%%%%%%%%%%%
% Ressources
%%%%%%%%%%%%%%%%%%%%%%%%%%%%%%%%%

\nouvelleressource{R111}{Expression-Culture-Communication Professionnelles (ECC1)}

\ajoutheures{30}{21}

%% Les compétences et les ACs
\ajoutcompetence{RT1-Administrer}{\niveauA}



\ajoutcompetence{RT2-Connecter}{\niveauA}

\ajoutac{AC215}{Communiquer avec un client ou un collaborateur}

\ajoutcompetence{RT3-Programmer}{\niveauA}

% Les SAE
\ajoutsae{SAÉ11}{Réseaux / cybersécurité / hygiène informatique}
\ajoutsae{SAÉ12}{Réseau d'entreprise ou personnel}
\ajoutsae{SAÉ13}{Supports de transmission / calculs}
\ajoutsae{SAÉ14}{Se présenter sur Internet}

% Les pre-requis


% Le descriptif
\ajoutancrage{

}

% Contenus
\ajoutcontenudetaille{
* Rechercher, sélectionner ses sources et questionner leur fiabilité
* Analyser et restituer des informations
* Produire des écrits courts, clairs, structurés, adaptés et répondant aux normes de présentation professionnelle et académique (mail, argumentation courte…)
* Réécrire et corriger ses documents
* Découvrir des outils de traitement de texte et de partage des données
* Renforcer les compétences linguistiques selon différents canaux
* Élaborer un discours clair et efficace dans un contexte simple
* Être attentif à ses manières de communiquer (dimensions verbale et nonverbale)
* Comprendre une situation de communication simple
* Savoir utiliser des outils multimédia pour une présentation orale
* Décrire et analyser l’image fixe et mobile
* Adopter des savoirêtre professionnels essentiels dans le travail en équipe (écoute, reformulation, transmission des informations, explications…)
* S’initier aux objectifs du développement durable
* Être sensible aux enjeux du monde contemporain
}

% Mots-clés
\ajoutmotscles{Recherche documentaire, expression écrite, rédaction technique, expression orale, médias, culture générale, esprit critique, développement durable}
