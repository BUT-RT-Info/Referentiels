%%%%%%%%%%%%%%%%%%%%%%%%%%%%%%%%%
% Ressources
%%%%%%%%%%%%%%%%%%%%%%%%%%%%%%%%%

\nouvelleressource{R214}{Analyse des signaux}

\ajoutheures{???}{???}

%% Les compétences et les ACs
\ajoutcompetence{RT1-Administrer}{\niveauA}

\ajoutac{AC111}{Maîtriser les lois fondamentales de l'électricité afin d'intervenir sur des équipements de réseaux et télécommunications}

\ajoutcompetence{RT2-Connecter}{\niveauA}

\ajoutac{AC211}{Mesurer et analyser les signaux}
\ajoutac{AC212}{Caractériser des systèmes de transmissions élémentaires et découvrir la modélisation mathématique de leur fonctionnement}

\ajoutcompetence{RT3-Programmer}{\niveauA}

% Les SAE
\ajoutsae{SAÉ22}{Mesures et caractérisation d’un signal ou d’un système}

% Les pre-requis
\ajoutprerequis{R113,}{None}

% Le descriptif
\ajoutancrage{

}

% Contenus
\ajoutcontenudetaille{
* ) Dérivée

	*  définition

	*  notation s’(t)=ds/dt

	*  équation de la tangente

	*  dérivée des fonctions usuelles

	*  opérations sur les dérivées (somme, produit, quotient, composition)

	*  sens de variation

	*  application à la recherche d’optimum local

		* ) Comportement local et asymptotique

	*  limites (opérations, formes indéterminées)

	*  fonctions négligeables, équivalents

			* ) Intégration

	*  définition d’une intégrale comme une surface

	*  primitive

	*  calcul d’une intégrale à l’aide d’une primitive

	*  intégration par parties et changement de variable
}

% Mots-clés
\ajoutmotscles{Dérivées, Intégrales, limites}
