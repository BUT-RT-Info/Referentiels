%%%%%%%%%%%%%%%%%%%%%%%%%%%%%%%%%
% Ressources
%%%%%%%%%%%%%%%%%%%%%%%%%%%%%%%%%

\nouvelleressource{R206}{Numérisation de l'information}

\ajoutheures{???}{???}

%% Les compétences et les ACs
\ajoutcompetence{RT1-Administrer}{\niveauA}

\ajoutac{AC112}{Comprendre l'architecture des systèmes numériques et les principes du codage de l'information}

\ajoutcompetence{RT2-Connecter}{\niveauA}

\ajoutac{AC211}{Mesurer et analyser les signaux}
\ajoutac{AC212}{Caractériser des systèmes de transmissions élémentaires et découvrir la modélisation mathématique de leur fonctionnement}

\ajoutcompetence{RT3-Programmer}{\niveauA}

% Les SAE


% Les pre-requis


% Le descriptif
\ajoutancrage{
Cette ressource apporte le socle de connaissances et savoir-faire pour les compétences de cœur de métier "Administrer les réseaux et l'Internet" (RT1) et "Connecter les entreprises et les usagers" (RT2)
Les systèmes de Réseaux et Télécoms véhiculent en permanence de données numérisées. Ce module vient donc présenter les principes de la numérisation de l’information, les contraintes de cette numérisation et les conséquences sur la qualité du signal. Il trouvera des prolongements en Téléphonie, ou en Télécommunications numériques.
}

% Contenus
\ajoutcontenudetaille{
* Comprendre la notion de signal numérique, et le principe de la numérisation et de la restitution de signaux analogique
* Échantillonnage des signaux : choix d’une fréquence adéquate d’échantillonnage
* Quantification des signaux – Erreur de quantification
* Filtre Antirepliement et filtre de restitution
* 
}

% Mots-clés
\ajoutmotscles{Numérisation – Échantillonnage – Quantification – Acquisition/Restitution – CAN & CNA}
