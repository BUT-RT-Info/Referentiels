%%%%%%%%%%%%%%%%%%%%%%%%%%%%%%%%%
% Ressources
%%%%%%%%%%%%%%%%%%%%%%%%%%%%%%%%%

\nouvelleressource{R101}{Initiation aux réseaux informatiques}

\ajoutheures{46}{24}

%% Les compétences et les ACs
\ajoutcompetence{RT1-Administrer}{\niveauA}

\ajoutac{AC113}{Configurer les fonctions de base du réseau local}
\ajoutac{AC115}{Identifier les dysfonctionnements du réseau local}
\ajoutac{AC116}{Installer un poste client}

\ajoutcompetence{RT2-Connecter}{\niveauA}

\ajoutac{AC213}{Déployer des supports de transmission}

\ajoutcompetence{RT3-Programmer}{\niveauA}

% Les SAE
\ajoutsae{SAÉ11}{Réseaux / cybersécurité / hygiène informatique}
\ajoutsae{SAÉ12}{Réseau d'entreprise ou personnel}

% Les pre-requis


% Le descriptif
\ajoutancrage{
Cette ressource apporte le socle de connaissances et savoirs-faire pour les compétences de cœur de métier "Administrer les réseaux et l'Internet" (RT1) et "Connecter les entreprises et les usagers" (RT2). Elle contribue aussi à la compétence "Créer des outils et applications informatiques pour les R&T" (RT3) à travers la découverte du poste client et de son environnement logiciel. 
À travers des exercices de mise en place progressive de réseaux locaux, d'interconnection d'équipements et de prise en main des fonctions de base des systèmes d'exploitation, l'étudiant découvrira les principaux concepts utilisés dans les réseaux informatiques, et commencera à comprendre le rôle et les principes des normes et protocoles essentiels des réseaux locaux, comme Ethernet, TCP/IP, DHCP, DNS.
On introduira des notions de sécurité informatique (les ressources associées aux recommandations de l’ANSSI, CyberEdu, CyberMalveillance pourront servir de support)
}

% Contenus
\ajoutcontenudetaille{
* / Initiation au réseau

	*  Découverte et prise en main du réseau local

	*  Adressage IPv4 : classes d'adresses, masques naturels, adressage statique, adressage
dynamique (DHCP)

	*  Notion de routage, de passerelle et de serveur DNS

		* / Bases du système d'exploitation

	*  Architecture d'un système d'exploitation

	*  Différents types de systèmes d'exploitation : les clients, les serveurs, les
systèmes embarqués,

	*  Systèmes d'exploitation commerciaux et Open Sources.

	*  Administration des systèmes d'exploitation

	*  Architectures réseaux et systèmes d'exploitation

			* / Architecture client-serveur dans un réseau local

	*  Mise en place d'une architecture client/serveur simple (serveur d'authentification
/ de fichiers et postes clients associés)
XXX expliciter la place des notions de sécurité informatique citées au dessus.
}

% Mots-clés
\ajoutmotscles{Réseau, système d'exploitation, TCP/IP, LAN, hygiène informatique.}
