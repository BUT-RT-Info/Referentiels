%%%%%%%%%%%%%%%%%%%%%%%%%%%%%%%%%
% Ressources
%%%%%%%%%%%%%%%%%%%%%%%%%%%%%%%%%

\nouvelleressource{R203}{Bases des services réseaux}

\ajoutheures{???}{18}

%% Les compétences et les ACs
\ajoutcompetence{RT1-Administrer}{\niveauA}

\ajoutac{AC113}{Configurer les fonctions de base du réseau local}
\ajoutac{AC114}{Maîtriser les rôles et les principes fondamentaux des systèmes d'exploitation afin d'interagir avec ceux-ci pour la configuration et administration des réseaux et services fournis}
\ajoutac{AC115}{Identifier les dysfonctionnements du réseau local}

\ajoutcompetence{RT2-Connecter}{\niveauA}



\ajoutcompetence{RT3-Programmer}{\niveauA}

% Les SAE


% Les pre-requis


% Le descriptif
\ajoutancrage{
Cette ressource apporte les connaissances et compétences de base nécessaires à la mise en oeuvre des services réseaux  dans un système d’information 
}

% Contenus
\ajoutcontenudetaille{
* Rappels sur les protocoles de transport (TCP, UDP)
* Utilisation de ssh pour l’accès distant
* Principe, installation, configuration et tests des services:
* DHCP
*  DNS (fonctions de base)
* HTTP
* TFTP, FTP
* NTP
On introduira des notions de sécurité informatique (les ressources associées aux recommandations de l’ANSSI, CyberEdu, CyberMalveillance pourront servir de support)
}

% Mots-clés
\ajoutmotscles{Protocoles et ports applicatifs, }
