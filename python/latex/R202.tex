%%%%%%%%%%%%%%%%%%%%%%%%%%%%%%%%%
% Ressources
%%%%%%%%%%%%%%%%%%%%%%%%%%%%%%%%%

\nouvelleressource{R202}{Administration système et fondamentaux de la virtualisation}

\ajoutheures{???}{20}

%% Les compétences et les ACs
\ajoutcompetence{RT1-Administrer}{\niveauA}

\ajoutac{AC113}{Configurer les fonctions de base du réseau local}
\ajoutac{AC114}{Maîtriser les rôles et les principes fondamentaux des systèmes d'exploitation afin d'interagir avec ceux-ci pour la configuration et administration des réseaux et services fournis}
\ajoutac{AC115}{Identifier les dysfonctionnements du réseau local}
\ajoutac{AC116}{Installer un poste client}

\ajoutcompetence{RT2-Connecter}{\niveauA}



\ajoutcompetence{RT3-Programmer}{\niveauA}

% Les SAE
\ajoutsae{SAÉ21}{Construction d’un réseau informatique pour une petite structure}
\ajoutsae{SAÉ24}{Projet intégratif de S2}

% Les pre-requis


% Le descriptif
\ajoutancrage{

}

% Contenus
\ajoutcontenudetaille{
* Scripts pour l’automatisation de séquences de commandes
* Gestions de processus et services (exemple: systemd)
* Gestion de ressources utilisateurs (comptes, quotas)
* Sauvegardes: principes et outils
* Diagnostiquer un système (journaux, outils)
* Concepts, architectures pour la virtualisation et la conteneurisation
* Mise en oeuvre d’infrastructures de virtualisation
* Introduction du Cloud
}

% Mots-clés
\ajoutmotscles{Systèmes d'exploitation, Linux, Windows, Scripts, Virtualisation, Conteneurisation, Cybersécurité.
}
