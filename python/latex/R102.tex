%%%%%%%%%%%%%%%%%%%%%%%%%%%%%%%%%
% Ressources
%%%%%%%%%%%%%%%%%%%%%%%%%%%%%%%%%

\nouvelleressource{R102}{Principes et architecture des réseaux}

\ajoutheures{30}{12}

%% Les compétences et les ACs
\ajoutcompetence{RT1-Administrer}{\niveauA}

\ajoutac{AC114}{Maîtriser les rôles et les principes fondamentaux des systèmes d'exploitation afin d'interagir avec ceux-ci pour la configuration et administration des réseaux et services fournis}
\ajoutac{AC115}{Identifier les dysfonctionnements du réseau local}

\ajoutcompetence{RT2-Connecter}{\niveauA}



\ajoutcompetence{RT3-Programmer}{\niveauA}

% Les SAE
\ajoutsae{SAÉ11}{Réseaux / cybersécurité / hygiène informatique}
\ajoutsae{SAÉ12}{Réseau d'entreprise ou personnel}

% Les pre-requis
\ajoutprerequis{R101}{Initiation aux réseaux informatiques}
\ajoutprerequis{R106}{Architecture des systèmes numériques et informatiques}

% Le descriptif
\ajoutancrage{
Cette ressource apporte le socle de connaissances et savoirs-faire pour les compétences de cœur de métier "Administrer les réseaux et l'Internet" (RT1). 
On introduira des notions de sécurité informatique (les ressources associées aux recommandations de l’ANSSI, CyberEdu, CyberMalveillance pourront servir de support)....à rédiger
}

% Contenus
\ajoutcontenudetaille{
* Approche en couches et encapsulation ,
* Etude détaillée des protocoles Ethernet, ARP, ICMP
* Découverte des protocoles IPv4&6, ICMPv6, TCP, UDP et des protocoles applicatifs,
* Topologies de réseaux,
* Principes de normalisation des technologies de l’Internet,
* Notions sur les métriques de performances: débit, fiabilité, gigue, taux de pertes
Outils préconisés: logiciels du type wireshark, gns3, packet tracer, scapy, marionnet.
}

% Mots-clés
\ajoutmotscles{Architecture en couches, topologies, protocoles, modèle TCP/IP. }
