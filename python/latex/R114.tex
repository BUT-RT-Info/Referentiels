%%%%%%%%%%%%%%%%%%%%%%%%%%%%%%%%%
% Ressources
%%%%%%%%%%%%%%%%%%%%%%%%%%%%%%%%%

\nouvelleressource{R114}{Mathématiques des transmissions}

\ajoutheures{30}{6}

%% Les compétences et les ACs
\ajoutcompetence{RT1-Administrer}{\niveauA}

\ajoutac{AC111}{Maîtriser les lois fondamentales de l'électricité afin d'intervenir sur des équipements de réseaux et télécommunications}

\ajoutcompetence{RT2-Connecter}{\niveauA}

\ajoutac{AC211}{Mesurer et analyser les signaux}
\ajoutac{AC212}{Caractériser des systèmes de transmissions élémentaires et découvrir la modélisation mathématique de leur fonctionnement}

\ajoutcompetence{RT3-Programmer}{\niveauA}

% Les SAE
\ajoutsae{SAÉ13}{Supports de transmission / calculs}
\ajoutsae{SAÉ22}{Mesures et caractérisation d’un signal ou d’un système}

% Les pre-requis


% Le descriptif
\ajoutancrage{

}

% Contenus
\ajoutcontenudetaille{
* ) Trigonométrie

	*  formules  , idem avec sinus

	*  lien avec les vecteurs et le produit scalaire

	*  forme

	*  fonctions trigonométriques réciproques (en particulier arctangente)

		* ) Fonctions logarithme et exponentielle, puissances

	*  graphes

	*  propriétés, retour sur les propriétés des puissances

	*  application au dB

			* ) Nombres complexes

	*  forme algébrique

	*  addition, multiplication et division avec la forme algébrique

	*  forme exponentielle (retour sur les propriétés de l’expo)

	*  addition, multiplication et division avec la forme exponentielle

	*  formules d’Euler

	*  interprétation géométrique, lien avec les vecteurs

	*  lien avec la trigonométrie

	*  racines complexes d’un polynôme de degré 2 (à coefficients réels)
}

% Mots-clés
\ajoutmotscles{Trigonométrie, logarithme, exponentielle, complexes}
