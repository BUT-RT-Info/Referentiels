%%%%%%%%%%%%%%%%%%%%%%%%%%%%%%%%%
% Ressources
%%%%%%%%%%%%%%%%%%%%%%%%%%%%%%%%%

\nouvelleressource{R115}{Gestion de projet}

\ajoutheures{8}{4}

%% Les compétences et les ACs
\ajoutcompetence{RT1-Administrer}{\niveauA}



\ajoutcompetence{RT2-Connecter}{\niveauA}

\ajoutac{AC215}{Communiquer avec un client ou un collaborateur}

\ajoutcompetence{RT3-Programmer}{\niveauA}

% Les SAE
\ajoutsae{SAÉ11}{Réseaux / cybersécurité / hygiène informatique}
\ajoutsae{SAÉ12}{Réseau d'entreprise ou personnel}
\ajoutsae{SAÉ13}{Supports de transmission / calculs}
\ajoutsae{SAÉ14}{Se présenter sur Internet}

% Les pre-requis


% Le descriptif
\ajoutancrage{

}

% Contenus
\ajoutcontenudetaille{
* Partager de façon collective l’information :

	* Utilisation avancée du mail : création d’une adresse générique, utilisation du
CC et du CCI .

	* Utilisation d’outils collaboratifs adaptés (par exemple Slack, MSTeams, Mattermost,
drive, OnlyOffice)
* Organiser son travail et celui de l’équipe à partir d’outils de planification (Gantt, PERT)
* Prendre sa place dans une équipe en connaissant les différents rôles d’une équipe projet
* Conceptualiser les étapes des tâches à réaliser à l’aide d’outils adaptés (cartes mentales, infographies, etc.)
* Prendre  conscience des délais et échéances dans un travail en mode projet
* Savoir s’adapter à des profils professionnels différents (manager, collaborateur, client) qui interviennent dans un projet
* Apprendre à faire un bilan régulier sur l’avancée d’un projet : points bloquants, solutions apportées
* Appliquer la critique constructive dans l’intérêt du projet
* Organiser des réunions de projet
* Présenter un projet selon les spécificités du projet et le public visé.
}

% Mots-clés
\ajoutmotscles{Planification, partage d’informations, organisation, conceptualisation,  réunion.}
