%%%%%%%%%%%%%%%%%%%%%%%%%%%%%%%%%
% Ressources
%%%%%%%%%%%%%%%%%%%%%%%%%%%%%%%%%

\nouvelleressource{R207}{Sources de données}

\ajoutheures{20}{???}

%% Les compétences et les ACs
\ajoutcompetence{RT1-Administrer}{\niveauA}

\ajoutac{AC112}{Comprendre l'architecture des systèmes numériques et les principes du codage de l'information}

\ajoutcompetence{RT2-Connecter}{\niveauA}



\ajoutcompetence{RT3-Programmer}{\niveauA}

% Les SAE
\ajoutsae{SAÉ23}{Mise en place d'une solution informatique pour l’entreprise}
\ajoutsae{SAÉ24}{Projet intégratif de S2}

% Les pre-requis


% Le descriptif
\ajoutancrage{
Le professionnel R&T traite un grand nombre de données. Ces données (par exemple : l’annuaire des utilisateurs du réseau ou l’état des équipements informatiques) peuvent servir à la configuration et à l’administration des services réseau d’une entreprise (compétence RT1-Administrer) ou à alimenter les pages d’un site Web (compétence RT3-Programmer). Il est donc amené à stocker, organiser, gérer, protéger des données provenant de différentes sources (thématiques intégrant le PIX, cf. <a href="https://pix.fr/competences">https://pix.fr/competences</a>), mais aussi à les traiter en développant différents outils informatiques pour ses besoins personnels ou pour son équipe (compétence RT3-Programmer). Plus largement, il contribue activement à l’exploitation et à la maintenance du système d’information de l’ entreprise. 
Cette ressource introduit les éléments fondamentaux des systèmes de gestion de base de données en contribuant à la validation des apprentissages critiques mentionnés précédemment. Elle présente différentes alternatives technologiques pour le stockage et le codage de l’information, en fonction des données, de leur usage. L’accès aux données utilise des langages et des scripts spécifiques qui seront introduits. 
}

% Contenus
\ajoutcontenudetaille{
*  Stockage et accès aux données  :
* système de gestion de données (relationnel/non relationnel);
* structuration des données: fichiers (CSV, JSON), exemples de sources ouvertes (open data), web scraping;
* sensibilisation à la réglementation française et internationale (CNIL, RGPD).
*  Base de données relationnelles :
* Schéma relationnel d'une base de données
* Sensibilisation aux contraintes d'intégrité
* Création de tables simples
* Interrogation de données
* Ajout et modification de données
*  Lecture d'une documentation technique (UML, diagramme de classes)
L’utilisation de l’anglais est préconisée dans la documentation du code.
}

% Mots-clés
\ajoutmotscles{Base de données, langages informatiques, algorithmes}
