%%%%%%%%%%%%%%%%%%%%%%%%%%%%%%%%%
% Ressources
%%%%%%%%%%%%%%%%%%%%%%%%%%%%%%%%%

\nouvelleressource{R213}{Mathématiques des systèmes numériques}

\ajoutheures{???}{???}

%% Les compétences et les ACs
\ajoutcompetence{RT1-Administrer}{\niveauA}

\ajoutac{AC112}{Comprendre l'architecture des systèmes numériques et les principes du codage de l'information}

\ajoutcompetence{RT2-Connecter}{\niveauA}

\ajoutac{AC212}{Caractériser des systèmes de transmissions élémentaires et découvrir la modélisation mathématique de leur fonctionnement}

\ajoutcompetence{RT3-Programmer}{\niveauA}

% Les SAE
\ajoutsae{SAÉ22}{Mesures et caractérisation d’un signal ou d’un système}
\ajoutsae{SAÉ23}{Mise en place d'une solution informatique pour l’entreprise}

% Les pre-requis


% Le descriptif
\ajoutancrage{

}

% Contenus
\ajoutcontenudetaille{
* ) Suites, récurrence, signal numérique

	*  raisonnement par récurrence

	*  suites récurrentes

	*  signal discret (exemples : Kronecker, échelon échantillonné…)

	*  convergence d’une suite (opérations sur les limites)

		* ) Vecteurs en D et 3D

	*  définitions

	*  opérations (addition et multiplication externe)

	*  produit scalaire (lien avec la trigonométrie)

	*  application au calcul d’une équation de droite

			* ) Matrices et vecteurs

	* définition

	* Opérations

	* Résolutions de systèmes linéaires (pivot de Gauss)
}

% Mots-clés
\ajoutmotscles{Suites, ensembles, vecteurs, matrices}
