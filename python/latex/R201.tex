%%%%%%%%%%%%%%%%%%%%%%%%%%%%%%%%%
% Ressources
%%%%%%%%%%%%%%%%%%%%%%%%%%%%%%%%%

\nouvelleressource{R201}{Technologies de l’Internet}

\ajoutheures{???}{30}

%% Les compétences et les ACs
\ajoutcompetence{RT1-Administrer}{\niveauA}

\ajoutac{AC113}{Configurer les fonctions de base du réseau local}
\ajoutac{AC115}{Identifier les dysfonctionnements du réseau local}
\ajoutac{AC116}{Installer un poste client}

\ajoutcompetence{RT2-Connecter}{\niveauA}

\ajoutac{AC213}{Déployer des supports de transmission}

\ajoutcompetence{RT3-Programmer}{\niveauA}

% Les SAE
\ajoutsae{SAÉ21}{Construction d’un réseau informatique pour une petite structure}
\ajoutsae{SAÉ24}{Projet intégratif de S2}

% Les pre-requis
\ajoutprerequis{R101}{Initiation aux réseaux informatiques}
\ajoutprerequis{R102}{Principes et architecture des réseaux}
\ajoutprerequis{R103}{Réseaux locaux et équipements actifs}

% Le descriptif
\ajoutancrage{
On introduira des notions de sécurité informatique (les ressources associées aux recommandations de l’ANSSI, CyberEdu, CyberMalveillance pourront servir de support)
}

% Contenus
\ajoutcontenudetaille{
* Protocole et adressage IPv4&6,
* Traduction d'adresses (NAT/PAT),
* Routage statique et routage dynamique (OSPF),
* TCP, UDP,
* Politiques de filtrage ACL
}

% Mots-clés
\ajoutmotscles{Plan d’adressage, routage état de lien, stratégies de filtrage,, CIDR, VLSM, agrégation de routes, IPv6, NDP}
