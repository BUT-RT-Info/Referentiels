%%%%%%%%%%%%%%%%%%%%%%%%%%%%%%%%%
% Ressources
%%%%%%%%%%%%%%%%%%%%%%%%%%%%%%%%%

\nouvelleressource{R106}{Architecture des systèmes numériques et informatiques}

\ajoutheures{24}{12}

%% Les compétences et les ACs
\ajoutcompetence{RT1-Administrer}{\niveauA}

\ajoutac{AC112}{Comprendre l'architecture des systèmes numériques et les principes du codage de l'information}

\ajoutcompetence{RT2-Connecter}{\niveauA}



\ajoutcompetence{RT3-Programmer}{\niveauA}

% Les SAE


% Les pre-requis


% Le descriptif
\ajoutancrage{
Cette ressource apporte le socle de connaissances et savoir-faire pour les compétences de cœur de métier "Administrer les réseaux et l'Internet" (RT1) et " Créer des outils et applications informatiques pour les R&T" (RT3)
 Les systèmes informatiques et numériques sont au cœur de la spécialité Réseaux
et Télécoms. Cette ressource vise tout d’abord à permettre la compréhension du codage et du stockage des données. Puis elle permet de comprendre de façon très fine le comportement interne des systèmes numériques avec notamment des notions de temps d’exécution. Enfin elle permettra aux étudiants de relier ces systèmes au monde extérieur.
}

% Contenus
\ajoutcontenudetaille{
Codage des nombres, des caractères, des images.
Fonctions logiques - Logique combinatoire et séquentielle  - Notion d’ALU.
Structure d’un processeur - Différents types de mémoires.
Périphériques et entrées-sorties. Exemples GPIO,  liaison série, ...
}

% Mots-clés
\ajoutmotscles{Nombres binaires - codage - Fonctions logiques - Processeur - ALU - }
