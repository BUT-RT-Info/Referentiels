%%%%%%%%%%%%%%%%%%%%%%%%%%%%%%%%%
% Ressources
%%%%%%%%%%%%%%%%%%%%%%%%%%%%%%%%%

\nouvelleressource{R208}{Analyse et traitement de données structurées}

\ajoutheures{16}{???}

%% Les compétences et les ACs
\ajoutcompetence{RT1-Administrer}{\niveauA}



\ajoutcompetence{RT2-Connecter}{\niveauA}



\ajoutcompetence{RT3-Programmer}{\niveauA}

% Les SAE
\ajoutsae{SAÉ23}{Mise en place d'une solution informatique pour l’entreprise}
\ajoutsae{SAÉ24}{Projet intégratif de S2}

% Les pre-requis


% Le descriptif
\ajoutancrage{
Le professionnel R&T est amené à développer différents outils informatiques à usage personnel ou interne à l’équipe (compétence RT3-Programmer). Ces outils peuvent traiter des données complexes, viser des fonctionnalités multiples et être développé en équipe : il est alors nécessaire - pour le professionnel R&T - de structurer son travail, tant sur les variables manipulant les données, les fichiers qui les sauvegardent ou les restaurent, que sur l’organisation (arborescence) de son projet. La ressource introduit ses éléments structurels en contribuant à l’acquisition des apprentissages critiques mentionnés précédemment.
}

% Contenus
\ajoutcontenudetaille{
* Structure d'un programme : arborescence de fichiers, modules et packages
* Contexte d’exécution : programme principal vs script
* Structure complexe de données :

	* listes 2D, tableaux associatifs/dictionnaires

	* notion de classes (instance, attributs, méthodes)

* Manipulation de fichiers avancée :

	* fichiers structurés (XML, CSV, JSON, YAML)

	* gestion de l’arborescence par le code

	* lecture/écriture de fichiers structurés

	* notion de sérialisation

	* notion de persistance des données

* Initiation aux expressions régulières
* Introduction au traitement des erreurs
L’utilisation de l’anglais est préconisée dans la documentation du code.
}

% Mots-clés
\ajoutmotscles{Algorithmes, langages de programmation, structure de données, méthodologie de développement, gestion de versions}
