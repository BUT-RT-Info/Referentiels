%%%%%%%%%%%%%%%%%%%%%%%%%%%%%%%%%
% Ressources
%%%%%%%%%%%%%%%%%%%%%%%%%%%%%%%%%

\nouvelleressource{R103}{Réseaux locaux et équipements actifs}

\ajoutheures{30}{16}

%% Les compétences et les ACs
\ajoutcompetence{RT1-Administrer}{\niveauA}

\ajoutac{AC113}{Configurer les fonctions de base du réseau local}
\ajoutac{AC115}{Identifier les dysfonctionnements du réseau local}
\ajoutac{AC116}{Installer un poste client}

\ajoutcompetence{RT2-Connecter}{\niveauA}

\ajoutac{AC213}{Déployer des supports de transmission}

\ajoutcompetence{RT3-Programmer}{\niveauA}

% Les SAE
\ajoutsae{SAÉ11}{Réseaux / cybersécurité / hygiène informatique}
\ajoutsae{SAÉ12}{Réseau d'entreprise ou personnel}

% Les pre-requis
\ajoutprerequis{R101}{Initiation aux réseaux informatiques}

% Le descriptif
\ajoutancrage{
--	Comprendre le fonctionnement des réseaux locaux basés sur la technologie Ethernet
--	Configurer les équipements actifs constituant les réseaux locaux
On introduira des notions de sécurité informatique (les ressources associées aux recommandations de l’ANSSI, CyberEdu, CyberMalveillance pourront servir de support)
}

% Contenus
\ajoutcontenudetaille{
* Câblage réseaux
* Différentes topologies physiques et logiques
* Normalisation Ethernet 802 (802.1, 802.2, 802.3)
* Commutation Ethernet : apprentissage des adresses MAC, diffusion, Broadcast.
* Différents équipements actifs : commutateur, routeur, etc...
* Configuration d'un réseau segmenté en VLAN, lien Multivlan, communication Intervlan,
* Redondance et détection de boucles dans un réseau commuté: STP, RSTP….
}

% Mots-clés
\ajoutmotscles{réseaux locaux, Ethernet, commutateurs, routeurs, VLAN, 802.1Q, 802.1P, STP, RSTP}
