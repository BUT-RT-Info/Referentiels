%%%%%%%%%%%%%%%%%%%%%%%%%%%%%%%%%
% Ressources
%%%%%%%%%%%%%%%%%%%%%%%%%%%%%%%%%

\nouvelleressource{R205}{Signaux et Systèmes pour les transmissions}

\ajoutheures{???}{???}

%% Les compétences et les ACs
\ajoutcompetence{RT1-Administrer}{\niveauA}

\ajoutac{AC111}{Maîtriser les lois fondamentales de l'électricité afin d'intervenir sur des équipements de réseaux et télécommunications}

\ajoutcompetence{RT2-Connecter}{\niveauA}

\ajoutac{AC211}{Mesurer et analyser les signaux}
\ajoutac{AC212}{Caractériser des systèmes de transmissions élémentaires et découvrir la modélisation mathématique de leur fonctionnement}
\ajoutac{AC213}{Déployer des supports de transmission}

\ajoutcompetence{RT3-Programmer}{\niveauA}

% Les SAE


% Les pre-requis


% Le descriptif
\ajoutancrage{
Cette ressource apporte le socle de connaissances et savoir-faire pour les compétences de cœur de métier "Administrer les réseaux et l'Internet" (RT1) et "Connecter les entreprises et les usagers" (RT2)
La caractérisation du comportement d’un système télécom en fonction de la fréquence permet au technicien d’appréhender la notion de bande passante et d’introduire celle de canal de transmission.
La représentation spectrale des signaux permet de comprendre quelles modifications ces signaux vont subir dans un système télécom.
}

% Contenus
\ajoutcontenudetaille{
Étude de la fonction de transfert d’un système linéaire  – Notion de filtrage – Réponse fréquentielle d’un support de transmission – Notion de bande passante. Atténuation, amplification des systèmes.
Représentations temporelles et fréquentielles des signaux - Analyse spectrale de signaux réels (exemples : audio, WiFi, ADSL, …).
Influence de la fonction de transfert d’un système sur un signal (exemples : audio, numérique, …)
Bilans de liaison de systèmes de transmissions
}

% Mots-clés
\ajoutmotscles{Représentations temporelles et fréquentielles des signaux - Fonction de transfert - Bande passante - Analyse spectrale}
