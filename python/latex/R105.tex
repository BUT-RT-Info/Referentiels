%%%%%%%%%%%%%%%%%%%%%%%%%%%%%%%%%
% Ressources
%%%%%%%%%%%%%%%%%%%%%%%%%%%%%%%%%

\nouvelleressource{R105}{Supports de transmission pour les réseaux}

\ajoutheures{12}{6}

%% Les compétences et les ACs
\ajoutcompetence{RT1-Administrer}{\niveauA}



\ajoutcompetence{RT2-Connecter}{\niveauA}

\ajoutac{AC211}{Mesurer et analyser les signaux}
\ajoutac{AC213}{Déployer des supports de transmission}

\ajoutcompetence{RT3-Programmer}{\niveauA}

% Les SAE


% Les pre-requis


% Le descriptif
\ajoutancrage{
Il s’agit d’étudier les concepts fondamentaux des supports de transmission.  
}

% Contenus
\ajoutcontenudetaille{
* Types de support de transmission (réseau d’entreprise, réseau opérateur)
* Caractéristiques d’un ou plusieurs types de supports (exemples: retard de propagation, atténuation, continuité, échos, bruit, perturbations, identifier un défaut, bande passante,... ) à partir de mesures et d’analyse des signaux
* Prolongement possible : recettage, certification LAN, ...
}

% Mots-clés
\ajoutmotscles{supports de transmission (fibre optique, cuivre, radio), mesures., }
