%%%%%%%%%%%%%%%%%%%%%%%%%%%%%%%%%
% Ressources
%%%%%%%%%%%%%%%%%%%%%%%%%%%%%%%%%

\nouvelleressource{R211}{Expression-Culture-Communication Professionnelles (ECC2)}

\ajoutheures{30}{21}

%% Les compétences et les ACs
\ajoutcompetence{RT1-Administrer}{\niveauA}



\ajoutcompetence{RT2-Connecter}{\niveauA}

\ajoutac{AC215}{Communiquer avec un client ou un collaborateur}

\ajoutcompetence{RT3-Programmer}{\niveauA}

% Les SAE
\ajoutsae{SAÉ21}{Construction d’un réseau informatique pour une petite structure}
\ajoutsae{SAÉ22}{Mesures et caractérisation d’un signal ou d’un système}
\ajoutsae{SAÉ23}{Mise en place d'une solution informatique pour l’entreprise}

% Les pre-requis


% Le descriptif
\ajoutancrage{

}

% Contenus
\ajoutcontenudetaille{
* Utiliser les outils et ressources documentaires de manière professionnelle
* Analyser et restituer des informations de façon synthétique
* S’initier au résumé
* Produire des écrits longs et clairs, structurés, adaptés au destinataire et répondant aux normes de présentation professionnelle et académique (dossier, présentation longue, exploitation de la mise en forme pour alléger les contenus et guider la lecture…)
* Réécrire et corriger ses documents ;
* Exploiter efficacement des outils de traitement de texte et de partage des données
* Renforcer les compétences linguistiques
* Élaborer un discours clair et efficace dans différents contextes
* Adapter sa communication verbale et nonverbale
* Comprendre une situation de communication complexe
* Savoir utiliser à bon escient des  outils multimédia pour une présentation orale
* Décrire et analyser l’image fixe et mobile
* Produire un document audiovisuel court
* Adopter des savoirêtre professionnels essentiels dans le travail en équipe (coopération, prise en compte de l’opinion d’autrui, adaptation, prise d’initiative...)
* S’initier à la gestion de projet : argumenter, défendre son point de vue
* Agir en cohérence avec les objectifs du développement durable
* Comprendre et s’approprier les enjeux du monde contemporain
Création de supports vidéo (film, tutoriel, notice) - outils de veille documentaire - critique des médias sociaux - participation à des actions culturelles - résumé - synthèse d’un document - débat - revue de presse
}

% Mots-clés
\ajoutmotscles{Synthèse, résumé, expression écrite, rédaction technique, expression orale, médias, culture générale, esprit critique, développement durable}
