%%%%%%%%%%%%%%%%%%%%%%%%%%%%%%%%%
% Ressources
%%%%%%%%%%%%%%%%%%%%%%%%%%%%%%%%%

\nouvelleressource{R209}{Initiation au développement Web}

\ajoutheures{24}{???}

%% Les compétences et les ACs
\ajoutcompetence{RT1-Administrer}{\niveauA}

\ajoutac{AC112}{Comprendre l'architecture des systèmes numériques et les principes du codage de l'information}
\ajoutac{AC114}{Maîtriser les rôles et les principes fondamentaux des systèmes d'exploitation afin d'interagir avec ceux-ci pour la configuration et administration des réseaux et services fournis}

\ajoutcompetence{RT2-Connecter}{\niveauA}



\ajoutcompetence{RT3-Programmer}{\niveauA}

% Les SAE
\ajoutsae{SAÉ23}{Mise en place d'une solution informatique pour l’entreprise}
\ajoutsae{SAÉ24}{Projet intégratif de S2}

% Les pre-requis
\ajoutprerequis{R107}{Fondamentaux de la programmation}
\ajoutprerequis{R109}{Introduction aux technologies Web}
\ajoutprerequis{R207}{Sources de données}
\ajoutprerequis{R208}{Analyse et traitement de données structurées}

% Le descriptif
\ajoutancrage{
Le professionnel R&T peut être amené à développer, pour ses besoins personnels ou pour ses collaborateurs, un site Web (compétence RT3-Programmer), par exemple pour fournir une interface de présentation du réseau informatique. 
Il doit en appréhender tous les éléments : il doit aussi bien connaître les protocoles de communication du Web que veiller à la sécurité de ceux-ci. Il doit également pouvoir accéder, traiter et afficher des informations provenant de différentes sources de données telles que des SGBD, des API ou des fichiers structurés. La présente ressource contribue de fait aux apprentissages critiques mentionnés précédemment.
}

% Contenus
\ajoutcontenudetaille{
* Introduction au protocole HTTP
* Mise en forme de pages Web :

	* Balises HTML avancées

	* Structure d’une page avec son DOM

	* CSS avancé ou Framework

	* Initiation au dynamisme côté client (JavaScript, jQuery, …)

* Scripts côté serveur
* Eléments d’interaction clientserveur (requête HTTP, URL, formulaire)
* Interrogation d'un SGBD ou d'une API
* Sensibilisation à la sécurisation de sites : failles XSS / XSS stockée / injections SQL
L’utilisation de l’anglais est préconisée dans la documentation du code.
}

% Mots-clés
\ajoutmotscles{Web, Développement, Algorithmes, SGBD, API, Sécurité, Environnement client-serveur}
