%%%%%%%%%%%%%%%%%%%%%%%%%%%%%%%%%
% Ressources
%%%%%%%%%%%%%%%%%%%%%%%%%%%%%%%%%

\nouvelleressource{R113}{Mathématiques du signal}

\ajoutheures{30}{6}

%% Les compétences et les ACs
\ajoutcompetence{RT1-Administrer}{\niveauA}

\ajoutac{AC111}{Maîtriser les lois fondamentales de l'électricité afin d'intervenir sur des équipements de réseaux et télécommunications}

\ajoutcompetence{RT2-Connecter}{\niveauA}

\ajoutac{AC211}{Mesurer et analyser les signaux}
\ajoutac{AC212}{Caractériser des systèmes de transmissions élémentaires et découvrir la modélisation mathématique de leur fonctionnement}

\ajoutcompetence{RT3-Programmer}{\niveauA}

% Les SAE
\ajoutsae{SAÉ13}{Supports de transmission / calculs}
\ajoutsae{SAÉ22}{Mesures et caractérisation d’un signal ou d’un système}

% Les pre-requis


% Le descriptif
\ajoutancrage{

}

% Contenus
\ajoutcontenudetaille{
* ) Introduction aux signaux

	*  graphe d’un signal

	*  symétries : parité, imparité, ...

	*  causalité, support temporel

	*  équation de droite

	*  fonction définie par morceaux (ex. : valeur absolue)

	*  signaux de base : (ex. : porte, triangle, échelon, rampe…)

	*  opérations sur les signaux : avance, retard, dilatation, amplification, offset
(interprétation géométrique sur le graphe), somme de signaux
Cette partie sera l’occasion de réviser les règles de calculs de base par l’intermédiaire du calcul d’images et d’antécédents.

		* ) Éléments de trigonométrie

	*  définition du radian

	*  cercle trigonométrique

	*  formules , , , et  et idem avec sinus

	*  angles remarquables

			* ) Signaux périodiques

	*  période, fréquence, pulsation

	*  signaux périodiques de base : créneau, dent de scie, sinus, cosinus...

	*  fréquence/période/pulsation d’un signal dilaté, d’une combinaison linéaire de
signaux périodiques

	*  graphe des signaux avancés, retardés, dilatés…

	*  graphe de  ,
}

% Mots-clés
\ajoutmotscles{Signaux, signaux périodiques}
