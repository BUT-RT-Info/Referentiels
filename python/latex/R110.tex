%%%%%%%%%%%%%%%%%%%%%%%%%%%%%%%%%
% Ressources
%%%%%%%%%%%%%%%%%%%%%%%%%%%%%%%%%

\nouvelleressource{R110}{Anglais de communication et initiation au vocabulaire technique}

\ajoutheures{30}{20}

%% Les compétences et les ACs
\ajoutcompetence{RT1-Administrer}{\niveauA}

\ajoutac{AC115}{Identifier les dysfonctionnements du réseau local}

\ajoutcompetence{RT2-Connecter}{\niveauA}



\ajoutcompetence{RT3-Programmer}{\niveauA}

% Les SAE
\ajoutsae{SAÉ11}{Réseaux / cybersécurité / hygiène informatique}
\ajoutsae{SAÉ12}{Réseau d'entreprise ou personnel}
\ajoutsae{SAÉ13}{Supports de transmission / calculs}
\ajoutsae{SAÉ14}{Se présenter sur Internet}

% Les pre-requis


% Le descriptif
\ajoutancrage{

}

% Contenus
\ajoutcontenudetaille{
Objectifs visés
* Développer sa confiance en soi
* Se présenter, présenter quelqu’un, interroger
* Savoir structurer son discours oral et écrit (courriel, conversation téléphonique, visioconférence…)
* Décrire, expliquer un élément technique
* Savoir écouter, comprendre et analyser les demandes de son interlocuteur, suggérer des solutions
* Reformuler, expliciter un message
* Appréhender le vocabulaire technique des domaines cibles
*  Extensions possibles : télécollaboration, télétandem.
}

% Mots-clés
\ajoutmotscles{Anglais général et technique, situations de communication, expression et compréhension.}
