\phantomsection \label{subsubsec:#codelatex}
\addcontentsline{toc}{subsubsection}{#codelatex}

% Le tableau complet présentant une ressource

\setlength{\tabcolsep}{0.125cm} % Marge des colonnes
\setlength{\extrarowheight}{2pt} % Marge des lignes

\arrayrulecolor{ressourceC}

% 1er tableau : Nom/Code/Heures
\begin{tabular}[t]{|P|T|}
\hline % 1ère ligne : code & titre
	\cellcolor{ressourceC} \textcolor{compCAp!20!white}{\bfseries \hypertarget{#codelatex}{#code}}
    & \cellcolor{ressourceC} \textcolor{compCAp!20!white}{\bfseries #nom (#codeRT)}
\\
\end{tabular}

\begin{tabular}[t]{|P|Q|V|}
\hline % 2ème ligne : semestre
	\textcolor{ressourceC}{\bfseries Semestre}
	& \multicolumn{2}{l|}{ #semestre } \\
\hline % 3ème ligne : Heures
\hline
    \textcolor{ressourceC}{\bfseries Heures}
    &
    \textcolor{ressourceC}{\bfseries Formation encadrée}
    & {#heures_formation}h, dont {#heures_tp}h de TP \\
\hline
\hline % 4ème ligne : Parcours
	\textcolor{ressourceC}{\bfseries Parcours}
	& \multicolumn{2}{l|}{#parcours} \\
\hline
\end{tabular}

% Description
\begin{tabular}{|G|}
	\hline
	\textcolor{ressourceC}{\bfseries Description} \\
	\hline
	 #contexte
	\\
\hline
\end{tabular}

% Contenu
\begin{tabular}{|G|}
	\textcolor{ressourceC}{\bfseries Contenu} \\
	\hline
    #contenu
	\\
\end{tabular}

% 2ème tableau : Compétences et apprentissages critiques
\begin{tabular}[t]{|P|T|}
\hline
    \begin{tabular}[t]{@{}P@{}}
        \bfseries \textcolor{ressourceC}{Compétences /} \tabularnewline
        \bfseries \textcolor{ressourceC}{Apprentissages} \tabularnewline
        \bfseries \textcolor{ressourceC}{critiques}
    \end{tabular}
    &
    #competences_et_acs
    \\
\hline
\end{tabular}
% \ifcsdef{Rcoeff\CODE A}{coef. {\csname Rcoeff\CODE A\endcsname}}{}

% SAE
\begin{tabular}[t]{|P|T|}
\hline
    \textcolor{ressourceC}{\bfseries SAÉ concernées }
    &
    #listeSAE %\listeSAE{\CODE}
    \\
\hline
\end{tabular}

% Pré-requis
\begin{tabular}[t]{|P|T|}
\hline
    \textcolor{ressourceC}{\bfseries Prérequis}
    &
    #listePreRequis %\listePrerequis{\CODE}
    \\
\hline
\end{tabular}

% Mots-clés
\begin{tabular}[t]{|P|T|}
\hline
    \textcolor{ressourceC}{\bfseries Mots-clés}
    &
    #motsCles % {\csname Rmotscles\CODE\endcsname}
    \\
\hline

\end{tabular}
