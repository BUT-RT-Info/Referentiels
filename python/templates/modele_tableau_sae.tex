% Le tableau complet présentant une ressource

\setlength{\tabcolsep}{0.125cm} % Marge des colonnes
\setlength{\extrarowheight}{2pt} % Marge des lignes

\arrayrulecolor{saeC}

% 1er tableau : Nom/Code/Heures
\begin{tabular}[t]{|P|T|}
\hline % 1ère ligne : code & titre
	\cellcolor{saeC} \textcolor{compCAp!20!white}{\bfseries \hypertarget{#codelatex}{#code}}
    & \cellcolor{saeC} \textcolor{compCAp!20!white}{\bfseries #nom (#codeRT)}
\\
\end{tabular}

\begin{tabular}[t]{|P|Q|V|}
\hline % 2ème ligne : semestre
	\textcolor{saeC}{\bfseries Semestre}
	& \multicolumn{2}{l|}{ #semestre } \\
\hline % 3ème ligne : Heures
\hline
    \textcolor{saeC}{\bfseries Heures}
    &
    \textcolor{saeC}{\bfseries Formation encadrée} & {#heures_formation}h, dont {#heures_tp}h de TP \\
\cline{2-3}
	& \textcolor{saeC}{\bfseries Projet} & {#heures_projet}h \\
\hline
\hline % 4ème ligne : Parcours
	\textcolor{saeC}{\bfseries Parcours}
	& \multicolumn{2}{l|}{#parcours} \\
\hline
\end{tabular}

% Description
\begin{tabular}{|G|}
	\hline
	\textcolor{saeC}{\bfseries Objectifs et problématique professionnelle} \\
	\hline
	 #objectifs
	\\
\hline
\end{tabular}

% Contenu
\begin{tabular}{|G|}
	\textcolor{saeC}{\bfseries Description générique} \\
	\hline
    #description
	\\
    #livrables
\end{tabular}

% 2ème tableau : Compétences et apprentissages critiques
\begin{tabular}[t]{|P|T|}
\hline
    \begin{tabular}[t]{@{}P@{}}
        \bfseries \textcolor{saeC}{Compétences /} \tabularnewline
        \bfseries \textcolor{saeC}{Apprentissages} \tabularnewline
        \bfseries \textcolor{saeC}{critiques}
    \end{tabular}
    &
    #competences_et_acs
    \\
\hline
\end{tabular}
% \ifcsdef{Rcoeff\CODE A}{coef. {\csname Rcoeff\CODE A\endcsname}}{}

% SAE
\begin{tabular}[t]{|P|T|}
\hline
    \textcolor{saeC}{\bfseries Ressources combinées }
    &
    #listeRessources %\listeSAE{\CODE}
    \\
\hline
\end{tabular}


% Mots-clés
\begin{tabular}[t]{|P|T|}
\hline
    \textcolor{saeC}{\bfseries Mots-clés}
    &
    #motsCles % {\csname Rmotscles\CODE\endcsname}
    \\
\hline
\end{tabular}

% Mots-clés
\begin{tabular}[t]{|P|T|}
\hline
        \begin{tabular}[t]{@{}P@{}}
        \bfseries \textcolor{saeC}{Exemples de} \tabularnewline
        \bfseries \textcolor{saeC}{mise en \oe{}uvre}
    \end{tabular}
    &
     % {\csname Rmotscles\CODE\endcsname}
    \\
\hline
\end{tabular}
