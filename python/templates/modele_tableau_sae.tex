
\phantomsection \label{subsubsec:#codelatex}
\addcontentsline{toc}{subsubsection}{#codelatex}

% Le tableau complet présentant une ressource

\setlength{\tabcolsep}{0.125cm} % Marge des colonnes
\setlength{\extrarowheight}{2pt} % Marge des lignes

\arrayrulecolor{saeC}

% 1er tableau : Nom/Code/Heures
\begin{tabular}[t]{|P|T|}
\hline % 1ère ligne : code & titre
	\cellcolor{saeC} \textcolor{compCAp!20!white}{\bfseries \hypertarget{#codelatex}{#code}}
    & \cellcolor{saeC} \textcolor{compCAp!20!white}{\bfseries #codeRT : #nom}
\\
\end{tabular}

\begin{tabular}[t]{|P|T|}
\hline % 2ème ligne : semestre
	\textcolor{saeC}{\bfseries Cursus}
	& #cursus  \\
\hline % 3ème ligne : Heures
\hline
    \textcolor{saeC}{\bfseries Heures}
    &
    \textcolor{saeC}{\bfseries Formation encadrée publiée au PN} : {\bfseries {#heures_formation_pn}h}, dont {#heures_tp_pn}h de TP \\ % dont {#heures_cm}h de CM, {#heures_td}h de TD,
\cline{2-2}
    &
    \textcolor{saeC}{\bfseries Formation encadrée préconisée en adaptation locale} : {\bfseries {#heures_formation}h}, dont {#heures_tp}h de TP \\
\cline{2-2}
	& \textcolor{saeC}{\bfseries Projet} : {#heures_projet}h \\
\hline
\hline % 4ème ligne : Parcours
	\textcolor{saeC}{\bfseries Parcours}
	& #parcours \\
\hline
\end{tabular}

% Description
\begin{tabular}{|G|}
	\hline
	\textcolor{saeC}{\bfseries Objectifs et problématique professionnelle} \\
	\hline
	 #objectifs
	\\
\hline
\end{tabular}

% Contenu
\begin{tabular}{|G|}
	\textcolor{saeC}{\bfseries Description générique} \\
	\hline
    #description \\
\hline
\end{tabular}

% 2ème tableau : Compétences et apprentissages critiques
\begin{tabular}[t]{|P|T|}
\hline
    \begin{tabular}[t]{@{}P@{}}
        \bfseries \textcolor{saeC}{Compétences /} \tabularnewline
        \bfseries \textcolor{saeC}{Apprentissages} \tabularnewline
        \bfseries \textcolor{saeC}{critiques}
    \end{tabular}
    &
    #competences_et_acs \\
\hline
\end{tabular}
% \ifcsdef{Rcoeff\CODE A}{coef. {\csname Rcoeff\CODE A\endcsname}}{}

% SAE
\begin{tabular}[t]{|P|T|}
\hline
    \textcolor{saeC}{\bfseries Ressources combinées }
    &
    #listeRessources %\listeSAE{\CODE}
    \\
\hline
\end{tabular}




