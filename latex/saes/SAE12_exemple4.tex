%%%%%%%%%%%%%%%%%%%%%%%%%%%%%%%%%
% Exemple de SAE
%%%%%%%%%%%%%%%%%%%%%%%%%%%%%%%%%

\nouvelexemple{Configuration d'un RaspberryPi connecté}

\ajoutESproblematique{Il s'agit de configurer un RaspberryPI pour que l'on puisse le
programmer sans avoir à toujours y brancher un écran et un clavier mais
en y accédant simplement en \textabbrv{SSH}, peu important le réseau sur lequel il
est connecté. On peut placer sur le Raspberry une \textabbrv{LED} qui s'allumerait
et pour les plus avancés une photoRésistance (avec un pont diviseur
entre photorésistance et une résistance de 10kOhms) pour récupérer la
valeur de la luminosité ambiante.}

\ajoutESdescription{
La SAÉ va servir de base pour de futures SAE plus poussées en IoT ou en
réseaux.\\
Les étapes à réaliser seraient les suivantes~:
\begin{itemize}[topsep=5pt]
\item
  Connecter un Raspberry PI avec un clavier, un écran et récupérer son
  adresse \textabbrv{MAC}.
\item
  Changer le login et le mot de passe par défaut.
\item
  Activer \textabbrv{SSH} sur le Raspberry PI. Récupérer l'adresse \textabbrv{IP} en \textabbrv{DHCP}. Voir
  comment accéder en \textabbrv{SSH}.
\item
  Débrancher le Raspberry PI et le brancher sur un autre réseau
  ailleurs. Puis utiliser les commandes
  ssh pour retrouver l'adresse de la carte sur le nouveau réseau et
  communiquer avec elle.
\item
  Utiliser via \textabbrv{SSH} le Raspberry PI pour allumer la \textabbrv{LED} qui est connectée
  et pour les plus avancés récupérer
  la valeur du capteur (Photorésistance) pour évaluer la luminosité dans
  la pièce où on a placé la carte.
\end{itemize}
}

\ajoutESformes{TP, projet}



\ajoutESmodalite{\vspace{-5pt}
\begin{itemize}[topsep=5pt]
\item
  Compte-rendu
\item
  et/ou vidéo de démonstration.
\end{itemize}}