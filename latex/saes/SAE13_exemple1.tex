%%%%%%%%%%%%%%%%%%%%%%%%%%%%%%%%%
% Exemple de SAE
%%%%%%%%%%%%%%%%%%%%%%%%%%%%%%%%%

\nouvelexemple{Caractériser des supports de transmission}

\ajoutESproblematique{L'entreprise demande à l'étudiant de~:
\begin{itemize}[topsep=5pt]
\item
  savoir lire un document technique de mesures\,;
\item
  analyser un schéma/une structure de câblage\,;
\item
  de caractériser un support de transmission par différentes mesures
  (retard de propagation, atténuation,
  continuité, échos, bruit, perturbations, \ldots) en vu d'identifier un
  défaut, voire de certifier un câblage
  \textabbrv{LAN} et de comprendre les principaux critères de choix d'un support
\item
  savoir présenter des résultats à un client ou un collaborateur.
\end{itemize}
L'étudiant saura mobiliser les compétences pour réaliser des mesures ou
pour identifier et caractériser un support et savoir rédiger un
compte-rendu de mesure.\\[3pt]}

\ajoutESdescription{
L'étudiant devra s'appuyer sur ses connaissances, notamment les concepts
fondamentaux de l'étude des supports de transmissions dans les réseaux,
les concepts fondamentaux des systèmes électroniques, le vocabulaire en
architecture des réseaux numériques, des concepts mathématiques pour les
signaux de base, pour les calculs de puissance, d'atténuation.\\
Le support pourra être~:
\begin{itemize}[topsep=5pt]
\item
  le cuivre, avec pour mesures envisagées~: des mesures temporelles
  (échelon, sinus), le retard de propagation,
  l'atténuation, les échos (réflexion), les perturbations, supposant
  l'utilisation de \textabbrv{GBF}, d'oscilloscope
  et de câbles\,;
\item
  la fibre optique, avec pour outils envisagés~: la soudure, le crayon
  optique, la sonde d'inspection,la photométrie\,;
\item
  une liaison radio, les mesures traitant de l'atténuation ou des effets
  d'interférence.
\end{itemize}
L'étudiant devra~:
\begin{itemize}[topsep=5pt]
\item
  lire des documents techniques de support de transmission\,;
\item
  déterminer les types de mesures et les types de signaux nécessaires
  pour caractériser les supports
  et estimer les résultats attendus\,;
\item
  paramétrer les outils de mesure\,;
\item
  réaliser des mesures\,;
\item
  analyser et exploiter des résultats de tests.
\end{itemize}
Exemples de mise en oeuvre~:
\begin{itemize}[topsep=5pt]
\item
  Mesure temporelle (échelon, sinus), retard de propagation,
  atténuation, échos (réflexion), perturbations,
  (\textabbrv{GBF}, oscillo, câble)\,;
\item
  Vérifier la conformité par rapport à un cahier des charges, une norme,
  ou une réglementation, comme
  par exemple le schéma de câblage avec vérification de la continuité du
  support, mesure de longueur,
  d'atténuation, \ldots{}
\item
  Vérifier la conformité des mesures\,;
\item
  Diagnostiquer des anomalies et proposer une reprise du câblage, un
  changement du support.
\end{itemize}
}

\ajoutESformes{Mini-projet en binôme associant un TP long, encadré par un enseignant et
des heures non encadrées pour, par exemple, la préparation du TP puis
pour la rédaction du compte rendu.\\[3pt]}



\ajoutESmodalite{L'étudiant doit être capable de rédiger un compte-rendu de mesure avec
explications.\\
On pourra s'appuyer sur~:
\begin{itemize}[topsep=5pt]
\item
  dossier ou rapport d'étude (compte-rendu)\,;
\item
  rapport de mesures\,;
\item
  \textabbrv{QCM} sur les mesures\,;
\item
  grille de suivi du travail\,;
\item
  présentation orale des mesures réalisées.
\end{itemize}}