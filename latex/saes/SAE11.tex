%%%%%%%%%%%%%%%%%%%%%%%%%%%%%%%%%
% SAE
%%%%%%%%%%%%%%%%%%%%%%%%%%%%%%%%%

\nouvellesae{SAÉ11}{Sensibilisation à l'hygiène informatique et à la cybersécurité}

\ajoutSheures{7}{5}{12}


%% Les compétences et les ACs
\ajoutScompetence{RT1-Administrer}{\niveauA}

\ajoutSac{AC0112}{Comprendre l'architecture des systèmes numériques et les principes du codage de l'information}
\ajoutSac{AC0114}{Maîtriser les rôles et les principes fondamentaux des systèmes d'exploitation afin d'interagir avec ceux-ci pour la configuration et administration des réseaux et services fournis}
\ajoutSac{AC0115}{Identifier les dysfonctionnements du réseau local}

\ajoutScompetence{RT2-Connecter}{\niveauA}



\ajoutScompetence{RT3-Programmer}{\niveauA}



% Le description
\ajoutSdescription{
Il s'agit de faire prendre conscience aux étudiants les risques
potentiels pris par l'usager d'un environnement numérique et de leur
fournir les réflexes afin de devenir un usager conscient, averti et
responsable. L'hygiène informatique et les bonnes pratiques de l'usage
du numérique sont des connaissances que doivent maîtriser et appliquer
les étudiants avant d'aller en stage/alternance en entreprise, où ils
devront respecter la charte informatique imposée par la DSI. A plus long
terme, en tant que professionnels des services informatiques de
l'entreprise, ils auront à leur tour à charge de sensibiliser les
utilisateurs et de leur faire connaître et accepter la charte de bon
usage des moyens informatiques.
}

% Les ressources
\ajoutSressources{R101}{Initiation aux réseaux informatiques}

% Livrable
\ajoutSlivrables{
Rapport d'analyse des risques numériques et présentation diaporama ou
conception d'une courte vidéo de sensibilisation (style 180'\,') à
destination des proches ou de la famille de l'étudiant.
}

% Mots-clés
\ajoutSmotscles{Sécurité numérique, utilisation d'internet}
