%%%%%%%%%%%%%%%%%%%%%%%%%%%%%%%%%
% Exemple de SAE
%%%%%%%%%%%%%%%%%%%%%%%%%%%%%%%%%

\nouvelexemple{Construire un réseau d'entreprise avec des services de base et des règles de sécurité}

\ajoutESproblematique{L'étudiant se met dans le rôle d'une société locale de services du
numérique, il est contacté par une entreprise qui lui passe une commande
: déployer un réseau d'entreprise dans le nouveau siège de la société.\\
Sa mission consiste à~:
\begin{itemize}[topsep=5pt]
\item
  Proposer et mettre en place une infrastructure de réseau d'entreprise
  dotée de services réseaux et
  de règles de sécurité\,;
\item
  Mettre en place une \textabbrv{DMZ} pour héberger les serveurs publics de
  l'entreprise\,;
\item
  Configurer l'accès à Internet\,;
\item
  Mettre en place des règles de sécurité\,;
\item
  Produire une documentation technique sur la solution déployée.
\end{itemize}}

\ajoutESdescription{
L'objectif de cette SAE est de mettre en œuvre un réseau d'entreprise
basé sur une architecture segmentée en \textabbrv{VLAN} qui intègre différents
services réseaux. Certaines applications serveur (\textabbrv{HTTP}, \textabbrv{DHCP}, \textabbrv{SSH})
seront à installer et à configurer par l'étudiant et d'autres seront
préalablement configurées (\textabbrv{DNS} notamment).\\
Le concept de \textabbrv{DMZ} sera abordé et les mécanismes de translation d'adresse
statique et dynamique seront traités. L'initiation au filtrage de flux
sera également amenée au travers de cette séquence d'apprentissage.\\
Un outil d'émulation est préférable à une solution matérielle afin que
l'exercice puisse être construit progressivement au fur et à mesure des
séances. Les logiciels d'émulation comme GNS3, VIRL, EVE-NG peuvent être
utilisés et complétés avec VirtualBox ou \textabbrv{VM}ware. Il faut aussi que ces
outils soient mis à disposition des étudiants pour qu'ils puissent
continuer le projet en dehors des heures encadrées.\\
L'étudiant doit adopter une approche de type projet et découper son
travail en tâches. Il devra valider chaque étape par des tests adaptés
avant de passer à la suivante. Les résultats (fichiers de logs,
résultats de commandes, acquisition de trames\ldots) obtenus devront
être justifiés.\\
L'infrastructure réseau est volontairement simple afin que l'étudiant
puisse se concentrer sur des concepts fondamentaux. Cette architecture
est constituée de 2 commutateurs d'accès (L2), d'un commutateur de
distribution (L3) qui assure le routage inter-vlan et d'un routeur
passerelle qui fait office de pare-feu. Les services réseaux \textabbrv{HTTP}, DNS,
\textabbrv{DHCP} et \textabbrv{SSH} sont installés sur des machines virtuelles.\\
A partir du cahier des charges fourni, l'étudiant sera amené à réaliser
différentes activités dont voici quelques exemples~:
\begin{itemize}[topsep=5pt]
\item
  Plan d'adressage
\item
  Création des \textabbrv{VLAN}
\item
  Routage inter-\textabbrv{VLAN}
\item
  Mise en place de \textabbrv{VM}
\item
  Accès à Internet
\item
  Configuration du serveur \textabbrv{DHCP}, \textabbrv{SSH} et \textabbrv{HTTP}
\item
  Ajout d'entrées au serveur \textabbrv{DNS}
\item
  Configuration du pare-feu (une règle de filtrage)
\end{itemize}
}

\ajoutESformes{De façon individuelle ou collective, sur des heures encadrées et non
encadrées, l'étudiant ou l'équipe, sera confronté aux formes
pédagogiques suivantes~:
\begin{itemize}[topsep=5pt]
\item
  Création d'un réseau segmenté en \textabbrv{VLAN} avec mise en place d'une \textabbrv{DMZ} et
  de services réseaux\,;
\item
  Élaboration d'une méthode efficace pour tester progressivement la
  configuration réalisée\,;
\item
  Utilisation d'outils de diagnostics et analyse des résultats\,;
\item
  Rédaction de fiches opératoires (notice d'utilisation).
\end{itemize}}



\ajoutESmodalite{\vspace{-5pt}
\begin{itemize}[topsep=5pt]
\item
  Réseau entreprise~: opérationnel répondant aux problématiques
  suivantes~:
  \begin{itemize}
  \item
    un utilisateur interne à l'entreprise peut-il bénéficier des
    services \textabbrv{HTTP}, \textabbrv{DNS} et \textabbrv{DHCP} internes
    et se connecter à Internet.
  \item
    un utilisateur lambda peut-il accéder au serveur \textabbrv{HTTP} de
    l'entreprise depuis Internet ?
  \item
    l'administrateur réseau de l'entreprise peut-il gérer les
    équipements à l'aide d'une connexion \textabbrv{SSH}
    dans l'entreprise et hors de l'entreprise ?
  \end{itemize}
\item
  Documentation~: les procédures sont-elles applicables par une autre
  équipe ?
\item
  Réunions~: présentation finale (soutenance) de la solution mise en
  place.
\end{itemize}}