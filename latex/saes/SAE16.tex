%%%%%%%%%%%%%%%%%%%%%%%%%%%%%%%%%
% SAE
%%%%%%%%%%%%%%%%%%%%%%%%%%%%%%%%%

\nouvellesae{SAÉ16}{Portfolio}

\ajoutSheures{3}{3}{0}


%% Les compétences et les ACs
\ajoutScompetence{RT1-Administrer}{\niveauA}

\ajoutScoeff{0}

\ajoutSac{AC0111}{Maîtriser les lois fondamentales de l'électricité afin d'intervenir sur des équipements de réseaux et télécommunications}
\ajoutSac{AC0112}{Comprendre l'architecture des systèmes numériques et les principes du codage de l'information}
\ajoutSac{AC0113}{Configurer les fonctions de base du réseau local}
\ajoutSac{AC0114}{Maîtriser les rôles et les principes fondamentaux des systèmes d'exploitation afin d'interagir avec ceux-ci pour la configuration et administration des réseaux et services fournis}
\ajoutSac{AC0115}{Identifier les dysfonctionnements du réseau local}
\ajoutSac{AC0116}{Installer un poste client}

\ajoutScompetence{RT2-Connecter}{\niveauA}

\ajoutScoeff{0}

\ajoutSac{AC0211}{Mesurer et analyser les signaux}
\ajoutSac{AC0212}{Caractériser des systèmes de transmissions élémentaires et découvrir la modélisation mathématique de leur fonctionnement}
\ajoutSac{AC0213}{Déployer des supports de transmission}
\ajoutSac{AC0214}{Connecter les systèmes de \textabbrv{ToIP}}
\ajoutSac{AC0215}{Communiquer avec un client ou un collaborateur}

\ajoutScompetence{RT3-Programmer}{\niveauA}

\ajoutScoeff{0}

\ajoutSac{AC0311}{Utiliser un système informatique et ses outils}
\ajoutSac{AC0312}{Lire, exécuter, corriger et modifier un programme}
\ajoutSac{AC0313}{Traduire un algorithme, dans un langage et pour un environnement donné}
\ajoutSac{AC0314}{Connaître l'architecture et les technologies d'un site Web}
\ajoutSac{AC0315}{Choisir les mécanismes de gestion de données adaptés au développement de l'outil}
\ajoutSac{AC0316}{S'intégrer dans un environnement propice au développement et au travail collaboratif}

% Le description
\ajoutSdescription{Au sein d'un dossier et quels qu'en soient la forme, l'outil ou le
support, l'objectif d'un portfolio est de permettre à l'apprenant
d'adopter une posture qui, loin d'être déclarative, est fondamentalement
réflexive et critique vis-à-vis des compétences acquises ou en voie
d'acquisition. Autrement dit, au sein du portfolio, l'apprenant
documente et analyse sa trajectoire de développement en mobilisant des
traces, des preuves issues de l'ensemble des mises en situation
professionnelle (SAÉ) qu'il a vécues. Il pourra s'appuyer sur les outils
portfolio mis en place par l'établissement~: carnet papier, document
bureautique ou logiciel dédié.\\
Le portfolio est un élément structurant des formations en Approche Par
Compétence (\textabbrv{APC}). En effet, le portfolio~:
\begin{itemize}
\item
  soutient l'apprentissage par la constitution d'un dossier de traces
  (échantillon de preuves, sélectionnées
  par l'étudiant dans le but de rendre compte d'apprentissages
  aboutissant à la maîtrise progressive d'un
  domaine de compétences)\,;
\item
  permet la validation et la certification de savoir-agir complexes tout
  au long du parcours de formation\,;
\item
  favorise l'auto-détermination du parcours de formation de l'étudiant
  et accompagne son parcours d'insertion
  professionnelle (permet également de cultiver son identité numérique à
  savoir la présentation et le
  choix de rendre public des documents sur soi).
\end{itemize}
En outre, en tant qu'il suppose un engagement de la part de l'apprenant
lui-même, le portfolio soutient le développement des compétences et
l'individualisation du cursus de formation.\\
Aussi le portfolio est-il fondamentalement à penser comme un processus
continu porté par chaque apprenant au cours duquel il prend pleinement
conscience de ce qu'il a ou non acquis, et des ajustements nécessaires à
opérer au regard du référentiel de compétences et des objectifs de la
formation.\\
Consistant en une analyse réflexive des mises en situation
professionnelle vécues (SAÉ), le portfolio nécessite la mobilisation et
la combinaison de ressources telles que l'expression et la
communication. Et parce que cette démarche portfolio repose sur une
démonstration par l'apprenant de sa professionnalisation, le portfolio
s'appuie nécessairement sur le \textabbrv{PPP} en tant que ressource.\\
Aussi, parallèlement à ses objectifs traditionnels issus de l'expérience
acquise dans le cadre du DUT, le \textabbrv{PPP} devra, tel un fil conducteur,
permettre à l'étudiant d'être guidé dans la compréhension et
l'appropriation de son cursus de formation, ainsi que dans la
méthodologie d'écriture du portfolio.}

% Les ressources
\ajoutSressources{R111}{Expression-Culture-Communication Professionnelles 1}
\ajoutSressources{R112}{Projet Personnel et Professionnel}

% Livrable
\ajoutSlivrables{
Portfolio~: ensemble de traces et de preuves de l'acquisition des
compétences.
}

% Mots-clés
\ajoutSmotscles{Portfolio, Compétences.}
