%%%%%%%%%%%%%%%%%%%%%%%%%%%%%%%%%
% Exemple de SAE
%%%%%%%%%%%%%%%%%%%%%%%%%%%%%%%%%

\nouvelexemple{Comment découvrir mon réseau informatique ?}

\ajoutESproblematique{Il s'agit de donner la méthodologie de travail pour faire découvrir le
réseau informatique de chaque étudiant. On s'appuiera sur les concepts
fondamentaux des systèmes d'exploitation, de vocabulaire en
réseaux-informatiques, des protocoles réseaux et des outils logiciel
réseau de base. L'étudiant s'intéressera également à recenser les
caractéristiques de consommation d'énergie des équipements du réseau.}

\ajoutESdescription{
Il s'agit de comprendre l'agencement des briques réseaux (accès au
réseau Internet, Box en général), des équipements (routeur, switch,
firewall, WiFi), des terminaux (\textabbrv{PC}, smartphone, imprimantes, consoles de
jeux, media-center, \textabbrv{NAS}, \ldots) et des protocoles (\textabbrv{IP}, \textabbrv{DHCP}, \textabbrv{DNS}, Mail,
Web, \ldots) qui permettent leur fonctionnement.\\
On peut s'appuyer sur le réseau domestique de chaque étudiant avec une
Box d'accès Internet (\textabbrv{xDSL}, \textabbrv{FO}), ou bien une maquette TP avec un routeur
de sortie accédant à Internet via les ressources informatiques du
département.On peut caractériser simplement l'adressage \textabbrv{IP}v4 dynamique,
le masque de sous-réseaux, la passerelle par défaut, les serveurs \textabbrv{DNS}.
On peut également faire paramétrer un adressage \textabbrv{IP}v4 statique sur un
poste client.On peut s'appuyer sur les commandes de base~: ipconfig,
ifconfig, ip, ping, arp, traceroute, arp-scan (ArpCacheWatch sous
Windows) pour lister les adresses \textabbrv{MAC} présentes dans le réseau local.\\
On peut faire découvrir les outils pour connaître son adresse \textabbrv{IP}
Publique~: \url{https://github.com/VREMSoftwareDevelopment/WiFiAnalyzer.}\\
Enfin, on pourra initier les étudiants les plus avancés à l'usage de la
distribution Linux Kali en \textabbrv{VM} avec l'outil nmap pour découvrir (en
interne) les ports ouverts sur les équipements du réseau local
domestique. Bien expliquer que l'usage de cet outil de test de
pénétration doit être réalisé en respectant l'éthique.
}

\ajoutESformes{TP, projet}



\ajoutESmodalite{L'étudiant doit être capable de retranscrire son réseau grâce à un outil
de schéma réseau et de le détailler avec l'ensemble des observations
relevées. Il doit être capable de produire une fiche technique de
recette ou une présentation type powerpoint. Un bilan de consommation
énergétique des équipements du réseau doit être également fourni, il
peut être complété par les informations provenant d'un compteur
électrique intelligent.\\
On peut utiliser un logiciel de dessin technique (type
\url{https://app.diagrams.net,} Microsoft Visio ou Lucidchart) pour les
schémas réseaux en utilisant à bon escient les symboles et pictogrammes
(switch, routeur, firewall, WiFi, \ldots).}