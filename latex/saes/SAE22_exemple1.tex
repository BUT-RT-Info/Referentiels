%%%%%%%%%%%%%%%%%%%%%%%%%%%%%%%%%
% Exemple de SAE
%%%%%%%%%%%%%%%%%%%%%%%%%%%%%%%%%

\nouvelexemple{Analyse de lignes \textabbrv{ADSL}}

\ajoutESproblematique{Les lignes \textabbrv{ADSL} peuvent être perturbées par des signaux parasites créés
par exemple par une alimentation défectueuse ou un moteur d'ascenseur
avec des problèmes de \textabbrv{CEM}. Ces perturbateurs peuvent entraîner une
déconnexion totale d'une ligne \textabbrv{ADSL} voire de toutes les lignes d'un
immeuble. L'analyse spectrale et la recherche de ces perturbateurs est
donc une fonction du maintien en condition opérationnelle de ces lignes.}

\ajoutESdescription{
La SAE porte sur l'analyse d'un signal \textabbrv{ADSL} dans le domaine spectral et
pourra comporter jusqu'à 5 parties~:
\begin{itemize}
\item
  Affichage de la \textabbrv{FFT} d'un signal \textabbrv{ADSL} ou \textabbrv{ADSL}2+. On donnera par exemple
  le signal sous forme d'un fichier
  Excel (une colonne pour le temps et une pour l'amplitude, soit 2
  vecteurs) et l'étudiant devra afficher
  la \textabbrv{FFT}.
\item
  Détermination de la largeur de la bande montante et descendante
  (changement de valeur de la \textabbrv{DSP}).
\item
  Détermination de la norme \textabbrv{ADSL} ou \textabbrv{ADSL}2+ (en fonction de la largeur de
  bande descendante).
\item
  Calcul de puissance de la bande montante et descendante (intégration
  de la \textabbrv{DSP}).
\item
  Recherche d'un perturbateur électromagnétique~: on donnera une autre
  capture avec un perturbateur
  sinusoïdal (soit une raie en fréquence), il faudra détecter la
  fréquence de ce perturbateur dans le
  spectre.
\end{itemize}
Elle pourra s'appuyer sur une librairie pour calculer la \textabbrv{FFT} (par
exemple numpy en Python) et une autre pour l'affichage d'une courbe (par
exemple matplotlib en Python)
}

\ajoutESformes{TP et projet semi-autonome.}



\ajoutESmodalite{Présentation des résultats dans un notebook Python, avec explications
claires prouvant la compréhension.}