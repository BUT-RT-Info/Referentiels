%%%%%%%%%%%%%%%%%%%%%%%%%%%%%%%%%
% SAE
%%%%%%%%%%%%%%%%%%%%%%%%%%%%%%%%%

\nouvellesae{SAÉ23}{Mettre en place une solution informatique pour l'entreprise}

\ajoutSheures{17}{14}{12}


%% Les compétences et les ACs
\ajoutScompetence{RT1-Administrer}{\niveauA}





\ajoutScompetence{RT2-Connecter}{\niveauA}





\ajoutScompetence{RT3-Programmer}{\niveauA}

\ajoutScoeff{18}

\ajoutSac{AC0311}{Utiliser un système informatique et ses outils}
\ajoutSac{AC0312}{Lire, exécuter, corriger et modifier un programme}
\ajoutSac{AC0313}{Traduire un algorithme, dans un langage et pour un environnement donné}
\ajoutSac{AC0314}{Connaître l'architecture et les technologies d'un site Web}
\ajoutSac{AC0315}{Choisir les mécanismes de gestion de données adaptés au développement de l'outil}
\ajoutSac{AC0316}{S'intégrer dans un environnement propice au développement et au travail collaboratif}

% Le description
\ajoutSdescription{Puisqu'il est au cœur du système d'information de l'entreprise, le
professionnel R\&T peut être amené à développer différentes solutions
informatiques~: ces solutions peuvent faciliter son travail quotidien
(outil pour centraliser les données d'administration de son réseau) ou
être commandé pour les besoins de ses collaborateurs (annuaire des
personnels, partage d'informations, \ldots). Ces solutions sont plus
larges que le simple traitement des données (abordé au semestre 1) et
visent le développement d'un outil informatique complet partant d'un
cahier des charges donnés~: elles incluent la gestion de données
structurées (base de données, fichiers de données), leur traitement et
les éléments d'interaction utilisateur via une interface conviviale et
pratique. Elles peuvent être documentées grâce à des pages Web voire
mises à disposition des utilisateurs directement dans leur simple
navigateur Web.\\
Le professionnel R\&T doit donc mobiliser son expertise en développement
informatique pour le compte de son entreprise.}

% Les ressources
\ajoutSressources{R107}{Fondamentaux de la programmation}
\ajoutSressources{R108}{Bases des systèmes d'exploitation}
\ajoutSressources{R109}{Introduction aux technologies Web}
\ajoutSressources{R207}{Sources de données}
\ajoutSressources{R208}{Analyse et traitement de données structurées}
\ajoutSressources{R209}{Initiation au développement Web}
\ajoutSressources{R211}{Expression-Culture-Communication Professionnelles 2}

% Livrable
\ajoutSlivrables{
\vspace{-5pt}
\begin{itemize}
\item
  Codes informatiques du site Web fonctionnel et dynamique
\item
  Documentation technique, informative présentant la méthode de
  validation (exemple~: cahier de tests,
  tests unitaires)
\item
  Tutoriel d'installation et d'utilisation
\item
  Démonstration technique commentée
\item
  Présentation de l'outil utilisé pour le développement
\item
  Méthode de validation
\end{itemize}
L'étudiant s'approprie son portfolio. Des temps sont prévus pour qu'il y
synthétise sa production technique et son analyse argumentée.
}

% Mots-clés
\ajoutSmotscles{Algorithmique, Programmation, Développement Web, Documentation technique.}
