%%%%%%%%%%%%%%%%%%%%%%%%%%%%%%%%%
% Exemple de SAE
%%%%%%%%%%%%%%%%%%%%%%%%%%%%%%%%%

\nouvelexemple{Comprendre les menaces et agir}

\ajoutESproblematique{Il s'agit de présenter avec une approche éducative et technologique les
menaces numériques communes (cybersécurité) et de savoir mettre en place
les actions pour y remédier.}

\ajoutESdescription{
On pourra faire un focus particulier sur les points suivants:
\begin{itemize}
\item
  L'arnaque au faux support technique;
\item
  Les attaques en déni de service (\textabbrv{DDoS});
\item
  Le chantage à l'ordinateur ou à la webcam prétendus piratés;
\item
  L'escroquerie aux faux ordres de virement (\textabbrv{FOVI});
\item
  La défiguration de site internet;
\item
  Les fausses offres d'emploi créées par des fraudeurs;
\item
  La fraude à la carte bancaire;
\item
  L'hameçonnage (phishing en anglais);
\item
  Le piratage de compte;
\item
  Le piratage de compte de l'espace d'un recruteur;
\item
  Les propositions d'emploi non sollicitées;
\item
  Les rançongiciels (ransomwares en anglais);
\item
  Le spam électronique;
\item
  Le spam téléphonique.
\end{itemize}
On pourra également utiliser les supports:
\begin{itemize}
\item
  Cybermalveillance
  :\url{https://www.cybermalveillance.gouv.fr/bonnes-pratiques}
\item
  \textabbrv{MOOC} \textabbrv{ANSSI} :\url{https://secnumacademie.gouv.fr/}
\end{itemize}
}

\ajoutESformes{TP, projet.}



\ajoutESmodalite{L'étudiant doit démontrer qu'il est capable de présenter de façon
claire, concise et vulgarisée les menaces et attaques employées
communément sur les réseaux numériques. L'étudiant doit illustrer par
des exemples concrets les bonnes pratiques pour y remédier. Il doit
également se positionner en tant que futur professionnel au sein d'un
service informatique de son entreprise.\\
Cette démonstration pourra se faire sous la forme de présentation orale
ou écrite et accompagnée de différents média (infographie, affiche,
vidéo\ldots).}