%%%%%%%%%%%%%%%%%%%%%%%%%%%%%%%%%
% Exemple de SAE
%%%%%%%%%%%%%%%%%%%%%%%%%%%%%%%%%

\nouvelexemple{Construire son identité numérique}

\ajoutESproblematique{Pour se présenter sur Internet, le professionnel R\&T peut être amené à
créer ses propres pages Web perso, qu'il peut diffuser sur l'intranet de
son entreprise ou sur le Web. Rédiger ses pages suppose à la fois d'en
travailler le contenu (choix des informations) que la forme (outils
technologiques des sites Web) en prenant en compte les spécificités du
lecteur (collaborateurs francophones ou internationales, réseau
professionnel, \ldots).}

\ajoutESdescription{
L'étudiant développera ses premières pages personnelles sous la forme
d'un site Web statique afin de construire son identité numérique
professionnelle.\\
Le contenu du site pourra par exemple :
\begin{itemize}
\item
  être un curriculum vitae numérique;
\item
  recenser les compétences techniques et les projets techniques réalisés
  ;
\item
  contenir des liens vers des réseaux socionumériques vers les outils,
  voire vers les outils numériques
  qu'il est amené à utiliser pendant sa formation (emploi du temps, ENT,
  \ldots) ;
\item
  présenter un centre d'intérêt ;
\item
  présenter son projet professionnel.
\end{itemize}
Un affichage en anglais de tout ou partie pourra être envisagé.\\
Une attention particulière sera portée sur les contenus eux-mêmes qui
pourront par exemple être travaillés de concert avec les enseignants de
communication, d'anglais, de \texttt{PPP} et d'informatique.\\
La réalisation pourra éventuellement utiliser un système de gestion de
contenu (CMS, par exemple Wordpress).\\
Le travail pourra être intégré au portfolio de l'étudiant.
}

\ajoutESformes{TP, projet, séminaire de traces dans le cadre du \texttt{PPP}.}



\ajoutESmodalite{\vspace{-5pt}
\begin{itemize}
\item
  Mise en ligne de leur présentation numérique
\item
  Démonstration commentée du travail avec démonstration de leur aptitude
  à modifier/rajouter simplement
  des informations
\end{itemize}}