%%%%%%%%%%%%%%%%%%%%%%%%%%%%%%%%%
% SAE
%%%%%%%%%%%%%%%%%%%%%%%%%%%%%%%%%

\nouvellesae{SAÉ14}{Se présenter sur Internet}

\ajoutSheures{10}{8}{12}


%% Les compétences et les ACs
\ajoutScompetence{RT1-Administrer}{\niveauA}



\ajoutScompetence{RT2-Connecter}{\niveauA}



\ajoutScompetence{RT3-Programmer}{\niveauA}

\ajoutSac{AC0311}{Utiliser un système informatique et ses outils}
\ajoutSac{AC0314}{Connaître l'architecture et les technologies d'un site Web}

% Le description
\ajoutSdescription{
L'identité numérique professionnelle prend une place de plus en plus
importante dans la carrière d'un professionnel R\&T : elle joue un rôle
dans sa recherche d'emploi avec la valorisation de ses expériences
professionnelles comme personnelles. Elle peut également intervenir en
entreprise : certaines prévoient - dans un annuaire sur l'intranet - des
``pages personnelles'' renseignées par les salariés pour y présenter
leurs activités et dynamiser les interactions entre collaborateurs. Elle
contribue également à développer son réseau professionnel et social,
avec lequel il peut partager ses centres d'intérêt.
Le professionnel R\&T doit donc savoir se présenter sur Internet, tout
en mesurant l'importance et la portée des contenus qu'il diffuse
(e-réputation, segmentation vie privée/vie publique, \ldots).
}

% Les ressources
\ajoutSressources{R108}{Bases des systèmes d'exploitation}
\ajoutSressources{R109}{Introduction aux technologies Web}
\ajoutSressources{R110}{Anglais de communication et initiation au vocabulaire technique}
\ajoutSressources{R111}{Expression-Culture-Communication Professionnelles 1}
\ajoutSressources{R112}{PPP: Connaître son champ d'activité}
\ajoutSressources{R115}{Gestion de projet}

% Livrable
\ajoutSlivrables{
\vspace{-5pt}
\begin{itemize}
\item
  dossier ou rapport d'étude
\item
  prototype
\item
  grille de suivi
\end{itemize}
}

% Mots-clés
\ajoutSmotscles{Identité numérique, site Web}
