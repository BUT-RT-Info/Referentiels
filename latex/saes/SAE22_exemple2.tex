%%%%%%%%%%%%%%%%%%%%%%%%%%%%%%%%%
% Exemple de SAE
%%%%%%%%%%%%%%%%%%%%%%%%%%%%%%%%%

\nouvelexemple{Qualité de réception de signaux de type radio}

\ajoutESproblematique{Dans un contexte professionnel, comme chez un particulier, les signaux
radios reçus peuvent être de qualité très variable.\\
Cette qualité dépend par exemple de la puissance reçue, de la fréquence
ou de la bande de fréquence à recevoir, des conditions d'environnement,
de la localisation du récepteur, ou encore de la présence de parasites.}

\ajoutESdescription{
L'étudiant devra appréhender quels sont les paramètres pertinents pour
un système de transmission donné, déterminer quels types de mesures il
devra effectuer, quels sont les appareils adéquats, quels devront être
leurs réglages.\\
Une fois les mesures effectuées, il devra être capable de les analyser
et, par exemple, de produire une information de type cartographie de
réception.\\
Les exemples de signaux à étudier pourront être de type~:
\begin{itemize}[topsep=5pt]
\item
  signal WiFi\,;
\item
  téléphonie portable\,;
\item
  réception \textabbrv{TV}~: \textabbrv{DVB-S} ou \textabbrv{DVB-T};
\item
  réception \textabbrv{FM} ou \textabbrv{DAB}.
\end{itemize}
}

\ajoutESformes{\vspace{-5pt}
\begin{itemize}[topsep=5pt]
\item
  Travaux pratiques, notamment pour les mesures\,;
\item
  Projet, notamment pour la recherche sur le système de transmission
  retenu, pour le choix des mesures
  à faire et du paramétrage des appareils, et pour l'analyse des
  mesures.
\end{itemize}}



\ajoutESmodalite{}