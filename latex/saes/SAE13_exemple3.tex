%%%%%%%%%%%%%%%%%%%%%%%%%%%%%%%%%
% Exemple de SAE
%%%%%%%%%%%%%%%%%%%%%%%%%%%%%%%%%

\nouvelexemple{Caractériser un support radio}

\ajoutESproblematique{L'entreprise demande à l'étudiant de savoir lire un document technique
de mesure et/ou de mettre en place une liaison radio et/ou d'analyser la
structure d'une liaison radio, de caractériser par des mesures le
support radio (atténuation, effet des interférences, \ldots) voire de
certifier un câblage \textabbrv{LAN}. L'étudiant pourra ainsi comprendre les
principaux critères de choix d'un support et/ou de savoir présenter des
résultats à un client ou un collaborateur.\\
L'étudiant saura mobiliser les compétences pour réaliser des mesures ou
pour identifier et caractériser un ou des types de support radio et
savoir rédiger un compte-rendu de mesure.}

\ajoutESdescription{
L'étudiant devra s'appuyer sur ses connaissances, notamment les concepts
fondamentaux de l'étude des supports de transmissions dans les réseaux,
les concepts fondamentaux des systèmes électroniques, le vocabulaire en
architecture des réseaux numériques, des concepts mathématiques pour les
signaux de base, pour les calculs de puissance, d'atténuation.\\
L'étudiant devra :
\begin{itemize}
\item
  lire des documents techniques de support de transmission;
\item
  déterminer les types de mesures et les types de signaux nécessaires
  pour caractériser les supports
  et estimer les résultats attendus;
\item
  paramétrer les outils de mesure;
\item
  réaliser des mesures;
\item
  analyser et exploiter des résultats de tests.
\end{itemize}
}

\ajoutESformes{Mini-projet en binôme associant un TP long, encadré par un enseignant et
des heures non encadrées pour, par exemple, la préparation du TP puis
pour la rédaction du compte rendu.}



\ajoutESmodalite{L'étudiant doit être capable de rédiger un compte-rendu de mesure avec
explications.\\
On pourra s'appuyer sur:
\begin{itemize}
\item
  dossier ou rapport d'étude (compte-rendu);
\item
  rapport de mesures ;
\item
  \textabbrv{QCM} sur les mesures;
\item
  grille de suivi du travail;
\item
  présentation orale des mesures réalisées.
\end{itemize}}