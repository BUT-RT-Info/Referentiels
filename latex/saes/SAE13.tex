%%%%%%%%%%%%%%%%%%%%%%%%%%%%%%%%%
% SAE
%%%%%%%%%%%%%%%%%%%%%%%%%%%%%%%%%

\nouvellesae{SAÉ13}{Découvrir un dispositif de transmission}

\ajoutSheures{7}{5}{16}


%% Les compétences et les ACs
\ajoutScompetence{RT1-Administrer}{\niveauA}





\ajoutScompetence{RT2-Connecter}{\niveauA}

\ajoutScoeff{33}

\ajoutSac{AC0211}{Mesurer et analyser les signaux}
\ajoutSac{AC0213}{Déployer des supports de transmission}
\ajoutSac{AC0215}{Communiquer avec un client ou un collaborateur}

\ajoutScompetence{RT3-Programmer}{\niveauA}





% Le description
\ajoutSdescription{Dans cette SAE, l'étudiant(e) saura mobiliser les compétences pour
mettre en œuvre ou analyser une liaison physique (support
cuivre/fibre/radio), faire des mesures pour un premier niveau de
caractérisation, savoir présenter des résultats de mesure à un client ou
un collaborateur.}

% Les ressources
\ajoutSressources{R103}{Réseaux locaux et équipements actifs}
\ajoutSressources{R104}{Fondamentaux des systèmes électroniques}
\ajoutSressources{R105}{Supports de transmission pour les réseaux locaux}
\ajoutSressources{R106}{Architecture des systèmes numériques et informatiques}
\ajoutSressources{R113}{Mathématiques du signal}
\ajoutSressources{R114}{Mathématiques des transmissions}

% Livrable
\ajoutSlivrables{
L'évaluation s'appuiera sur tout ou partie des éléments suivants :
\begin{itemize}
\item
  dossier ou rapport d'étude (compte rendu);
\item
  rapport de mesures;
\item
  \textabbrv{QCM} sur les mesures;
\item
  grille de suivi du travail;
\item
  présentation orale des mesures réalisées.
\end{itemize}
L'étudiant s'approprie son portfolio. Des temps sont prévus pour qu'il y
synthétise sa production technique et son analyse argumentée.
}

% Mots-clés
\ajoutSmotscles{Mesures, Supports de transmission, Fibre optique, Cuivre, Radio.}
