%%%%%%%%%%%%%%%%%%%%%%%%%%%%%%%%%
% Exemple de SAE
%%%%%%%%%%%%%%%%%%%%%%%%%%%%%%%%%

\nouvelexemple{Installation automatisée de postes clients}

\ajoutESproblematique{Les entreprises qui proposent des formations sont contraintes de
réinstaller les ordinateurs ayant servis lors de ces stages
(applications spécifiques d'une formation à une autre\ldots). Le service
informatique de l'entreprise confie à l'étudiant qui endosse le rôle d'
``Assistant administrateur réseau'' la mission d'automatiser le
déploiement des systèmes d'exploitation sur les postes clients de la
salle de formation.}

\ajoutESdescription{
Il s'agit d'explorer les techniques classiques d'installation système.
Ces techniques peuvent se décliner avec plusieurs systèmes
d'exploitation différents (GNU/Linux, Solaris, Aix, OpenBSD, NetBSD,
FreeBSD, GNU/Hurd) et peuvent s'adapter à l'installation de matériel
embarqué quand le bootloader (U-boot) est accessible. La mise en oeuvre
comprend~:
\begin{itemize}[topsep=5pt]
\item
  Démarrage réseau (\textabbrv{PXE}) pour charger un bootloader (\textabbrv{PXE} Linux)\,;
\item
  Configuration d'un serveur \textabbrv{TFTP} (hpa-tftpd)\,;
\item
  Modification du fichier de configuration pour choisir le mode
  (utilisation normale / installation)\,;
\item
  Installation manuelle pour créer le fichier de réponses\,;
\item
  Installation du fichier de réponses sur un serveur Web (disponible ou
  installé)\,;
\item
  Utilisation de stratégies pour partitionner le disque dur\,;
\item
  Adaptation des clés d'identifications des ordinateurs.
\end{itemize}
A minima, deux ordinateurs (physiques et/ou virtuels) sont nécessaires~:
un serveur et un client.
}

\ajoutESformes{De façon individuelle ou collective, sur des heures encadrées et non
encadrées, l'étudiant ou l'équipe, sera confronté aux formes
pédagogiques suivantes~:
\begin{itemize}[topsep=5pt]
\item
  Mise en place un serveur \textabbrv{TFTP} et permettre aux machines. clientes de
  démarrer sur le réseau (\textabbrv{PXE})
  pour récupérer une image du système d'exploitation\,;
\item
  Élaboration d'une méthode efficace pour tester progressivement la
  configuration réalisée\,;
\item
  Utilisation d'outils de diagnostic\,;
\item
  Rédaction de fiches opératoires (notice d'utilisation).
\end{itemize}}



\ajoutESmodalite{\vspace{-5pt}
\begin{itemize}[topsep=5pt]
\item
  Déploiement~: opérationnel et répondant aux problématiques suivantes~:
  \begin{itemize}
  \item
    Le stagiaire peut-il ouvrir une session sur le système
    d'exploitation ?
  \item
    Le stagiaire bénéficie t-il d'un environnement informatique conforme
    aux besoins de sa formation
    ?
  \end{itemize}
\item
  Documentation~: les procédures sont-elles applicables par une autre
  équipe ?
\item
  Réunions~: présentation finale (soutenance) de la solution mise en
  place.
\end{itemize}}