%%%%%%%%%%%%%%%%%%%%%%%%%%%%%%%%%
% Exemple de SAE
%%%%%%%%%%%%%%%%%%%%%%%%%%%%%%%%%

\nouvelexemple{Études comparatives de solutions de numérisation de l'information}

\ajoutESproblematique{Dans un contexte professionnel, le choix d'une solution de numérisation
de l'information a des conséquences directes sur la qualité du signal
transmis, le débit ou la bande passante nécessaire, le coût des
équipements (codec gratuit ou payant par exemple). Il est donc
intéressant de pouvoir comparer plusieurs solutions de numérisation afin
de déterminer la plus judicieuse, en se focalisant, dans ce contexte de
1ère année de BUT, sur des signaux de type audio.}

\ajoutESdescription{
Les diverses solutions de numérisation pourront faire intervenir la
fréquence d'échantillonnage, le nombre de bits de conversion, la loi de
conversion.\\
L'étudiant devra appréhender quels sont les paramètres pertinents pour
un système de transmission donné, déterminer quels types de mesures il
devra effectuer, quels sont les appareils adéquats, quels devront être
leurs réglages.\\
Une fois les mesures effectuées, il devra être capable de les analyser
et, par exemple, de produire une information de type cartographique.\\
Les exemples de numérisation à comparer pourront être de type~:
\begin{itemize}
\item
  audio en téléphonie\,;
\item
  audio avec une qualité \textabbrv{HiFi};
\item
  codec G711.
\end{itemize}
}

\ajoutESformes{\vspace{-5pt}
\begin{itemize}
\item
  Travaux pratiques, notamment pour les mesures\,;
\item
  Projet, notamment pour la recherche sur les standards de numérisations
  retenus, pour le choix des
  mesures à faire et du paramétrage des appareils, et pour l'analyse des
  mesures effectuées.
\end{itemize}}



\ajoutESmodalite{\vspace{-5pt}
\begin{itemize}
\item
  Rapport écrit\,;
\item
  et/ou présentation orale des performances mesurées et de leur analyse.
\end{itemize}}