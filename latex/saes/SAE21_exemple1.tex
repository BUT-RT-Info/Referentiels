%%%%%%%%%%%%%%%%%%%%%%%%%%%%%%%%%
% Exemple de SAE
%%%%%%%%%%%%%%%%%%%%%%%%%%%%%%%%%

\nouvelexemple{Construction de réseau routé (statique/dynamique), routage inter-\textabbrv{VLAN}~:  approche matérielle ou virtuelle}

\ajoutESproblematique{\vspace{-5pt}
\begin{itemize}
\item
  Comprendre et construire une architecture de réseaux d'entreprise et
  d'Internet.
\item
  Élaborer une méthode efficace pour tester progressivement la
  configuration réalisée.
\item
  Construire un réseau local virtuel \textabbrv{VLAN}.
\item
  Intercepter un trafic entre 2 ordinateurs et identifier le chemin
  utilisé.
\item
  Construire une passerelle entre un réseau émulé et un réseau réel.
\end{itemize}}

\ajoutESdescription{
L'objectif est de construire un réseau local de niveau 2 (commutation)
et 3 (routage) en introduisant le concept de réseau local virtuel
(\textabbrv{VLAN}). Le réseau répondra à un besoin d'entreprise ``simple'', par ex:
le réseau d'une entreprise localisée dans 3 villes différentes. Un outil
d'émulation est utilisé avec production d'un projet enregistrable pour
que l'exercice puisse être construit progressivement au fur et à mesure
des séances. L'étudiant doit s'organiser pour construire par étapes son
réseau et surtout vérifier à chaque étape que l'ajout qu'il vient
d'effectuer permet au réseau de toujours fonctionner.\\
Développement et configuration d'une architecture de réseau d'entreprise
simple composée de 6 \textabbrv{VLAN} et 3 routeurs.\\
Équipements réseau~: 4 switches et 3 routeurs.\\
Extensions possibles~: effectuer du \textabbrv{VRF}, ajouter des tunnels, introduire
un \textabbrv{NAT}, effectuer du filtrage sur un \textabbrv{VLAN} spécifique.\\
Préconisations~:
\begin{itemize}
\item
  Fourniture d'un cahier des charges pour 2 étudiants
\item
  Utilisation d'un logiciel d'émulation type GNS Packet Tracer / EVE-NG
  / Marionnet pour que le projet
  puisse être travaillé dans une salle de TP ou à la maison.
\end{itemize}
}

\ajoutESformes{TP, séances de projet}



\ajoutESmodalite{Tout ou partie des éléments suivants~:
\begin{itemize}
\item
  Projet réseau fonctionnel\,;
\item
  Vérification d'une interconnexion complète\,;
\item
  Vérification de la sécurisation/isolement d'un \textabbrv{VLAN} vis à vis des
  autres postes clients\,;
\item
  Démontrer à l'aide d'un document le respect du cahier des charges. Il
  pourra être effectué une démonstration
  filmée incluant des scénarios de test. On évaluera la qualité de la
  démonstration et la pédagogie de
  la vidéo\,;
\item
  Présentation orale avec diaporama ou entretien.
\end{itemize}}