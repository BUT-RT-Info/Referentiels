%%%%%%%%%%%%%%%%%%%%%%%%%%%%%%%%%
% Exemple de SAE
%%%%%%%%%%%%%%%%%%%%%%%%%%%%%%%%%

\nouvelexemple{Construction de réseau routé (statique/dynamique), routage inter-vlan :  approche matérielle ou virtuelle}

\ajoutESproblematique{Comprendre et construire une architecture de réseaux d'entreprise et
d'Internet.\\
Élaborer une méthode efficace pour tester progressivement la
configuration réalisée.\\
Construire un réseau local virtuel \textabbrv{VLAN}\\
Intercepter un trafic entre 2 ordinateurs et identifier le chemin
utilisé.\\
Construire une passerelle entre un réseau émulé et un réseau réel}

\ajoutESdescription{
L'objectif est de construire un réseau local de niveau 2 (commutation)
et 3 (routage) en introduisant le concept de réseau local virtuel
(\textabbrv{VLAN}). Un outil d'émulation est utilisé avec production d'un projet
enregistrable pour que l'exercice puisse être construit progressivement
au fur et à mesure des séances. L'étudiant doit s'organiser pour
construire par étapes son réseau et surtout vérifier à chaque étape que
l'ajout qu'il vient d'effectuer permet au réseau de toujours
fonctionner.\\
Développement et configuration d'une architecture de réseau d'entreprise
simple composée de 6 \textabbrv{VLAN} et 3 routeurs.\\
Équipements réseau : 4 switches et 3 routeurs.\\
Extensions possible : effectuer du \textabbrv{VRF}, ajouter des tunnels, introduire
un \textabbrv{NAT}, effectuer du filtrage sur un \textabbrv{VLAN} spécifique\\
Il faut de toute façon définir un besoin d'entreprise \og  simple \fg{}.
L'exemple consiste à élaborer un réseau d'une entreprise localisée dans
3 villes différentes.\\
Préconisations :\\
Fourniture d'un cahier des charges pour 2 étudiants Utilisation d'un
logiciel d'émulation type gns\textgreater packetracer/EVE-NG/Marionnet
pour que le projet puisse être travaillé dans une salle de TP ou à la
maison.\\
L'étudiant devra mobiliser ses connaissances en :
\begin{itemize}
\item
  Initiation au réseau informatique
\item
  Principes et architecture des réseaux
\item
  Réseaux locaux et équipements actifs
\item
  Technologie de l'Internet
\item
  Administration système
\item
  E\textabbrv{CC}
\end{itemize}
}

\ajoutESformes{TP, séances de projet}



\ajoutESmodalite{Tout ou partie des éléments suivants :
\begin{itemize}
\item
  Projet réseau fonctionnel
\item
  Vérification d'une interconnexion complète
\item
  Vérification de la sécurisation/isolement d'un \textabbrv{VLAN} vis à vis des
  autres postes clients
\item
  Démontrer à l'aide d'un document le respect du cahier des charges,
  effectuer une démonstration filmée
  incluant des scénarios de test. On évaluera la qualité de la
  démonstration et la pédagogie de la vidéo.
\item
  Présentation orale avec diaporama ou entretien.
\end{itemize}}