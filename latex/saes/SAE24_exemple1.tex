%%%%%%%%%%%%%%%%%%%%%%%%%%%%%%%%%
% Exemple de SAE
%%%%%%%%%%%%%%%%%%%%%%%%%%%%%%%%%

\nouvelexemple{Découvrir mon réseau informatique domestique}

\ajoutESproblematique{Le réseau informatique domestique est une \og  petite \fg{} instance
d'un réseau d'entreprise (par exemple, box, points d'accès WiFi, réseaux
du CROUS, modem, \textabbrv{4G}, ordinateurs, téléphonie, objets connectés).\\
Dans cette SAÉ, l'étudiant devra comprendre l'agencement des briques
télécoms (accès au réseau), des équipements (routeur, switch, firewall,
WiFi), des terminaux (\textabbrv{PC}, smartphone, \ldots) et des protocoles (IP,
\textabbrv{DHCP}, \textabbrv{DNS}, Mail, Web, \ldots) qui permettent le fonctionnement de son
réseau domestique.\\
Il devra également reconnaître les \textabbrv{OS} des différents appareils connectés
à ce réseau et décrire leurs caractéristiques, notamment en termes de
sécurité.}

\ajoutESdescription{
Il s'agit d'un projet individuel permettant à l'étudiant de développer
une méthodologie de travail pour découvrir son réseau domestique
(architecture, technologies, services offerts). On s'appuiera sur les
concepts fondamentaux des systèmes d'exploitation, des protocoles
réseaux et des outils logiciel réseau de base et exprimera les résultats
à l'aide des termes professionnels du domaine réseaux-informatiques.\\
L'étudiant devra mobiliser toutes les ressources vues jusqu'à présent :
\begin{itemize}
\item
  Cours réseaux, informatique, télécommunications;
\item
  Expression-communication: recherche documentaire, rédaction, exposé;
\item
  Vocabulaire anglais en réseaux et télécoms;
\item
  Outils numériques de schéma réseau, outil de présentation type
  powerpoint.
\end{itemize}
}

\ajoutESformes{Il s'agit d'apprendre comment s'architecture un réseau informatique par
l'observation et la pratique en s'appuyant sur l'environnement réseau
domestique propre à chaque étudiant.}



\ajoutESmodalite{L'étudiant doit être capable de retranscrire son réseau grâce à un outil
de schéma réseau et de le détailler avec l'ensemble des observations
relevées. Il doit être capable de produire une présentation type
powerpoint et de la présenter oralement en 5 minutes maximum.}