%%%%%%%%%%%%%%%%%%%%%%%%%%%%%%%%%
% SAE
%%%%%%%%%%%%%%%%%%%%%%%%%%%%%%%%%

\nouvellesae{SAÉ21}{Construire un réseau informatique pour une petite structure}

\ajoutSheures{10}{8}{12}


%% Les compétences et les ACs
\ajoutScompetence{RT1-Administrer}{\niveauA}

\ajoutScoeff{23}

\ajoutSac{AC0112}{Comprendre l'architecture des systèmes numériques et les principes du codage de l'information}
\ajoutSac{AC0113}{Configurer les fonctions de base du réseau local}
\ajoutSac{AC0114}{Maîtriser les rôles et les principes fondamentaux des systèmes d'exploitation afin d'interagir avec ceux-ci pour la configuration et administration des réseaux et services fournis}
\ajoutSac{AC0115}{Identifier les dysfonctionnements du réseau local}
\ajoutSac{AC0116}{Installer un poste client}

\ajoutScompetence{RT2-Connecter}{\niveauA}





\ajoutScompetence{RT3-Programmer}{\niveauA}





% Le description
\ajoutSdescription{Le professionnel R\&T peut être sollicité pour construire et mettre en
place le réseau informatique d'une entreprise. L'objectif de cette SAE
est d'amener l'étudiant à répondre aux besoins de commutation, de
routage, de services réseaux de base et de sécurité formulés par une
petite structure multisite. Ce réseau s'appuie sur des équipements et
des services informatiques incontournables mais fondamentaux pour
fournir à la structure un réseau fonctionnel et structuré.}

% Les ressources
\ajoutSressources{R101}{Initiation aux réseaux informatiques}
\ajoutSressources{R102}{Principes et architecture des réseaux}
\ajoutSressources{R103}{Réseaux locaux et équipements actifs}
\ajoutSressources{R108}{Bases des systèmes d'exploitation}
\ajoutSressources{R201}{Technologie de l'Internet}
\ajoutSressources{R202}{Administration système et fondamentaux de la virtualisation}
\ajoutSressources{R203}{Bases des services réseaux}

% Livrable
\ajoutSlivrables{
\vspace{-5pt}
\begin{itemize}
\item
  Maquette du projet
\item
  Dossier ou rapport décrivant l'architecture physique, les \texttt{VLAN},
  l'adressage \texttt{IP}, les principaux points
  de vérification du projet, des captures de trafic et différents
  scénarios permettant de valider les
  contraintes du cahier des charges (scénarios de routage, d'accès aux
  ressources publiques de l'entreprise\ldots)
\item
  Vidéo de démonstration du fonctionnement
\end{itemize}
}

% Mots-clés
\ajoutSmotscles{Adressage \texttt{IP}, \texttt{VLAN}, \texttt{VTP}, routage inter-\texttt{VLAN}, \texttt{NAT}, \texttt{PAT}, \texttt{ACL}, \texttt{DNS}, \texttt{HTTP}, \texttt{SSH}, routage (vecteur de distance / état de lien), \texttt{PXE}, \texttt{TFTP}}
