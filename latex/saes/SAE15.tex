%%%%%%%%%%%%%%%%%%%%%%%%%%%%%%%%%
% SAE
%%%%%%%%%%%%%%%%%%%%%%%%%%%%%%%%%

\nouvellesae{SAÉ15}{Traiter des données}

\ajoutSheures{10}{8}{20}


%% Les compétences et les ACs
\ajoutScompetence{RT1-Administrer}{\niveauA}





\ajoutScompetence{RT2-Connecter}{\niveauA}





\ajoutScompetence{RT3-Programmer}{\niveauA}

\ajoutScoeff{26}

\ajoutSac{AC0311}{Utiliser un système informatique et ses outils}
\ajoutSac{AC0312}{Lire, exécuter, corriger et modifier un programme}
\ajoutSac{AC0313}{Traduire un algorithme, dans un langage et pour un environnement donné}
\ajoutSac{AC0314}{Connaître l'architecture et les technologies d'un site Web}
\ajoutSac{AC0315}{Choisir les mécanismes de gestion de données adaptés au développement de l'outil}
\ajoutSac{AC0316}{S'intégrer dans un environnement propice au développement et au travail collaboratif}

% Le description
\ajoutSdescription{Le professionnel R\&T est régulièrement amené à traiter des données
provenant du système d'information de l'entreprise pour ses besoins
personnels ou ceux de ses collaborateurs. Ces données peuvent par
exemple être liées à l'infrastructure réseau qu'il administre (état des
équipements, des machines) ou aux utilisateurs du réseau. Généralement
obtenues sous forme brute, elles sont ensuite traitées avec des
objectifs très variés (nettoyage des données, extraction d'informations
comptables , archivage, \ldots) pour être réutilisées à d'autres fins ou
être présentées dans des vues synthétiques. Ces traitements peuvent être
récurrents (mensualisation de bilan, sauvegarde de données périodique,
\ldots) gagnent à être automatisés.\\
Le professionnel R\&T doit donc développer des scripts ou des programmes
pour gérer de façon efficace le traitement de ces données.}

% Les ressources
\ajoutSressources{R107}{Fondamentaux de la programmation}
\ajoutSressources{R108}{Bases des systèmes d'exploitation}
\ajoutSressources{R109}{Introduction aux technologies Web}
\ajoutSressources{R110}{Anglais de communication et initiation au vocabulaire technique}
\ajoutSressources{R111}{Expression-Culture-Communication Professionnelles 1}
\ajoutSressources{R115}{Gestion de projet}

% Livrable
\ajoutSlivrables{
\vspace{-5pt}
\begin{itemize}
\item
  Codes informatiques développés\,;
\item
  Démonstration technique commentée\,;
\item
  et/ou Rapport technique avec tutoriel d'installation\,;
\item
  et/ou Soutenance orale présentant le travail réalisé.
\end{itemize}
L'étudiant s'approprie son portfolio. Des temps sont prévus pour qu'il y
synthétise sa production technique et son analyse argumentée.
}

% Mots-clés
\ajoutSmotscles{Algorithmique, Programmation, Script.}
