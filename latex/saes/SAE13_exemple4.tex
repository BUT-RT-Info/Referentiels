%%%%%%%%%%%%%%%%%%%%%%%%%%%%%%%%%
% Exemple de SAE
%%%%%%%%%%%%%%%%%%%%%%%%%%%%%%%%%

\nouvelexemple{Caractériser des supports de transmission}

\ajoutESproblematique{L'étudiant saura mobiliser les compétences pour réaliser des mesures ou
pour identifier et caractériser un ou des types de support et savoir
rédiger un compte-rendu de mesure.}

\ajoutESdescription{
\vspace{-5pt}
\begin{itemize}
\item
  Contexte professionnel :
\end{itemize}
L'entreprise demande à l'étudiant de caractériser par des mesures un ou
plusieurs types de supports (retard de propagation, atténuation,
continuité, échos, bruit, perturbations, identifier un défaut) voire de
certifier un câblage \texttt{LAN} (cuivre, fibre, radio) afin de comprendre et
comparer les principaux critères de choix d'un support et/ou de savoir
présenter des résultats à un client ou un collaborateur.\\
L'étudiant devra s'appuyer sur ses connaissances, notamment les concepts
fondamentaux de l'étude des supports de transmissions dans les réseaux,
les concepts fondamentaux des systèmes électroniques, le vocabulaire en
architecture des réseaux numériques, des concepts mathématiques pour les
signaux de base, pour les calculs de puissance, d'atténuation.
}

\ajoutESformes{Mini-projet en binôme associant un TP long, encadré par un enseignant et
des heures non encadrées pour, par exemple, la préparation du TP puis
pour la rédaction du compte rendu.\\
L'étudiant devra :
\begin{itemize}
\item
  lire des documents techniques de support de transmission;
\item
  déterminer les types de mesures et les types de signaux nécessaires
  pour caractériser les supports
  et estimer les résultats attendus;
\item
  paramétrer les outils de mesure;
\item
  réaliser des mesures;
\item
  analyser et exploiter des résultats de tests.
\end{itemize}
Exemples de mise en oeuvre :
\begin{itemize}
\item
  sur un support cuivre : mesure temporelle (échelon, sinus), retard de
  propagation, atténuation,échos
  (réflexion), perturbations, (\texttt{GBF}, oscillo, câble);
\item
  sur un support \texttt{FO} : soudure, crayon optique, sonde d'inspection,
  photométrie;
\item
  sur un support radio : atténuation,\ldots{}
\end{itemize}}



\ajoutESmodalite{L'étudiant doit être capable de rédiger un compte-rendu de mesure avec
explications.\\
Modalités :\\
On pourra s'appuyer sur:
\begin{itemize}
\item
  dossier ou rapport d'étude (compte-rendu);
\item
  rapport de mesures ;
\item
  \texttt{QCM} sur les mesures;
\item
  grille de suivi du travail;
\item
  présentation orale des mesures réalisées.
\end{itemize}}