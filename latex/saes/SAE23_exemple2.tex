%%%%%%%%%%%%%%%%%%%%%%%%%%%%%%%%%
% Exemple de SAE
%%%%%%%%%%%%%%%%%%%%%%%%%%%%%%%%%

\nouvelexemple{Application Web à usage interne de l'entreprise}

\ajoutESproblematique{Le professionnel R\&T peut être sollicité pour développer un outil
informatique répondant aux besoins de ses collaborateurs (par ex: une
solution de cartographie du matériel de l'entreprise, une interface de
gestion des informations sur le personnel pour les ressources humaines,
\ldots). Il peut choisir de concevoir cet outil sous forme d'un site Web
; l'outil sera ainsi facilement accessible des utilisateurs, grâce à un
simple navigateur Web. Le professionnel doit alors - en plus du
développement - documenter les fonctionnalités de son outil et former
les utilisateurs à son utilisation.}

\ajoutESdescription{
Cette SAÉ propose à l'étudiant de s'initier au développement d'une telle
application Web. Partant des besoins utilisateurs, l'étudiant devra~:
\begin{itemize}[topsep=5pt]
\item
  mettre en place de son environnement de travail
\item
  choisir et utiliser les technologies Web adéquates pour produire son
  site Web, mettre en place la
  gestion des données du site et programmer leur traitement
\item
  présenter le travail réalisé aux utilisateurs pour les former à son
  utilisation, certains utilisateurs
  pouvant être anglophones.
\end{itemize}
La SAÉ pourra être réalisée par un groupe de 2 étudiants.\\
Partant d'un cahier des charges fourni, et pour un binôme d'étudiants,
la SAé pourra être mise en oeuvre avec différentes étapes~:
\begin{itemize}[topsep=5pt]
\item
  Phase 1~: mise en place de l'environnement de développement~:
  \begin{itemize}
  \item
    Utilisation d'une machine virtuelle ou accès à distance aux
    ressources (par exemple~: ferme ESX,
    Proxmox, Guacamole, Docker, \ldots), partage par clés \textabbrv{USB}
  \item
    Installation ou utilisation d'un serveur web non chiffré (type Nginx
    ou Apache)
  \item
    Utilisation possible d'un framework python (Django, Flask) ou
    JavaScript (jQuery), Java (play), etc\ldots{}
  \end{itemize}
\item
  Phase 2~: réalisation documentée incluant~:
  \begin{itemize}
  \item
    Algorithmique (script serveur, dépôt de codes)
  \item
    Technologie Web (\textabbrv{HTML}, css)
  \item
    Base de données avec manipulation de données (ajout, suppression,
    modification)
  \end{itemize}
\item
  Phase 3~: organisation d'une session de formation en anglais à
  l'application Web auprès des collaborateurs
  de la société, avec documentation de l'application en anglais.
\end{itemize}
}

\ajoutESformes{TP, projet}



\ajoutESmodalite{\vspace{-10pt}
\begin{itemize}[topsep=5pt]
\item
  Site fonctionnel et dynamique
\item
  Données manipulées dans une \textabbrv{BDD} (ajout, suppression, modification)
\item
  Documentation technique, informative
\item
  Tutoriel d'installation, d'utilisation
\item
  Dépôt du code
\item
  Démonstration
\item
  Présentation de l'outil utilisé pour le développement
\item
  Méthode de validation (exemple~: cahier de tests, tests unitaires)
\end{itemize}}