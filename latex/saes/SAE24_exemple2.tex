%%%%%%%%%%%%%%%%%%%%%%%%%%%%%%%%%
% Exemple de SAE
%%%%%%%%%%%%%%%%%%%%%%%%%%%%%%%%%

\nouvelexemple{Déployer le système d'information d'une petite entreprise, filiale d'une entreprise internationale}

\ajoutESproblematique{L'étudiant se met dans le rôle d'agent d'une société locale de services
du numérique. Il est contacté par une entreprise qui lui passe une
commande~: refaire une partie du système et réseau.\\
Sa mission consiste à~:
\begin{itemize}
\item
  Déployer le réseau d'une petite entreprise\,;
\item
  Donner l'accès au système d'information de l'entreprise aux nouveaux
  utilisateurs\,;
\item
  Partager les documents de l'entreprise sur le réseau local\,;
\item
  Former les employés de l'entreprise à l'utilisation du système et à
  l'hygiène informatique.
\end{itemize}}

\ajoutESdescription{
L'étudiant devra installer des équipements réseaux, déployer un serveur
d'entreprise et les postes clients.\\
Il devra ensuite créer une dizaine de comptes utilisateurs ainsi que les
accès aux fichiers partagés pour quelques postes d'un réseau local. Il
devra en assurer le bon fonctionnement et la maintenance.\\
Il devra également maîtriser le vocabulaire technique en anglais et
savoir communiquer avec les autres filiales de l'entreprise
internationale.
}

\ajoutESformes{De façon individuelle ou collective, sur des heures encadrées et non
encadrées, l'étudiant ou l'équipe, sera confronté aux formes
pédagogiques suivantes~:
\begin{itemize}
\item
  Création d'un réseau et l'administration du système de manière
  physique et/ou virtuelle\,;
\item
  Entretiens oraux en anglais avec les utilisateurs de la maison mère
  pour résoudre un problème simple
  d'utilisation\,;
\item
  Rédaction de fiches opératoires (notice d'utilisation) en français et
  en anglais\,;
\item
  Audit d'évaluation par les pairs en aveugle\,;
\item
  Organisation de réunions en français et en anglais (présentations et
  formation, dont les bonnes pratiques
  en matière de sécurité).
\end{itemize}}



\ajoutESmodalite{\vspace{-5pt}
\begin{itemize}
\item
  Réseau entreprise opérationnel~: un utilisateur lambda peut-il se
  connecter et accéder aux applications
  et documents de l'entreprise ?
\item
  Notices~: les procédures sont-elles applicables par une autre équipe ?
\item
  Réunions~: présentation finale (soutenance) du système mis en place,
  en français et/ou en anglais.
\end{itemize}}