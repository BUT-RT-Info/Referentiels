%%%%%%%%%%%%%%%%%%%%%%%%%%%%%%%%%
% Exemple de SAE
%%%%%%%%%%%%%%%%%%%%%%%%%%%%%%%%%

\nouvelexemple{Comment construire son réseau d'entreprise ?}

\ajoutESproblematique{Le but est d'amener les étudiants à construire eux-mêmes un réseau
simplifié d'équipements actifs filaires interconnectés, sur la base d'un
cahier des charges général type \og  schéma de dépannage \fg{}.}

\ajoutESdescription{
Cette SAÉ nécessite un travail en amont de recensement des
fonctionnalités nécessaires et d'un recensement de matériel choisi sur
la base d'un recueil de fiches produit professionnelles (\og  datasheet
\fg{} en anglais) ou à défaut une sitographie commerçante anglophone.\\
Suite à ce choix, les étudiants produisent eux-mêmes au moins un des
câbles nécessaires à cette installation. Ils utiliseront les équipements
déjà en place dans l'établissement en remplacement de ceux qui auront
été déterminés pour achat sur catalogue. La SAÉ se termine par le
déploiement et la configuration des équipements et des postes clients et
d'en permettre leur administration dans l'avenir.\\
Une fois le réseau fonctionnel, il convient de définir un poste comme
point d'administration. Ce poste dispose d'un accès à une interface de
commande (console et ssh sur routeur et switch) et d'une interface
capturant le trafic sur lien inter-\textabbrv{VLAN}. Il est également possible
d'analyser les flux sur grâce aux outils de capture des trames (Tcpdump,
Wireshark).\\
En synthèse, l'étudiant sera confronté à~:
\begin{itemize}[topsep=5pt]
\item
  Analyse des besoins et chiffrage des achats nécessaires\,;
\item
  Construction (sertissage) des câbles et recettes de conformité\,;
\item
  Déploiement et configuration d'un \textabbrv{LAN} multi-réseaux, multi-\textabbrv{VLAN};
\item
  Déploiement des postes clients communicants\,;
\item
  Mise en place d'une solution d'administration et de surveillance des
  flux.
\end{itemize}
}

\ajoutESformes{TP, projet}



\ajoutESmodalite{\vspace{-5pt}
\begin{itemize}[topsep=5pt]
\item
  Livrable documentaire en tant qu'évaluation intermédiaire~:
  \begin{itemize}
  \item
    recensement des besoins de fonctionnalités\,;
  \item
    prévisionnel des achats nécessaires (hors poste client).
  \end{itemize}
\item
  Livrables de réalisation pratique en tant qu'évaluation terminale~:
  \begin{itemize}
  \item
    validation en TP des configurations déployées sur les équipements~:
    l'étudiant doit être en capacité
    de commenter ses choix et réalisations\,;
  \item
    ou soutenance justifiant les choix et présentant l'architecture mise
    en place.
  \end{itemize}
\end{itemize}}