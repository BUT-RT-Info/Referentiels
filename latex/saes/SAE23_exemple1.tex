%%%%%%%%%%%%%%%%%%%%%%%%%%%%%%%%%
% Exemple de SAE
%%%%%%%%%%%%%%%%%%%%%%%%%%%%%%%%%

\nouvelexemple{Application Web pour un prestataire}

\ajoutESproblematique{Le professionnel R\&T peut être amené à développer des applications Web,
sous forme de sites Web manipulables grâce à un navigateur Web~: elles
peuvent être destinées à ses collaborateurs pour mettre facilement à
leur disposition un outil informatique\,; elles peuvent aussi être le
cœur de métier de son entreprise avec des applications destinées à des
clients/commanditaires.\\
Cette SAÉ propose à l'étudiant de s'initier au développement d'une
application Web en le plaçant dans un contexte de réponse à un appel
d'offres~: un client cherche un prestataire pour développer une
application remplissant un cahier des charges précis~: par exemple, une
solution pour cartographier son matériel de réseau local sous la forme
d'un site web.\\[3pt]}

\ajoutESdescription{
Partant des spécifications fournies par le client, l'étudiant devra
proposer, développer un prototype de l'application puis présenter le
travail réalisé devant le jury de sélection du prestataire.\\
L'étudiant devra mettre en place son environnement de travail, choisir
et utiliser les technologies Web adéquates pour produire son site Web,
mettre en place la gestion des données du site et programmer leur
traitement.\\
La SAÉ pourra être réalisée par un groupe de 2 étudiants.\\
Partant d'un cahier des charges fourni, et pour un binôme d'étudiants,
la SAé pourra être mise en oeuvre avec différentes étapes~:
\begin{itemize}[topsep=5pt]
\item
  Phase 1~: mise en place de l'environnement de développement~:
  \begin{itemize}
  \item
    Utilisation d'une machine virtuelle ou accès à distance aux
    ressources (par ex~: ferme ESX, Proxmox,
    Guacamole, Docker, \ldots), partage par clés \textabbrv{USB}
  \item
    Installation ou utilisation d'un serveur Web non chiffré (type Nginx
    ou Apache)
  \item
    Utilisation possible d'un framework python (Django, Flask) ou
    JavaScript (jQuery), java (play), etc\ldots{}
  \end{itemize}
\item
  Phase 2~: réalisation documentée incluant~:
  \begin{itemize}
  \item
    Algorithmique (script serveur, dépôt de codes)
  \item
    Technologie Web (\textabbrv{HTML}, css)
  \item
    Base de données avec manipulation de données (ajout, suppression,
    modification)
  \end{itemize}
\item
  Phase 3~: présentation du prototype devant le jury de sélection avec
  rédaction d'un rapport.
\end{itemize}
Le travail demandé pourra inclure a minima un livrable en anglais
(commentaires des codes, vidéo, présentation ou documentation).\\
Les transformations attendues chez l'étudiant sont~:
\begin{itemize}[topsep=5pt]
\item
  Apprendre en autonomie
\item
  Apprendre à apprendre
\item
  Initiation et découverte des architectures applicatives
\item
  Apprendre à valoriser son travail
\end{itemize}
}

\ajoutESformes{TP, Projet, Serious game\\[3pt]}



\ajoutESmodalite{\vspace{-10pt}
\begin{itemize}[topsep=5pt]
\item
  Site fonctionnel et dynamique
\item
  Documentation technique, informative
\item
  Tutoriel d'installation, d'utilisation
\item
  Dépôt du code
\item
  Démonstration
\item
  Présentation de l'outil utilisé pour le développement
\item
  Méthode de validation (exemple~: cahier de tests, tests unitaires)
\end{itemize}}