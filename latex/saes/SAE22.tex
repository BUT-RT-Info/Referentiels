%%%%%%%%%%%%%%%%%%%%%%%%%%%%%%%%%
% SAE
%%%%%%%%%%%%%%%%%%%%%%%%%%%%%%%%%

\nouvellesae{SAÉ22}{Mesurer et caractériser un signal ou un système}

\ajoutSheures{10}{10}{12}


%% Les compétences et les ACs
\ajoutScompetence{RT1-Administrer}{\niveauA}





\ajoutScompetence{RT2-Connecter}{\niveauA}

\ajoutScoeff{19}

\ajoutSac{AC0211}{Mesurer et analyser les signaux}
\ajoutSac{AC0212}{Caractériser des systèmes de transmissions élémentaires et découvrir la modélisation mathématique de leur fonctionnement}
\ajoutSac{AC0213}{Déployer des supports de transmission}
\ajoutSac{AC0215}{Communiquer avec un client ou un collaborateur}

\ajoutScompetence{RT3-Programmer}{\niveauA}





% Le description
\ajoutSdescription{Dans cette SAE, l'étudiant saura mobiliser les compétences pour analyser
des signaux d'un système de transmission, les exploiter, et les
présenter sous forme d'un bilan à un client ou un collaborateur.}

% Les ressources
\ajoutSressources{R104}{Fondamentaux des systèmes électroniques}
\ajoutSressources{R113}{Mathématiques du signal}
\ajoutSressources{R114}{Mathématiques des transmissions}
\ajoutSressources{R205}{Signaux et Systèmes pour les transmissions}
\ajoutSressources{R206}{Numérisation de l'information}
\ajoutSressources{R213}{Mathématiques des systèmes numériques}
\ajoutSressources{R214}{Analyse mathématique des signaux}

% Livrable
\ajoutSlivrables{
\vspace{-10pt}
\begin{itemize}[topsep=5pt]
\item
  Rapport écrit\,;
\item
  Présentation orale des performances mesurées.
\end{itemize}
L'étudiant s'approprie son portfolio. Des temps sont prévus pour qu'il y
synthétise sa production technique et son analyse argumentée.
}

% Mots-clés
\ajoutSmotscles{Spectre, Puissance, Décibels, Sensibilité, Atténuation, Gain.}
