%%%%%%%%%%%%%%%%%%%%%%%%%%%%%%%%%
% SAE
%%%%%%%%%%%%%%%%%%%%%%%%%%%%%%%%%

\nouvellesae{SAÉ12}{S'initier aux réseaux informatiques}

\ajoutSheures{10}{7}{20}


%% Les compétences et les ACs
\ajoutScompetence{RT1-Administrer}{\niveauA}

\ajoutScoeff{32}

\ajoutSac{AC0111}{Maîtriser les lois fondamentales de l'électricité afin d'intervenir sur des équipements de réseaux et télécommunications}
\ajoutSac{AC0112}{Comprendre l'architecture des systèmes numériques et les principes du codage de l'information}
\ajoutSac{AC0113}{Configurer les fonctions de base du réseau local}
\ajoutSac{AC0114}{Maîtriser les rôles et les principes fondamentaux des systèmes d'exploitation afin d'interagir avec ceux-ci pour la configuration et administration des réseaux et services fournis}
\ajoutSac{AC0115}{Identifier les dysfonctionnements du réseau local}
\ajoutSac{AC0116}{Installer un poste client}

\ajoutScompetence{RT2-Connecter}{\niveauA}





\ajoutScompetence{RT3-Programmer}{\niveauA}





% Le description
\ajoutSdescription{Dans cette SAÉ l'étudiant sera confronté à la découverte et la mise en
œuvre d'un premier réseau informatique. Il devra appréhender la
diversité de ses constituants et comprendre leurs interactions. Cette
compréhension est nécessaire avant toute intervention sur un élément
constitutif d'un réseau informatique. L'étudiant devra mettre en
pratique ses connaissances techniques de configuration de postes de
travail et d'équipements du réseau afin aboutir à un fonctionnement
stable.}

% Les ressources
\ajoutSressources{R101}{Initiation aux réseaux informatiques}
\ajoutSressources{R102}{Principes et architecture des réseaux}
\ajoutSressources{R103}{Réseaux locaux et équipements actifs}
\ajoutSressources{R104}{Fondamentaux des systèmes électroniques}
\ajoutSressources{R106}{Architecture des systèmes numériques et informatiques}
\ajoutSressources{R108}{Bases des systèmes d'exploitation}

% Livrable
\ajoutSlivrables{
\vspace{-5pt}
\begin{itemize}
\item
  Schéma réseau annoté avec le plan d'adressage et les services;
\item
  Démonstration technique commentée;
\item
  Rapport technique avec présentation diaporama.
\end{itemize}
L'étudiant s'approprie son portfolio. Des temps sont prévus pour qu'il y
synthétise sa production technique et son analyse argumentée.
}

% Mots-clés
\ajoutSmotscles{Réseau local, Connexion Internet, Équipements actifs.}
