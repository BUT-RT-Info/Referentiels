%%%%%%%%%%%%%%%%%%%%%%%%%%%%%%%%%
% Exemple de SAE
%%%%%%%%%%%%%%%%%%%%%%%%%%%%%%%%%

\nouvelexemple{Catalogue des vulnérabilités}

\ajoutESproblematique{Il s'agit de faire découvrir aux étudiants les différents types de
vulnérabilités pouvant exister dans un système informatique, ainsi que
les conséquences qu'elles peuvent engendrer.}

\ajoutESdescription{
Cette étude commencera par une recherche documentaire permettant
d'établir un glossaire des termes désignant les différents types de
vulnérabilités et de proposer pour chacun une définition vulgarisée.\\
Sans toutefois entrer dans une technique très poussée, il sera demandé
que chaque type de vulnérabilité soit illustré par un exemple concret
d'attaque qu'il rend possible.\\
Enfin, les conséquences de ces attaques seront décrites en termes de
gravité d'atteinte à la disponibilité, à l'intégrité et/ou à la
confidentialité des biens impactés.\\
L'étudiant, dès la fin du S1, prendra ainsi conscience de la nécessité
d'une bonne hygiène informatique, en découvrant~:
\begin{itemize}[topsep=5pt]
\item
  l'intérêt des bons mots de passe (nombre de caractères, complexité de
  l'alphabet)\,;
\item
  les sauvegardes de données (risque des supports, de la non
  duplication, \ldots)\,;
\item
  la faiblesse du facteur humain (ingénierie sociale, \ldots)\,;
\item
  les types de logiciels malveillants (chevaux de troyes, bombes
  logiques, virus, vers, \ldots)\,;
\item
  les sites Web malveillants\,;
\item
  les sites Web mal écrits\,;
\item
  les dépassement de tampon\,;
\item
  les usurpations diverses (\textabbrv{ARP}, \textabbrv{DNS}, \ldots)\,;
\item
  les écoutes de réseau.
\end{itemize}
Cette liste n'est évidemment pas limitative.\\
On pourra également utiliser les supports~:
\begin{itemize}[topsep=5pt]
\item
  Cybermalveillance~:
  \url{https://www.cybermalveillance.gouv.fr/bonnes-pratiques}
\item
  \textabbrv{MOOC} \textabbrv{ANSSI}~: \url{https://secnumacademie.gouv.fr/}
\item
  Malette CyberEdu~:
  \url{https://www.ssi.gouv.fr/entreprise/formations/secnumedu/contenu-pedagogique-cyberedu/}
\item
  et d'autres ressources aisément disponibles sur le Web.
\end{itemize}
}

\ajoutESformes{TP, projet.}



\ajoutESmodalite{Chaque étudiant ou groupe d'étudiants doit produire un rapport sous
forme de catalogue de vulnérabilités que l'on pourrait destiner à une
campagne de sensibilisation pour \og grand public\fg{}. Le format de
\og 1 vulnérabilité, 1 ou 2 pages\fg{} doit être le format à viser pour
imposer une description synthétique et éviter les copier/coller
compulsifs avec détails techniques superflus. Les exemples d'attaques
présentés doivent être réalistes et compréhensibles par des non
spécialistes.}