%%%%%%%%%%%%%%%%%%%%%%%%%%%%%%%%%
% Exemple de SAE
%%%%%%%%%%%%%%%%%%%%%%%%%%%%%%%%%

\nouvelexemple{Concevoir un réseau informatique pour une manifestation évènementielle}

\ajoutESproblematique{L'objectif est d'être capable d'installer un réseau informatique avec
l'interconnexion de switches, un routeur d'accès Internet (\textabbrv{FO} ou \textabbrv{4G}), un
point d'accès WiFi, savoir effectuer la segmentation du réseau,
configurer le plan d'adressage (statique/\textabbrv{DHCP}) et le routage,
installation des postes clients (Windows, Linux) pour les utilisateurs.}

\ajoutESdescription{
Il s'agit de :
\begin{itemize}
\item
  répondre au cahier des charges;
\item
  savoir dimensionner les puissances électriques de l'installation
  réseau et télécom;
\item
  savoir dimensionner les équipements du réseau;
\item
  assurer l'interconnexion d'une installation temporaire;
\item
  garantir la sécurité des ports des switchs;
\item
  utiliser des \textabbrv{VLAN}s distincts (data, management);
\item
  mettre en place une politique sécurisée de mots de passe
  (utilisateurs, équipements);
\item
  savoir monitorer les éléments actifs et observer le trafic sur réseau;
\item
  savoir diagnostiquer les dysfonctionnements.
\end{itemize}
}

\ajoutESformes{TP, projet}



\ajoutESmodalite{L'étudiant doit être capable de retranscrire l'architecture réseau grâce
à un outil de schéma réseau et de le détailler avec l'ensemble des
paramètres du cahier des charges et des configurations.\\
Pour l'évaluation, les étudiants doivent être capables de faire une
démonstration technique progressive par tests unitaires, ou un rapport
de synthèse ou une présentation avec diaporama.\\
On peut utiliser un logiciel de dessin technique pour les schémas
réseaux en utilisant à bon escient les symboles et pictogrammes (switch,
routeur, firewall, WiFi, \ldots).}