%%%%%%%%%%%%%%%%%%%%%%%%%%%%%%%%%
% Exemple de SAE
%%%%%%%%%%%%%%%%%%%%%%%%%%%%%%%%%

\nouvelexemple{Cyberattaque~: exploitation de vulnérabilités}

\ajoutESproblematique{Il est essentiel dans une entreprise, quelle qu'en soit la taille, de
sensibiliser le personnel avec des exemples simples (et si possible
spectaculaires), aux conséquences d'une mauvaise hygiène informatique.\\
Il est du rôle du responsable du SI de savoir enseigner les bonnes
pratiques de la façon la plus simple et la plus convaincante possible.}

\ajoutESdescription{
Dans la continuité de la SAÉ ``sensibilisation à l'hygiène informatique
et à la cybersécurité'', l'objectif est d'aborder la cybersécurité sous
un angle plus technique.\\
Les étudiants seront amenés à reproduire des vulnérabilités et des
attaques dans un environnement d'étude spécifiquement mis en place~: un
réseau autonome réel ou simulé qui comportera quelques éléments actifs,
serveurs et clients opérationnels.\\
Cette étude permettra de se familiariser avec l'utilisation des
principaux outils utilisés tout autant par les hackers que par les
administrateurs des systèmes (nmap, john the ripper, burp suite, scapy,
metasploit, \ldots) afin d'exploiter les vulnérabilités volontairement
introduites dans la configuration.\\
La maquette devra illustrer les techniques d'exploitations d'un nombre
de vulnérabilités convenu en début d'étude, qui seront choisies dans le
``catalogue d'attaques'' produit dans la SAÉ précédente.\\
Les conséquences de ces attaques seront ici encore décrites en termes de
gravité d'atteinte à la disponibilité, à l'intégrité et/ou à la
confidentialité des biens impactés, afin d'insister sur l'importance de
l'hygiène informatique en contexte professionnel.
}

\ajoutESformes{\vspace{-5pt}
\begin{itemize}[topsep=5pt]
\item
  Sur des heures encadrées~: TP de R201, R202, R203, R204, R207, R208,
  R209
\item
  Sur des heures non encadrées~:
  \begin{itemize}
  \item
    Recherche documentaire
  \item
    Mise en place de maquette, configuration et test
  \item
    Rédaction de livrables, préparation de présentations
  \end{itemize}
\end{itemize}}



\ajoutESmodalite{\vspace{-5pt}
\begin{itemize}[topsep=5pt]
\item
  Une maquette autonome, réelle ou virtuelle, d'un réseau de \textabbrv{PME},
  intégrant des vulnérabilités et des
  outils permettant de les exploiter\,;
\item
  Une présentation de cette maquette, des vulnérabilités et des attaques
  sera faite dans le cadre d'un
  rapport et/ou d'une soutenance et/ou d'une démonstration. L'anglais
  pourra être demandé pour tout ou
  partie des livrables ou présentations.
\end{itemize}}