%%%%%%%%%%%%%%%%%%%%%%%%%%%%%%%%%
% Exemple de SAE
%%%%%%%%%%%%%%%%%%%%%%%%%%%%%%%%%

\nouvelexemple{Sensibilisation aux risques numériques}

\ajoutESproblematique{Il s'agit de faire prendre conscience aux étudiants des risques qu'ils
peuvent encourir sans considérer avec attention l'usage de leur
environnement numérique.}

\ajoutESdescription{
On pourra faire un focus particulier sur les points suivants.
\begin{itemize}[topsep=5pt]
\item
  Protéger ses accès avec des mots de passe solides\,; en TP on peut
  mettre en place l'usage d'un gestionnaire
  de mot de passe tel que Keepass (multi-plateformes Windows, Linux,
  Mac, Android, iPhone/iPad)\,;
\item
  Sécurité numérique~: sauvegarder ses données régulièrement\,;
\item
  Appliquer les mises à jour de sécurité sur tous ses appareils (\textabbrv{PC},
  tablettes, téléphones\ldots), et ce,
  dès qu'elles sont proposées\,;
\item
  Utiliser un antivirus\,;
\item
  Télécharger ses applications uniquement sur les sites officiels\,;
\item
  Sécurité numérique~: se méfier des messages inattendus\,;
\item
  Vérifier les sites sur lesquels sont effectués ses achats\,;
\item
  Maîtriser ses réseaux sociaux\,;
\item
  Séparer ses usages personnels et professionnels\,;
\item
  Sécurité numérique~: éviter les réseaux WiFi publics ou inconnus\,; il
  est possible de faire un TP (démo)
  sur l'usage d'un faux point d'accès WiFi et de collecter des
  identifiants de réseaux sociaux, \ldots.
\item
  Analyser les trames non chiffrées des protocoles (\textabbrv{TELNET}, \textabbrv{FTP}, \textabbrv{SMTP},
  \textabbrv{POP}, \textabbrv{IMAP}, \textabbrv{RTP}, \ldots) avec Wireshark,
  en extraire des champs significatifs avec Analyse/Follow/\textabbrv{TCP} Stream
  (ou \textabbrv{HTTP} Stream)\,; on peut utiliser
  un site Web (création personnelle ou sur Internet) contenant un
  formulaire d'enregistrement (ex:
  \url{http://www.supportduweb.com/signup.html})\,;
\item
  Utiliser des outils de codage de l'information (ex:
  \url{https://www.dcode.fr/fr}).
\end{itemize}
On pourra également utiliser les supports~:
\begin{itemize}[topsep=5pt]
\item
  Cybermalveillance~: \url{https://www.cybermalveillance.gouv.fr/bonnes-pratiques}
\item
  \textabbrv{MOOC} \textabbrv{ANSSI}~: \url{https://secnumacademie.gouv.fr/}
\end{itemize}
}

\ajoutESformes{TP, projet.}



\ajoutESmodalite{L'étudiant doit démontrer qu'il est capable de présenter de façon
claire, concise et vulgarisée les risques et les bons usages des outils
numériques tel que pourrait le faire un responsable d'un service
informatique à un collaborateur néo-entrant dans son entreprise. Cette
démonstration devra s'accompagner d'exemples concrets.\\
Cette démonstration pourra se faire sous la forme de présentation orale
ou écrite et accompagnée de différents média (infographie, affiche,
vidéo\ldots).}