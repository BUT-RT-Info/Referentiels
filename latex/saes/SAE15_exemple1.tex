%%%%%%%%%%%%%%%%%%%%%%%%%%%%%%%%%
% Exemple de SAE
%%%%%%%%%%%%%%%%%%%%%%%%%%%%%%%%%

\nouvelexemple{Collecter, traiter, présenter et publier des données}

\ajoutESproblematique{Cette SAÉ place l'étudiant dans le contexte d'un premier projet de
développement informatique. Le projet traite d'une activité fréquemment
demandée au professionnel R\&T : mettre à disposition de ses
collaborateurs une information extraite de différentes sources de
données, par exemple :
\begin{itemize}
\item
  le recensement des équipements informatiques, des services, des
  personnels;
\item
  l'état de réservations des salles mutualisées dans un bâtiment;
\item
  une synthèse de l'utilisation du réseau ou d'un de ses services, en
  travaillant sur des fichiers journaux
  (logs);
\item
  la visibilité de l'entreprise sur différents sites ou pages Web.
\end{itemize}
Dans ce contexte, le professionnel R\&T est amené à collecter des
données, les traiter pour en extraire une représentation
exploitable/parlante, puis en publier la présentation. L'objectif global
est d'automatiser au mieux les différentes étapes de son travail.}

\ajoutESdescription{
L'étudiant s'initiera aux différentes étapes d'un projet informatique :
la mise en place de son environnement de travail pour un système
d'exploitation donné, la programmation du traitement des données (en
s'appuyant sur les fondamentaux de programmation voire en explorant des
bibliothèques spécifiques éventuellement en anglais) et la présentation
de ses résultats via un site Web. Il pourra s'appuyer sur les techniques
de gestion de projet..\\
Il sera demandé à l'étudiant (individuellement ou en petit groupe) de
traiter des données simples (ne nécessitant pas une structuration
complexe dans le code informatique ni une base de données), avec
différentes étapes :
\begin{itemize}
\item
  Préparer l'environnement de travail pour accès distance aux ressources
  :
  \begin{itemize}
  \item
    mise en place de l'arborescence;
  \item
    installation/configuration des outils pour le développement;
  \item
    vérification de la connectivité, des droits d'accès;
  \item
    mise en place et configuration d'un système de versionnement (par
    ex: git, svn), etc.
  \end{itemize}
\item
  Acquérir des données (locales ou distantes) en les enregistrant dans
  un fichier texte (en se focalisant
  sur des données relativement simples à traiter). Les données pourront
  par exemple provenir :
  \begin{itemize}
  \item
    de sites Web;
  \item
    d'\textabbrv{API}, par exemple l'\textabbrv{API} Google pour cartographie permettant de
    traiter des données de géolocalisation,
    ou des sources ouvertes;
  \item
    de commandes locales (état de la machine) ou réseaux (état du
    réseau).
  \end{itemize}
\item
  Traiter les données pour préparer les éléments nécessaires à leur
  publication en se documentant (au
  besoin sur des bibliothèques spécifiques). Le traitement pourra par
  exemple consister à :
  \begin{itemize}
  \item
    isoler/choisir/organiser les informations pertinentes;
  \item
    extraire des statistiques (moyennes, histogrammes);
  \item
    produire des représentations graphiques (nuage de mots, tableaux
    comparatifs).
  \end{itemize}
\item
  Générer un document pour présenter les données collectées et le
  publier :
  \begin{itemize}
  \item
    le document pourra être un fichier texte simple ou structuré (page
    Markdown voire page Web statique);
  \item
    le document sera ensuite publié sur un serveur distant (en utilisant
    par exemple un serveur web non
    sécurisé);
  \item
    la publication (c'est-à-dire l'action de déposer le document
    lui-même) pourra être automatisée par
    un script, par exemple en déployant une archive dans le dossier
    public d'un serveur web. (Remarque :
    il ne s'agit pas ici de créer une présentation/une page Web
    dynamique dépendant d'une requête).
  \end{itemize}
\end{itemize}
}

\ajoutESformes{TP, projet.}



\ajoutESmodalite{\vspace{-5pt}
\begin{itemize}
\item
  Codes informatiques développés : l'étudiant devra fournir l'ensemble
  de scripts et de codes informatiques,
  permettant la réalisation fonctionnelle du travail demandé. Sa
  production devra être documentée, dans
  le code et dans un compte rendu des étapes d'installation et de
  configuration qu'il aura réalisé.
\item
  Démonstration de l'installation, de l'accès, de la validité et de la
  cohérence des données présentées.
  Il devra également y expliquer sa démarche, ses choix
  d'implémentation, les analyses et outils mathématiques
  et scientifiques utilisés pour présenter les données et interpréter
  les résultats obtenus.
\end{itemize}}