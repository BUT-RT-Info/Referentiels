
%%%%%%%%%%%%%%%%%%%%%%%%%%%%%%%%%%%%%%%%%%%%%%%%%%%%%%%%%
%% Affichae des SAE (et leur exemple) pour un semestre %%
%%%%%%%%%%%%%%%%%%%%%%%%%%%%%%%%%%%%%%%%%%%%%%%%%%%%%%%%%
% \affichageSAE{code_semestre} avec code_semestre=A pour le S1, B pour le S2
\newcounter{nosae}
\newcounter{noexemple}
\newcounter{noac}
\newcommand{\affichageSAE}[1]{
		% \addcontentsline{toc}{subsection}{Fiches SAÉs}
    \xdef\sem{#1}
    
    \forLoop{1}{\value{cptS\sem}}{nosae}{ % Pour chaque SAE
        %A\Alph{noressource}
        \xdef\currentSAE{\sem\Alph{nosae}}

        \xdef\saecode{\csname Scode\currentSAE\endcsname}
	    \xdef\saenom{\csname Sname\currentSAE\endcsname}
        %\addtocontents{toc}{hihi} %\saecode : \saenom}
        % \addcontentsline{toc}{subsubsection}{{\saenom}}
        \phantomsection\label{lab\currentSAE}
        \tableauSAE{\currentSAE}
				\newpage
        \forLoop{1}{\value{cptSexemple\currentSAE}}{noexemple}{
            \tableauExempleSAE{\currentSAE}{\Alph{noexemple}}
        \newpage
        }

    }
}

% %%%%%%%%%%%%%%%%%%%%%%%%%%%%%%%%%%%%%%%%%%%%%%%%%%%%%%%%%%%%%%%%%%%%%%%%%%%%%%%%%%% %
% Tableau listant toutes les SAé et leur exemple en guise d'introduction au semestre  %
% %%%%%%%%%%%%%%%%%%%%%%%%%%%%%%%%%%%%%%%%%%%%%%%%%%%%%%%%%%%%%%%%%%%%%%%%%%%%%%%%%%% %
\newcommand{\listeTitreSAE}[1]{
    \setlength{\tabcolsep}{0.125cm} % Marge des colonnes
    \setlength{\extrarowheight}{2pt} % Marge des lignes
\begin{center}\small
    \begin{tabular}[t]{|X|Y|Z|}
        \hline {\bfseries Code} & {\bfseries Titre} & {\bfseries Pages} \\ \hline

        \forLoop{1}{\value{cptS#1}}{nosae}{ % Pour chaque SAE
            %A\Alph{noressource}
            \xdef\currentSAE{#1\Alph{nosae}} % code de la SAE
            \xdef\saecode{\csname Scode\currentSAE\endcsname}
            \xdef\saelink{\csname Slinkcode\currentSAE\endcsname}
            \xdef\saetitre{\csname Sname\currentSAE\endcsname}
            {\bfseries \hyperlink{sae:\saecode}{\textcolor{saeC}{\saecode}}}
            &
            {\bfseries {\saetitre}}
						&
						\pageref{lab\currentSAE}
						
            
            \ifthenelse{\value{cptSexemple\currentSAE}>0}{
							\forLoop{1}{\value{cptSexemple\currentSAE}}{noexemple}{
										\tabularnewline
											\xdef\exemplecode{\currentSAE\Alph{noexemple}}
											%\xdef\titreexemple{\csname ESname\exemplecode\endcsname}
											& 
											\itshape \hyperlink{exemple:\exemplecode}{\textcolor{exempleC}{Exemple \arabic{noexemple}}}~: 
											{\csname ESname\exemplecode\endcsname}
											&
											% \tabularnewline
							}
						}{}
            %\end{tabular}
            \tabularnewline
            \hline
        }
    \end{tabular}
		\end{center}
}

% %%%%%%%%%%%%%%%%%%%%%%%%%%%%%%%%%%%%%%%%%%%%%%%%%%%%%%%%%%%%%%%%%%%%%%%%%%%%%%%%%%%%%%%%%% %
% Tableau listant toutes les ressources et leur exemple en guise d'introduction au semestre  %
% -> obsolète avec le listeTitreRessourceIndex d'Emmanuel
% %%%%%%%%%%%%%%%%%%%%%%%%%%%%%%%%%%%%%%%%%%%%%%%%%%%%%%%%%%%%%%%%%%%%%%%%%%%%%%%%%%%%%%%%%% %
\newcommand{\listeTitreRessource}[1]{
    \setlength{\tabcolsep}{0.125cm} % Marge des colonnes
    \setlength{\extrarowheight}{2pt} % Marge des lignes
		
    \begin{tabular}[t]{|X|Y|Z|}
        \hline {\bfseries Code} & {\bfseries Nom} \\
        \hline

        \forLoop{1}{\value{cptR#1}}{noressource}{ % Pour chaque SAE
            %A\Alph{noressource}
            \def\CODE{#1\Alph{noressource}} % code de la SAE
            \xdef\rescode{\csname Rcode\CODE\endcsname}
            \xdef\resnom{\csname Rname\CODE\endcsname}
            {\bfseries \hyperlink{res:\rescode}{\textcolor{ressourceC}{\rescode}}}
            &
                {\resnom}
            \tabularnewline
            \hline
        }
    \end{tabular}
}

% Table des ressources pour index début document
\newcommand{\listeTitreRessourceIndex}[1]{
    \setlength{\tabcolsep}{0.125cm} % Marge des colonnes
    \setlength{\extrarowheight}{2pt} % Marge des lignes
    \begin{center}\small
    \begin{tabular}[t]{|X|Y|Z|}
        \hline {\bfseries Code} & {\bfseries Nom} & {\bfseries Page} \\
        \hline

        \forLoop{1}{\value{cptR#1}}{noressource}{ % Pour chaque SAE
            %A\Alph{noressource}
            \xdef\CODE{#1\Alph{noressource}} % code de la SAE
            \xdef\rescode{\csname Rcode\CODE\endcsname}
            \xdef\resnom{\csname Rname\CODE\endcsname}
            {\bfseries \hyperlink{res:\rescode}{\textcolor{ressourceC}{\rescode}}}
            &
                {\resnom}
            &
                \pageref{res\CODE}
            \tabularnewline
            \hline
        }
    \end{tabular}
    \end{center}
}

% Tableau listant toutes les SAé du semestre (sans les exemples), 
% avec numéros de page pour tableau index
\newcommand{\listeTitreSAEIndex}[1]{
    \setlength{\tabcolsep}{0.125cm} % Marge des colonnes
    \setlength{\extrarowheight}{2pt} % Marge des lignes
    \begin{center}\small
    \begin{tabular}[t]{|X|Y|Z|}
        \hline {\bfseries Code} & {\bfseries Titre} & {\bfseries Page} \\ \hline

        \forLoop{1}{\value{cptS#1}}{nosae}{ % Pour chaque SAE
            %A\Alph{noressource}
            \xdef\currentSAE{#1\Alph{nosae}} % code de la SAE
            \xdef\saecode{\csname Scode\currentSAE\endcsname}
            \xdef\saelink{\csname Slinkcode\currentSAE\endcsname}
            \xdef\saetitre{\csname Sname\currentSAE\endcsname}
            {\bfseries \hyperlink{sae:\saecode}{\textcolor{saeC}{\saecode}}}
            &
            {\saetitre}
            &
            \pageref{lab\currentSAE}

            \tabularnewline
            \hline
        }
    \end{tabular}
    \end{center}
}


%%%%%%%%%%%%%%%%%%%%%%%%%%%%%%
%% Affichage des ressources %%
%%%%%%%%%%%%%%%%%%%%%%%%%%%%%%
\newcounter{noressource}
\newcommand{\affichageRessource}[1]{
    \xdef\sem{#1}
    \forLoop{1}{\value{cptR\sem}}{noressource}{
        %A\Alph{noressource}
        \phantomsection\label{res#1\Alph{noressource}}
        \tableauRessource{#1\Alph{noressource}}
        \newpage
}

}