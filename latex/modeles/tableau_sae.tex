
% Le tableau complet présentant une sae
\newcommand\tableauSAE[1]{

\xdef\CODE{#1} % code de la sae (par ex: AA avec A=1er semestre, A=1ère ressource)

\setlength{\tabcolsep}{0.125cm} % Marge des colonnes
\setlength{\extrarowheight}{2pt} % Marge des lignes
\arrayrulecolor{saeC}

% 1er tableau : Nom/Code/Semestre
\begin{tabular}[t]{|P|Q|V|}
\hline % 1ère ligne
	\xdef\saecode{\csname Scode\CODE\endcsname}
	\xdef\saelink{\csname Slinkcode\CODE\endcsname}
	\cellcolor{saeC} \textcolor{compCAp!20!white}{\bfseries % SAÉ
	\hypertarget{sae:\saecode}{\saecode}}
	& \multicolumn{2}{l|}{
	\cellcolor{saeC}\textcolor{compCAp!20!white}{\bfseries \csname Sname\CODE\endcsname}} \\
\hline \hline % 2ème ligne
	% {\bfseries Code}
	% & \csname Scode\CODE\endcsname
	% &
	\textcolor{saeC}{\bfseries Semestre}
	& \multicolumn{2}{l|}{\csname Ssem\CODE\endcsname} \\
	% & \csname Ssem\CODE\endcsname \\
\hline
\hline
	\textcolor{saeC}{\bfseries Heures }
	&
	\textcolor{saeC}{\bfseries Formation encadrée}
		& {\csname Straining\CODE\endcsname}h,
			dont {\csname Stp\CODE\endcsname}h de TP \\
\cline{2-3}
	& \textcolor{saeC}{\bfseries Projet}
		& {\csname Sprojet\CODE\endcsname}h \\
%	{\bfseries Formation encadrée}
%	& {\csname Straining\CODE\endcsname}h
%	& {\bfseries dont TP}
%	& {\csname Stp\CODE\endcsname}h	\\
%\hline
	%{\bfseries Projet}
	% & \multicolumn{3}{l|}{{\csname Sprojet\CODE\endcsname}h} \\
\hline
\end{tabular}

% 2ème tableau
%\begin{tabular}[t]{|P|S|P|S|P|S|}
%% 	Heures
%{\bfseries Formation encadrée} & {\csname Straining\CODE\endcsname}h
%& {\bfseries dont TP}          & {\csname Stp\CODE\endcsname}h
%& {\bfseries Projet}           & {\csname Sprojet\CODE\endcsname}h \\
%\hline
%\end{tabular}

% 3ème tableau : descriptif
\begin{tabular}[t]{|P|T|}
\hline
{ \setlength{\extrarowheight}{0pt}
	\begin{tabular}[t]{@{}P@{}}
	\textcolor{saeC}{\bfseries Objectifs et} \\
	\textcolor{saeC}{\bfseries problématique} \\
	\textcolor{saeC}{\bfseries professionnelle}
	\end{tabular}
}
& \tableauChampLong{\csname Sdescriptif\CODE\endcsname} \\
\hline
\end{tabular}

% 3ème tableau : compétences et apprentissages critiques
\begin{tabular}[t]{|C|C|C|}
\hline
\multicolumn{3}{|c|}{\textcolor{saeC}{\bfseries Compétence(s) ciblée(s), coefficient(s) et apprentissage(s) critique(s) couvert(s)}} \\
\hline
	\hyperlink{comp:RT1}{\textcolor{compCA}{\textit{\csname Scomp\CODE A\endcsname}}} {\csname Sniveau\CODE A\endcsname}
	&
	\hyperlink{comp:RT2}{\textcolor{compCB}{\textit{\csname Scomp\CODE B\endcsname}}} {\csname Sniveau\CODE B\endcsname}
	&
	\hyperlink{comp:RT3}{\textcolor{compCC}{\textit{\csname Scomp\CODE C\endcsname}}} {\csname Sniveau\CODE C\endcsname}
\\
\hline
	\ifcsdef{Scoeff\CODE A}{coef. {\csname Scoeff\CODE A\endcsname}}{} &
	\ifcsdef{Scoeff\CODE B}{coef. {\csname Scoeff\CODE B\endcsname}}{} &
	\ifcsdef{Scoeff\CODE C}{coef. {\csname Scoeff\CODE C\endcsname}}{}
\\
\hline % AC de RT1
{\listeAC{S}{\CODE}{A}{RT1}
}
& % AC de RT2
{\listeAC{S}{\CODE}{B}{RT2}
}
& % AC de RT3
{\listeAC{S}{\CODE}{C}{RT3}
}
\\
\hline
\end{tabular}

% Dernier tableau : Ressources, livrable, mots clés
\begin{tabular}[t]{|P|T|}
\hline
{ \setlength{\extrarowheight}{0pt}
	\begin{tabular}[t]{@{}P@{}}
	\textcolor{saeC}{\bfseries Ressources mobilisées} \\
	\textcolor{saeC}{\bfseries et combinées} \\
	\end{tabular}
}
& \listeRessources{\CODE} \\
\hline
\end{tabular}

\begin{tabular}[t]{|P|T|}
\hline
{ \setlength{\extrarowheight}{0pt}
	\begin{tabular}[t]{@{}P@{}}
	\textcolor{saeC}{\bfseries Type de livrable ou} \\
	\textcolor{saeC}{\bfseries de production} \\
	\textcolor{saeC}{\itshape (traces pour le} \\
	\textcolor{saeC}{\itshape portfolio)} \\
	\end{tabular}
}
& \tableauChampLong{\csname Slivrables\CODE\endcsname} \\
\hline
\textcolor{saeC}{\bfseries Mots-clés} & {\csname Smotscles\CODE\endcsname} \\
\hline
\hline
{ \setlength{\extrarowheight}{0pt}
	\begin{tabular}[t]{@{}P@{}}
	\textcolor{saeC}{\bfseries Exemples de} \\
	\textcolor{saeC}{\bfseries mise en oeuvre} \\
	\end{tabular}
}
	&
\tableauExemples{\CODE} 
\\

\hline
\end{tabular}

}