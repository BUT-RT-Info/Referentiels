
% Le tableau complet présentant une sae
\newcommand\tableauSAE[1]{

\xdef\CODE{#1} % code de la sae (par ex: AA avec A=1er semestre, A=1ère ressource)

\setlength{\tabcolsep}{0.125cm} % Marge des colonnes
\setlength{\extrarowheight}{2pt} % Marge des lignes

% 1er tableau : Nom/Code/Semestre
\begin{tabular}[t]{|P|Q|P|Q|}
\hline % 1ère ligne
	{\bfseries Titre de la SAÉ}
	& \multicolumn{3}{l|}{\bfseries \csname Sname\CODE\endcsname} \\
\hline % 2ème ligne
	{\bfseries Code}
	& \csname Scode\CODE\endcsname
	& {\bfseries Semestre}
	& \csname Ssem\CODE\endcsname \\
\hline
\hline
	{\bfseries Formation encadrée}
	& {\csname Straining\CODE\endcsname}h
	& {\bfseries dont TP}
	& {\csname Stp\CODE\endcsname}h	\\
\hline
	{\bfseries Projet}
	& \multicolumn{3}{l|}{{\csname Sprojet\CODE\endcsname}h} \\
\hline
\end{tabular}

% 2ème tableau
%\begin{tabular}[t]{|P|S|P|S|P|S|}
%% 	Heures
%{\bfseries Formation encadrée} & {\csname Straining\CODE\endcsname}h
%& {\bfseries dont TP}          & {\csname Stp\CODE\endcsname}h
%& {\bfseries Projet}           & {\csname Sprojet\CODE\endcsname}h \\
%\hline
%\end{tabular}

% 3ème tableau : descriptif
\begin{tabular}[t]{|P|T|}
\hline
{ \setlength{\extrarowheight}{0pt}
	\begin{tabular}[t]{@{}P@{}}
	{\bfseries Objectifs et} \\
	{\bfseries problématique} \\
	{\bfseries professionnelle}
	\end{tabular}
}
& \tableauChampLong{\csname Sdescriptif\CODE\endcsname} \\
\hline
\end{tabular}

% 3ème tableau : compétences et apprentissages critiques
\begin{tabular}[t]{|C|C|C|}
\hline
\multicolumn{3}{|c|}{\bfseries Compétence(s) ciblées et apprentissage(s) critique(s) couvert(s)} \\
\hline
	\textit{\csname Scomp\CODE A\endcsname} | {\csname Sniveau\CODE A\endcsname}
	&
	\textit{\csname Scomp\CODE B\endcsname} | {\csname Sniveau\CODE B\endcsname}
	&
	\textit{\csname Scomp\CODE C\endcsname} | {\csname Sniveau\CODE C\endcsname}
\\
\hline % AC de RT1
{\listeAC{S}{\CODE}{A}
}
& % AC de RT2
{\listeAC{S}{\CODE}{B}
}
& % AC de RT3
{\listeAC{S}{\CODE}{C}
}
\\
\hline
\end{tabular}

% Dernier tableau : Ressources, livrable, mots clés
\begin{tabular}[t]{|P|T|}
\hline
{ \setlength{\extrarowheight}{0pt}
	\begin{tabular}[t]{@{}P@{}}
	{\bfseries Ressources mobilisées} \\
	{\bfseries et combinées} \\
	\end{tabular}
}
& \listeRessources{\CODE} \\
\hline
{ \setlength{\extrarowheight}{0pt}
	\begin{tabular}[t]{@{}P@{}}
	{\bfseries Type de livrable ou} \\
	{\bfseries de production} \\
	{\itshape (traces pour le} \\
	{\itshape portfolio)} \\
	\end{tabular}
}
& \tableauChampLong{\csname Slivrables\CODE\endcsname} \\
\hline
{\bfseries Mots-clés} & {\csname Smotscles\CODE\endcsname} \\
\hline

\end{tabular}

}