
% Le tableau complet présentant une sae
\newcommand\tableauSAE[1]{

\xdef\CODE{#1} % code de la sae (par ex: AA avec A=1er semestre, A=1ère ressource)

\setlength{\tabcolsep}{0.125cm} % Marge des colonnes
\setlength{\extrarowheight}{2pt} % Marge des lignes

\arrayrulecolor{saeC}

% 1er tableau : Nom/Code/Semestre
\begin{tabular}[t]{|P|Q|V|}
\hline % 1ère ligne
	\xdef\saecode{\csname Scode\CODE\endcsname}
	\xdef\saelink{\csname Slinkcode\CODE\endcsname}

	\cellcolor{saeC} \textcolor{compCAp!20!white}{\bfseries % SAÉ
	\hypertarget{sae:\saecode}{\saecode}}
	& \multicolumn{2}{l|}{
	\cellcolor{saeC}\textcolor{compCAp!20!white}{\bfseries \csname Sname\CODE\endcsname}} \\
\hline \hline % 2ème ligne
	% {\bfseries Code}
	% & \csname Scode\CODE\endcsname
	% &
	\textcolor{saeC}{\bfseries Semestre}
	& \multicolumn{2}{l|}{\csname Ssem\CODE\endcsname} \\
	% & \csname Ssem\CODE\endcsname \\
\hline
\hline
	\textcolor{saeC}{\bfseries Heures }
	&
	\textcolor{saeC}{\bfseries Formation encadrée}
		& {\csname Straining\CODE\endcsname}h,
			dont {\csname Stp\CODE\endcsname}h de TP \\
\cline{2-3}
	& \textcolor{saeC}{\bfseries Projet}
		& {\csname Sprojet\CODE\endcsname}h \\
\hline
\hline % 4ème ligne : Parcours
	\textcolor{saeC}{\bfseries Parcours}
	& \multicolumn{2}{l|}{Cyber, DevCloud, PilPro, IOM, ROM} \\
\hline
\end{tabular}


% 3ème tableau : objectifs
\begin{tabular}{|G|}
	\hline
	\textcolor{saeC}{\bfseries Objectifs et problématique professionnelle} \\
	\hline
	{\csname Sobjectifs\CODE\endcsname}
	\\
\hline
\end{tabular}

% 3ème tableau : description
\begin{tabular}{|G|}
	\hline
	\textcolor{saeC}{\bfseries Description générique} \\
	\hline
	{\csname Sdescription\CODE\endcsname}
	\\
\hline
\end{tabular}

% 4ème tableau : livrables
\begin{tabular}{|G|}
	\hline
	\textcolor{saeC}{\bfseries Livrable ou production} \\
	\hline
	{\csname Slivrables\CODE\endcsname}
	\\
\hline
\end{tabular}

% Compétences et apprentissages critiques
\begin{tabular}[t]{|P|T|}
\hline
\textcolor{saeC}{\bfseries \makecell[l]{Compétences / \\
Apprentissages \\ critiques} } & %\listeACs{\CODE}
\tableauChampLong{\listeSCompetencesEtACs{\CODE}} \\
\hline
\end{tabular}

% Ressources mobilisées
\begin{tabular}[t]{|P|T|}
\hline
\textcolor{saeC}{\bfseries \makecell[l]{Ressources \\
combinées} } & %\listeACs{\CODE}
\tableauChampLong{\listeRessources{\CODE}} \\
\hline
\end{tabular}

% Mots-clés
\begin{tabular}[t]{|P|T|}
\hline
\textcolor{saeC}{\bfseries Mots-clés } & %\listeACs{\CODE}
{\csname Smotscles\CODE\endcsname} \\
\hline
\end{tabular}

% Exemples de mise en oeuvre
\begin{tabular}[t]{|P|T|}
\hline
\textcolor{saeC}{\bfseries \makecell[l]{Exemples de \\
mise en oeuvre} } & %\listeACs{\CODE}
\tableauExemples{\CODE}  \\
\hline
\end{tabular}


}

