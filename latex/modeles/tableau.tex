

% Les formats de colonnes
\newcolumntype{P}{>{}m{3.6cm}<{}}  % Pour la colonne d'entete

%% Pour les champs horaires (lorsqu'il y en a 2)
\newcolumntype{Q}{>{}m{4.6cm}<{}}  % En 3 (4.1 pour équiréparti)

%% Pour les champs horaire lorsqu'il y en a 3
\newcolumntype{S}{>{}m{1.69cm}<{}}

%% Pour les compétences et les ACs
\newcolumntype{C}{>{}p{5.555cm}<{}}

%% Pour les champs de description (2de colonne)
\newcolumntype{T}{>{}p{13.325cm}<{}} % 12.825cm

\newcolumntype{V}{>{}p{8.45cm}<{}} % 12.825cm

% Pour les tables d'index
\newcolumntype{X}{>{}m{1.2cm}<{}} % Code
\newcolumntype{Y}{>{}p{10.50cm}<{}} % Nom
\newcolumntype{Z}{>{}w{c}{1.0cm}<{}} % No page

% Compteur de ligne pour les tableaux
\newcounter{noligne}

%%%%%%%%%%%%%%%%%%%%%%%%%%%%%%%%%%%%%%%%
%% Liste des apprentissages critiques %%
%%%%%%%%%%%%%%%%%%%%%%%%%%%%%%%%%%%%%%%%

% \listeAC{R}{code_ressource}{lettre_competence} pour une ressource
% \listeAC{S}{code_ressource}{lettre_competence} pour une sae
\newcommand*{\listeAC}[4]{%
\begin{tabular}[t]{@{}C@{}}%
    \xdef\TYPE{#1} % R pour ressource, S pour SAE
	\xdef\CODE{#2} % le code de la ressource sous la forme AA
	\xdef\COMP{#3} % le code de la compétence sous la forme A
	\xdef\targetcomp{#4}
  \forLoop{1}{\value{cpt\TYPE ac\CODE\COMP}}{noligne}{
  		\xdef\accode{\csname \TYPE accode\CODE\COMP\Alph{noligne}\endcsname}
		\hyperlink{comp:\targetcomp}{\textcolor{compC\COMP}{\accode}} {\csname \TYPE acname\CODE\COMP\Alph{noligne}\endcsname}
   \tabularnewline %
   }
\end{tabular}%
}%


% Liste des SAE (dans un tableau), utilisée pour décrire les SAE d'une ressource
\newcommand*{\listeSAE}[1]{%
	\xdef\CODE{#1} % le code de la ressource sous la forme AA
		\ifthenelse{\equal{\value{cptRsae\CODE}}{0}}{ % Si pas de prérequis
		Aucune
		}{
		\begin{tabular}[t]{@{}T@{}}%
			\forLoop{1}{\value{cptRsae\CODE}}{noligne}{
				\xdef\saecode{\csname Rsaecode\CODE\Alph{noligne}\endcsname}
				\hyperlink{sae:\saecode}{\textcolor{saeC}{\saecode}}~
						{\csname Rsaename\CODE\Alph{noligne}\endcsname}
			 \tabularnewline %
			 }
		\end{tabular}%
	}
}%


% Liste des pré-requis
\newcommand*{\listePrerequis}[1]{%
	\xdef\CODE{#1} % le code de la ressource sous la forme AA
	\ifthenelse{\equal{\value{cptRprerequis\CODE}}{0}}{ % Si pas de prérequis
Aucun
}{ % Si des pré-requis
\begin{tabular}[t]{@{}T@{}}%

  \forLoop{1}{\value{cptRprerequis\CODE}}{noligne}{
  		\xdef\rescode{\csname Rprerequiscode\CODE\Alph{noligne}\endcsname}
		\hyperlink{res:\rescode}{\textcolor{ressourceC}{\rescode}} {\csname Rprerequisname\CODE\Alph{noligne}\endcsname}
   \tabularnewline %
   }
\end{tabular}%
}
}%



% Descriptif detaillé
\newcommand*{\tableauDescriptif}[1]{%
	\xdef\CODE{#1} % le code de la ressource sous la forme AA

	\begin{tabular}[t]{@{}T@{}}
	% {\bfseries Ancrage et contexte professionnel :} \\
	\csname Rancrage\CODE\endcsname \\
	{\bfseries Contenus :} \\
	%{\setlength{\extrarowheight}{1pt}
	\csname Rcontenu\CODE\endcsname
	%}
	\end{tabular}
}

% Place un champ long dans un tableau => interprète les CR ?
\newcommand*{\tableauChampLong}[1]{
	\begin{tabular}[t]{@{}T@{}}
	#1 \\
	\end{tabular}
}

\newcommand*{\tableauChampCourt}[1]{
    \setlength{\extrarowheight}{0pt}
	\begin{tabular}[t]{@{}P@{}}
	#1 \\
	\end{tabular}
}

% Liste des ressources (dans un tableau), utilisée pour décrire les ressources d'une SAE
\newcommand*{\listeRessources}[1]{%
	\xdef\CODE{#1} % le code de la sae sous la forme AA
		\ifthenelse{\equal{\value{cptSressource\CODE}}{0}}{ % Si pas de ressource
		Aucune
		}{
		\begin{tabular}[t]{@{}T@{}}%
			\forLoop{1}{\value{cptSressource\CODE}}{noligne}{
					\xdef\rescode{\csname Sressourcecode\CODE\Roman{noligne}\endcsname}
				\hyperlink{res:\rescode}{\textcolor{ressourceC}{\rescode}}
						{\csname Sressourcename\CODE\Roman{noligne}\endcsname}
			 \tabularnewline %
			 }
		\end{tabular}%
	}
}%

% Tableau des exemples
\newcommand*{\tableauExemples}[1]{%
	\xdef\CODE{#1} % le code de la sae sous la forme AA

	\begin{tabular}[t]{@{}T@{}}
	% {\bfseries Ancrage et contexte professionnel :} \\
	\forLoop{1}{\value{cptSexemple\CODE}}{noligne}{
		\xdef\exemplecode{\csname EScodeE\CODE\Alph{noligne}\endcsname}
		%\xdef\titreexemple
		%\xdef\contenu{\titreexemple}
    \hyperlink{exemple:\exemplecode}{\textcolor{exempleC}{Exemple \arabic{noligne}}}~: {\csname ESname\exemplecode\endcsname}
    \tabularnewline
	}
	\end{tabular}
}

