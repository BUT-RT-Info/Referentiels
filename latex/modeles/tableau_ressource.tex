

% Le tableau complet présentant une ressource
\newcommand\tableauRessource[1]{

\xdef\CODE{#1} % code de la ressource (par ex: AA avec A=1er semestre, A=1ère ressource)

\setlength{\tabcolsep}{0.125cm} % Marge des colonnes
\setlength{\extrarowheight}{2pt} % Marge des lignes

\arrayrulecolor{ressourceC}

% \cellcolor{exempleC}\hypertarget{exemple:\exemplecode}{\textcolor{compCAp!20!white}
% 1er tableau : Nom/Code/Heures
\begin{tabular}[t]{|P|Q|V|}
\hline % 1ère ligne : code & titre
	\cellcolor{ressourceC} \xdef\rescode{\csname Rcode\CODE\endcsname} % Variable \rescode contenant le code de la ressource pour hyperlink
	\textcolor{compCAp!20!white}{\bfseries \hypertarget{res:\rescode}{Ressource \rescode} / {\csname RcodeRT\CODE\endcsname} }
	& \multicolumn{2}{l|}{\cellcolor{ressourceC} \textcolor{compCAp!20!white}{\bfseries \csname Rname\CODE\endcsname}}
	\\
\hline % 2ème ligne : semestre
	\textcolor{ressourceC}{\bfseries Semestre}
	& \multicolumn{2}{l|}{S{\csname Rsem\CODE\endcsname}} \\
\hline % 3ème ligne : Heures
\hline
\textcolor{ressourceC}{\bfseries Heures}
&
\textcolor{ressourceC}{\bfseries Formation encadrée}
& {\csname Rtraining\CODE\endcsname}h, dont {\csname Rtp\CODE\endcsname}h de TP \\
\hline
\hline % 4ème ligne : Parcours
	\textcolor{ressourceC}{\bfseries Parcours}
	& \multicolumn{2}{l|}{Cyber, DevCloud, PilPro, IOM, ROM} \\
\hline
\end{tabular}

% Description
\begin{tabular}{|G|}
	\hline
	\textcolor{ressourceC}{\bfseries Description} \\
	\hline
	{\csname Rancrage\CODE\endcsname}
	\\
\hline
\end{tabular}

% Contenu
\begin{tabular}{|G|}
	\textcolor{ressourceC}{\bfseries Contenu} \\
	\hline
	\tableauChampLong{\csname Rcontenu\CODE\endcsname}
	\\
\end{tabular}

% 2ème tableau : Compétences et apprentissages critiques
\begin{tabular}[t]{|P|T|}
\hline
\textcolor{ressourceC}{\bfseries \makecell[l]{Compétences / \\
Apprentissages \\ critiques} } & %\listeACs{\CODE}
\tableauChampLong{\listeCompetencesEtACs{\CODE}} \\
\hline
\end{tabular}
% \ifcsdef{Rcoeff\CODE A}{coef. {\csname Rcoeff\CODE A\endcsname}}{}

% SAE
\begin{tabular}[t]{|P|T|}
\hline
\textcolor{ressourceC}{\bfseries SAÉ concernée(s) } & \listeSAE{\CODE} \\
\hline
\end{tabular}

% Pré-requis
\begin{tabular}[t]{|P|T|}
\hline
\textcolor{ressourceC}{\bfseries Prérequis} & \listePrerequis{\CODE} \\
\hline
\end{tabular}

% Mots-clés
\begin{tabular}[t]{|P|T|}
\hline
\textcolor{ressourceC}{\bfseries Mots-clés} & {\csname Rmotscles\CODE\endcsname} \\
\hline

\end{tabular}


} % Fin de la commande