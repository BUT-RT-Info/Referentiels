

% Le tableau complet présentant une ressource
\newcommand\tableauRessource[1]{

\xdef\CODE{#1} % code de la ressource (par ex: AA avec A=1er semestre, A=1ère ressource)


\setlength{\tabcolsep}{0.125cm} % Marge des colonnes
\setlength{\extrarowheight}{2pt} % Marge des lignes

% 1er tableau : Nom/Code/Heures
\begin{tabular}[t]{|P|Q|V|}
\hline % 1ère ligne
	\xdef\rescode{\csname Rcode\CODE\endcsname}
	{\bfseries \hypertarget{res:\rescode}{Ressource \rescode}}
	& \multicolumn{2}{l|}{\bfseries \csname Rname\CODE\endcsname}
	\\
\hline % 2ème ligne
	%{\bfseries Code}
	%& \csname Rcode\CODE\endcsname
	%&
	{\bfseries Semestre}
	& \multicolumn{2}{l|}{\csname Rsem\CODE\endcsname} \\
\hline % 3ème ligne Heures
%{ \setlength{\extrarowheight}{0pt}
%	\begin{tabular}[t]{@{}Q@{}} \bfseries Heures de formation \\ \bfseries encadrées \end{tabular}
%}
\hline
{\bfseries Heures}
&
{\bfseries Formation encadrée}
& {\csname Rtraining\CODE\endcsname}h, dont {\csname Rtp\CODE\endcsname}h de TP \\
%& % {\bfseries dont heures de TP}
%	{\bfseries dont TP}
%& {\csname Rtp\CODE\endcsname}h \\
\hline
\end{tabular}

% 2ème tableau : compétences et apprentissages critiques
\begin{tabular}[t]{|C|C|C|}
\hline
\multicolumn{3}{|c|}{\bfseries Compétence(s) ciblées, coefficient(s) et apprentissage(s) critique(s) couvert(s)} \\
\hline
	\textit{\csname Rcomp\CODE A\endcsname} | {\csname Rniveau\CODE A\endcsname}
	&
	\textit{\csname Rcomp\CODE B\endcsname} | {\csname Rniveau\CODE B\endcsname}
	&
	\textit{\csname Rcomp\CODE C\endcsname} | {\csname Rniveau\CODE C\endcsname}
\\
\hline
	\ifcsdef{Rcoeff\CODE A}{coeff. {\csname Rcoeff\CODE A\endcsname}}{} &
	\ifcsdef{Rcoeff\CODE B}{coeff. {\csname Rcoeff\CODE B\endcsname}}{} &
	\ifcsdef{Rcoeff\CODE C}{coeff. {\csname Rcoeff\CODE C\endcsname}}{}
\\
\hline % AC de RT1
{\listeAC{R}{\CODE}{A}{RT1}
}
& % AC de RT2
{\listeAC{R}{\CODE}{B}{RT2}
}
& % AC de RT3
{\listeAC{R}{\CODE}{C}{RT3}
}
\\
\hline
\end{tabular}

% 3ème tableau : SAE, descriptif
\begin{tabular}[t]{|P|T|}
\hline
{\bfseries SAÉ concernée(s) } & \listeSAE{\CODE} \\
\hline
{\bfseries Prérequis} & \listePrerequis{\CODE} \\
\hline
% {\bfseries Descriptif } & \tableauDescriptif{\CODE} \\
{\bfseries Descriptif } & \tableauChampLong{\csname Rancrage\CODE\endcsname} \\
\hline
{\bfseries Contenus } & \tableauChampLong{\csname Rcontenu\CODE\endcsname}
\\
\hline
{\bfseries Mots-clés} & {\csname Rmotscles\CODE\endcsname} \\
\hline

\end{tabular}



} % Fin de la commande