
% Les formats de colonnes
\newcolumntype{P}{>{}m{3.6cm}<{}}
\newcolumntype{Q}{>{}m{4.6cm}<{}} % 4.1 pour équiréparti
\newcolumntype{C}{>{}p{5.555cm}<{}}
\newcolumntype{T}{>{}p{13.325cm}<{}} % 12.825cm

% Compteur de ligne pour les tableaux
\newcounter{noligne}

% Tableau des apprentissages critiques
\newcommand*{\tableauAC}[2]{%
\begin{tabular}[t]{@{}C@{}}%
	\xdef\CODE{#1} % le code de la ressource sous la forme AA
	\xdef\COMP{#2} % le code de la compétence sous la forme A
	
  \forLoop{1}{\value{cptressourceac\CODE\COMP}}{noligne}{
		\xdef\contenu{{\csname ressourceaccode\CODE\COMP\Alph{noligne}\endcsname} | {\csname ressourceacname\CODE\COMP\Alph{noligne}\endcsname}}
		\contenu			
   \tabularnewline %
   }
\end{tabular}%
}%

% Tableau des SAE
\newcommand*{\tableauSAE}[1]{%
	\xdef\CODE{#1} % le code de la ressource sous la forme AA
		\ifthenelse{\equal{\value{cptressourcesae\CODE}}{0}}{ % Si pas de prérequis
		Aucune
		}{
		\begin{tabular}[t]{@{}T@{}}%
			\forLoop{1}{\value{cptressourcesae\CODE}}{noligne}{
						\xdef\contenu{{\csname ressourcesaecode\CODE\Alph{noligne}\endcsname} | {\csname ressourcesaename\CODE\Alph{noligne}\endcsname}}
				\contenu			
			 \tabularnewline %
			 }
		\end{tabular}%
	}
}%

% Tableau des pré-requis
\newcommand*{\tableauPrerequis}[1]{%
	\xdef\CODE{#1} % le code de la ressource sous la forme AA
	\ifthenelse{\equal{\value{cptressourceprerequis\CODE}}{0}}{ % Si pas de prérequis
Aucun
}{ % Si des pré-requis
\begin{tabular}[t]{@{}T@{}}%

  \forLoop{1}{\value{cptressourceprerequis\CODE}}{noligne}{
				\xdef\contenu{{\csname ressourceprerequiscode\CODE\Alph{noligne}\endcsname} | {\csname ressourceprerequisname\CODE\Alph{noligne}\endcsname}}
		\contenu			
   \tabularnewline %
   }
\end{tabular}%
}
}%

% Tableau du descriptif detaillé
\newcommand*{\tableauDescriptif}[1]{%
	\xdef\CODE{#1} % le code de la ressource sous la forme AA
	
	\begin{tabular}[t]{@{}T@{}} 
	% {\bfseries Ancrage et contexte professionnel :} \\
	\csname ressourceancrage\CODE\endcsname \\
	{\bfseries Contenus :} \\
	\csname ressourcecontenudetaille\CODE\endcsname
	\end{tabular}
}

		%\begin{list}{$\bullet$}{\topsep=0pt \partopsep=0pt \parsep=0pt \itemsep=0pt \leftmargin=1em \itemindent=0em}
			



% Le tableau complet
\newcommand\tableauRessource[1]{

\xdef\CODE{#1} % code de la ressource (par ex: AA avec A=1er semestre, A=1ère ressource)


\setlength{\tabcolsep}{0.125cm} % Marge des colonnes
\setlength{\extrarowheight}{5pt} % Marge des lignes

% 1er tableau : Nom/Code/Heures
\begin{tabular}[t]{|P|Q|P|Q|}
\hline % 1ère ligne
	{\bfseries Ressource} 
	& \multicolumn{3}{l|}{\bfseries \csname ressourcename\CODE\endcsname} \\
\hline % 2ème ligne
	{\bfseries Code} 
	& \csname ressourcecode\CODE\endcsname 
	& {\bfseries Semestre} 
	& \csname ressourcesem\CODE\endcsname \\
\hline % 3ème ligne Heures
{ \setlength{\extrarowheight}{0pt}
	\begin{tabular}[t]{@{}Q@{}} \bfseries Heures de formation \\ \bfseries encadrées \end{tabular}
} 
& {\csname ressourcetraining\CODE\endcsname}h 
& {\bfseries dont heures de TP} 
& {\csname ressourcetp\CODE\endcsname}h \\
\hline
\end{tabular}

% 2ème tableau : compétences et apprentissages critiques
\begin{tabular}[t]{|C|C|C|}
\hline
\multicolumn{3}{|c|}{\bfseries Compétence(s) ciblées et apprentissage(s) critique(s) couvert(s)} \\
\hline
	\textit{\csname ressourcecomp\CODE A\endcsname} | {\csname ressourcecompniveau\CODE A\endcsname} 
	&
	\textit{\csname ressourcecomp\CODE B\endcsname} | {\csname ressourcecompniveau\CODE B\endcsname} 
	&
	\textit{\csname ressourcecomp\CODE C\endcsname} | {\csname ressourcecompniveau\CODE C\endcsname} 
\\
\hline % AC de RT1
{\tableauAC{\CODE}{A}
}
& % AC de RT2
{\tableauAC{\CODE}{B}
}
& % AC de RT3
{\tableauAC{\CODE}{C}
}
\\
\hline
\end{tabular}

% 3ème tableau : SAE, descriptif
\begin{tabular}[t]{|P|T|}
\hline
{\bfseries SAÉ concernée(s) } & \tableauSAE{\CODE} \\
\hline
{\bfseries Prérequis} & \tableauPrerequis{\CODE} \\
\hline
{\bfseries Descriptif détaillé } & \tableauDescriptif{\CODE} \\
\hline
{\bfseries Mots-clés} & {\csname ressourcemotscles\CODE\endcsname} \\
\hline

\end{tabular}



} % Fin de la commande