
% Le tableau complet présentant un exemple d'une SAE donné
\newcommand\tableauExempleSAE[2]{

\xdef\CODE{#1} % code de la sae (par ex: AA avec A=1er semestre, A=1ère ressource)
\xdef\NUMERO{#2} % numéro de l'exemple
\xdef\EXEMPLE{\CODE\NUMERO} % code de l'exemple pour latex

\setlength{\tabcolsep}{0.125cm} % Marge des colonnes
\setlength{\extrarowheight}{2pt} % Marge des lignes

\xdef\exemplecode{\csname EScodeE\EXEMPLE\endcsname}
% 1er tableau : Nom/Code/Semestre

\def\gris{0.9} % La profondeur du gris

% 3ème tableau : descriptif
\begin{tabular}[t]{|P|T|}
\hline
	%\cellcolor[gray]{\gris}
    \xdef\saecode{\csname EScode\EXEMPLE\endcsname}
	{\bfseries Exemple {\csname ESnumero\EXEMPLE\endcsname}} &
		\hyperlink{sae:\saecode}{\saecode}~| {\csname ESsae\EXEMPLE\endcsname} \\
\hline
\hline
	%\cellcolor[gray]{\gris}
	\hypertarget{exemple:\exemplecode}{\bfseries Titre} &
	\tableauChampLong{\bfseries \csname ESname\EXEMPLE\endcsname} \\
\hline
\hline
	{ \setlength{\extrarowheight}{0pt}
	\begin{tabular}[t]{@{}P@{}}
	{\bfseries Problématique} \\
	{\bfseries professionnelle} \\
	{\bfseries posée} \\
	\end{tabular}
} &
	\tableauChampLong{\csname ESproblematique\EXEMPLE\endcsname} \\
\hline
	{\bfseries Description} &
	\tableauChampLong{\csname ESdescription\EXEMPLE\endcsname} \\
\hline
\end{tabular}

\begin{tabular}[t]{|P|T|}
	\hline
{ \setlength{\extrarowheight}{0pt}
	\begin{tabular}[t]{@{}P@{}}
	{\bfseries Formes} \\
	{\bfseries pédagogiques} \\
	\end{tabular}
}
& \tableauChampLong{\csname ESformes\EXEMPLE\endcsname} \\
\hline
{ \setlength{\extrarowheight}{0pt}
	\begin{tabular}[t]{@{}P@{}}
	{\bfseries Modalités} \\
	{\bfseries d'évaluation} \\
	{\bfseries assurant l'acquisition} \\
	{\bfseries du niveau de} \\
	{\bfseries compétence visée} \\
	\end{tabular}
}
& \tableauChampLong{\csname ESmodalite\EXEMPLE\endcsname} \\
\hline
\end{tabular}
}