%%%%%%%%%%%%%%%%%%%%%%%%%%%%%%%%%
% Ressources
%%%%%%%%%%%%%%%%%%%%%%%%%%%%%%%%%

\nouvelleressource{R204}{Initiation à la téléphonie d'entreprise}

\ajoutRheures{30}{15}

%% Les compétences et les ACs
\ajoutRcompetence{RT1-Administrer}{\niveauA}

\ajoutRcoeff{8}

\ajoutRac{AC0113}{Configurer les fonctions de base du réseau local}
\ajoutRac{AC0115}{Identifier les dysfonctionnements du réseau local}
\ajoutRac{AC0116}{Installer un poste client}

\ajoutRcompetence{RT2-Connecter}{\niveauA}

\ajoutRcoeff{4}

\ajoutRac{AC0213}{Déployer des supports de transmission}

\ajoutRcompetence{RT3-Programmer}{\niveauA}

\ajoutRcoeff{0}

\ajoutRac{AC0311}{Utiliser un système informatique et ses outils}
% Les SAE
\ajoutRsae{SAÉ24}{Projet intégratif}

% Les pre-requis
\ajoutRprerequis{R101}{Initiation aux réseaux informatiques}
\ajoutRprerequis{R103}{Réseaux locaux et équipements actifs}

% Le descriptif
\ajoutRancrage{Cette ressource a pour objectif de donner aux étudiants les compétences
de mise en œuvre d'un système téléphonique d'entreprise. Il permettra
d'aborder les différents types de téléphonie (hors téléphonie mobile)
utilisés de nos jours, que ce soit sur un réseau dédié (téléphonie
analogique, numérique) ou un réseau partagé (\textabbrv{ToIP}). Une découverte des
services téléphoniques utiles à la communication dans l'entreprise sera
réalisée, ainsi qu'une première approche des réseaux publics existants.\\
En fonction du contexte local, on pourra orienter l'étude vers un réseau
téléphonique mixte (\textabbrv{TDM}/\textabbrv{IP}) ou \textabbrv{ToIP}.}

% Contenus
\ajoutRcontenudetaille{
\vspace{-5pt}
\begin{itemize}[topsep=5pt]
\item
  Présentation des principes généraux de la téléphonie.
\item
  Numérisation, utilisation de codecs en téléphonie et transport de la
  voix.
\item
  Scénario d'un appel de base.
\item
  Architectures des réseaux publics et privés (commutation,
  signalisation, services, normes de câblage,
  \textabbrv{PoE}).
\item
  Installation d'un système téléphonique d'entreprise
  (insertion/raccordement de postes, connexion au
  réseau de l'opérateur).
\item
  Configuration d'un système téléphonique d'entreprise et de ses
  services associés.
\end{itemize}
On veillera à relier chaque contenu à des problématiques de sécurité
informatique.
}

% Mots-clés
\ajoutRmotscles{\textabbrv{IPBX}, \textabbrv{PABX}, \textabbrv{TDM}/\textabbrv{IP}, postes et services téléphoniques, visiophonie, plan de numérotation, réseaux téléphoniques publics.}
