%%%%%%%%%%%%%%%%%%%%%%%%%%%%%%%%%
% Ressources
%%%%%%%%%%%%%%%%%%%%%%%%%%%%%%%%%

\nouvelleressource{R114}{Mathématiques des transmissions}

\ajoutRheures{30}{6}

%% Les compétences et les ACs
\ajoutRcompetence{RT1-Administrer}{\niveauA}

\ajoutRcoeff{4}

\ajoutRac{AC0111}{Maîtriser les lois fondamentales de l'électricité afin d'intervenir sur des équipements de réseaux et télécommunications}

\ajoutRcompetence{RT2-Connecter}{\niveauA}

\ajoutRcoeff{8}

\ajoutRac{AC0211}{Mesurer et analyser les signaux}
\ajoutRac{AC0212}{Caractériser des systèmes de transmissions élémentaires et découvrir la modélisation mathématique de leur fonctionnement}

\ajoutRcompetence{RT3-Programmer}{\niveauA}




% Les SAE
\ajoutRsae{SAÉ13}{Découvrir un dispositif de transmission}
\ajoutRsae{SAÉ22}{Mesurer et caractériser un signal ou un système}
\ajoutRsae{SAÉ24}{Projet intégratif}

% Les pre-requis
\ajoutRprerequis{R113}{Mathématiques du signal}

% Le descriptif
\ajoutRancrage{Les systèmes de transmission font intervenir des fonctions sinusoïdales
et peuvent être modélisés à l'aide de nombres complexes. De plus,
l'échelle logarithmique est couramment utilisée pour représenter
certains signaux dont la puissance est mesurée en décibels qui
nécessitent la connaissance des fonctions exponentielle et logarithme.
On veillera à montrer l'intérêt des concepts présentés pour modéliser
les systèmes électroniques et on choisira de préférence des exercices en
lien avec l'électronique et les télécommunications.}

% Contenus
\ajoutRcontenudetaille{
\vspace{-5pt}
\begin{itemize}
\item
  Trigonométrie :
  \begin{itemize}
  \item
    formules \(\cos(a\pm b)\), \(\cos(a)\cos(b)\), \(cos^2(a)\) et mêmes
    formules avec sinus;
  \item
    lien avec les vecteurs et le produit scalaire;
  \item
    forme
    \(a\cos(\omega_0 t)+b\sin(\omega_0 t) = A\cos(\omega_0 t +\phi)=A\cos(2\pi f_0 t+\phi)\);
  \item
    fonctions trigonométriques réciproques (en particulier arctangente).
  \end{itemize}
\item
  Fonctions logarithme et exponentielle, puissances :
  \begin{itemize}
  \item
    graphes;
  \item
    propriétés, retour sur les propriétés des puissances;
  \item
    application au dB.
  \end{itemize}
\item
  Nombres complexes :
  \begin{itemize}
  \item
    forme algébrique;
  \item
    addition, multiplication et division avec la forme algébrique
  \item
    forme exponentielle (retour sur les propriétés de l'expo);
  \item
    addition, multiplication et division avec la forme exponentielle;
  \item
    formules d'Euler;
  \item
    interprétation géométrique, lien avec les vecteurs;
  \item
    lien avec la trigonométrie;
  \item
    racines complexes d'un polynôme de degré 2 (à coefficients réels).
  \end{itemize}
\end{itemize}
}

% Mots-clés
\ajoutRmotscles{Trigonométrie, Logarithme, Exponentielle, Complexes.}
