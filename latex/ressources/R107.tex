%%%%%%%%%%%%%%%%%%%%%%%%%%%%%%%%%
% Ressources
%%%%%%%%%%%%%%%%%%%%%%%%%%%%%%%%%
\nouvelleressource{R107}{Fondamentaux de la programmation}
\ajoutheures{41}{30}
%% Les compétences et les ACs
\ajoutcompetence{RT1-Administrer}{\niveauA}

\ajoutcompetence{RT2-Connecter}{\niveauA}

\ajoutcompetence{RT3-Programmer}{\niveauA}
% Les SAE
\ajoutsae{SAÉ15}{Traiter des données}
% Les pre-requis

% Le descriptif
\ajoutancrage{
Elle fournit les bases conceptuelles et pratiques pour concevoir et spécifier formellement un traitement automatisé de l'information. Ces bases pourront venir en appui de nombreuses compétences techniques (en informatique, en réseau, en télécommunication, ...) que le professionnel R\&T doit développer et s’inscrivent dans de nombreuses situations professionnelles que rencontrent le professionnel R\&T, notamment  le développement d’outils informatiques à usage interne d'une équipe (compétence RT3-Programmer) ou l’automatisation du déploiement et de la maintenance des outils logiciels (compétence RT1-Administrer).  
}
% Contenus
\ajoutcontenudetaille{
En utilisant un langage de programmation, comme par exemple Python, les
contenus suivants seront traités : * Notions d'algorithmique :
\begin{verbatim}
* Variables, types de base (nombres, chaînes, listes/tableaux).
* Structures de contrôle : tests, répétitions.
* Fonctions et procédures.
* Portée des variables.
\end{verbatim}
\begin{itemize}
\tightlist
\item
  Tests et corrections d'un programme.
\item
  Prise en main d'un environnement de programmation (éditeur,
  environnement de développement).
\item
  Prise en main de bibliothèques, modules, d'objets existants (appels de
  méthodes), \ldots{}
\item
  Manipulation de fichiers texte.
\item
  Interaction avec le système d'exploitation et la ligne de commande :
  arguments, lancement de commandes.
\item
  Suivi de versions (git, svn, \ldots). L'utilisation de l'anglais est
  préconisée pour la documentation du code.
\end{itemize}
}
% Mots-clés
\ajoutmotscles{Algorithmes, langages de programmation, méthodologie de développement, suivi de versions}
