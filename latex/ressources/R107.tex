%%%%%%%%%%%%%%%%%%%%%%%%%%%%%%%%%
% Ressources
%%%%%%%%%%%%%%%%%%%%%%%%%%%%%%%%%

\nouvelleressource{R107}{Fondamentaux de la programmation}

\ajoutRheures{41}{30}

%% Les compétences et les ACs
\ajoutRcompetence{RT1-Administrer}{\niveauA}



\ajoutRcompetence{RT2-Connecter}{\niveauA}



\ajoutRcompetence{RT3-Programmer}{\niveauA}

\ajoutRac{AC0311}{Utiliser un système informatique et ses outils}
\ajoutRac{AC0312}{Lire, exécuter, corriger et modifier un programme}
\ajoutRac{AC0313}{Traduire un algorithme, dans un langage et pour un environnement donné}
\ajoutRac{AC0316}{S'intégrer dans un environnement propice au développement et au travail collaboratif}
% Les SAE
\ajoutRsae{SAÉ15}{Traiter des données}

% Les pre-requis


% Le descriptif
\ajoutRancrage{Elle fournit les bases conceptuelles et pratiques pour concevoir et
spécifier formellement un traitement automatisé de l'information. Ces
bases pourront venir en appui de nombreuses compétences techniques (en
informatique, en réseau, en télécommunication, \ldots) que le
professionnel R\&T doit développer et s'inscrivent dans de nombreuses
situations professionnelles que rencontrent le professionnel R\&T,
notamment le développement d'outils informatiques à usage interne d'une
équipe (compétence RT3-Programmer) ou l'automatisation du déploiement et
de la maintenance des outils logiciels (compétence RT1-Administrer).}

% Contenus
\ajoutRcontenudetaille{
En utilisant un langage de programmation, comme par exemple Python, les
contenus suivants seront traités :
\begin{itemize}
\item
  Notions d'algorithmique :
  \begin{itemize}
    \item
    Variables, types de base (nombres, chaînes, listes/tableaux).
  \item
    Structures de contrôle : tests, répétitions.
  \item
    Fonctions et procédures.
  \item
    Portée des variables.
  \end{itemize}
\item
  Tests et corrections d'un programme.
\item
  Prise en main d'un environnement de programmation (éditeur,
  environnement de développement).
\item
  Prise en main de bibliothèques, modules, d'objets existants (appels de
  méthodes), \ldots{}
\item
  Manipulation de fichiers texte.
\item
  Interaction avec le système d'exploitation et la ligne de commande :
  arguments, lancement de commandes.
\item
  Suivi de versions (git, svn, \ldots).
\end{itemize}
L'utilisation de l'anglais est préconisée pour la documentation du code.
}

% Mots-clés
\ajoutRmotscles{Algorithmes, langages de programmation, méthodologie de développement, suivi de versions}
