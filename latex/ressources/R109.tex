%%%%%%%%%%%%%%%%%%%%%%%%%%%%%%%%%
% Ressources
%%%%%%%%%%%%%%%%%%%%%%%%%%%%%%%%%

\nouvelleressource{R109}{Introduction aux technologies Web}

\ajoutRheures{9}{5}

%% Les compétences et les ACs
\ajoutRcompetence{RT1-Administrer}{\niveauA}





\ajoutRcompetence{RT2-Connecter}{\niveauA}





\ajoutRcompetence{RT3-Programmer}{\niveauA}

\ajoutRcoeff{4}

\ajoutRac{AC0314}{Connaître l'architecture et les technologies d'un site Web}
% Les SAE
\ajoutRsae{SAÉ14}{Se présenter sur Internet}

% Les pre-requis


% Le descriptif
\ajoutRancrage{Le professionnel R\&T peut être amené à modifier et à produire des
contenus Web pour le site Web et l'intranet d'une entreprise. Grâce aux
pages Web, il peut aisément mettre à disposition des collaborateurs les
outils-métiers qu'il aura développés (compétence RT3) et leurs
documentations. Plus généralement, il pourra même développer une
application Web.\\
La présente ressource fournit les bases conceptuelles et pratiques pour
écrire et modifier des pages Web dans un langage normalisé de
description de contenus et de sa présentation. Elle traite donc de la
création de contenus Web (un thème abordé par PIX,
\url{https://pix.fr/competences}) mais également des technologies mises
en œuvre pour délivrer ses contenus aux utilisateurs par le biais d'un
navigateur Web.}

% Contenus
\ajoutRcontenudetaille{
\vspace{-5pt}
\begin{itemize}[topsep=5pt]
\item
  Utilisation avancée d'un navigateur Web
\item
  Structure d'un site Web~: client-serveur, arborescence, \textabbrv{URL}
\item
  Structure d'une page~: langage à balise, mise en forme et feuilles de
  styles (notions élémentaires
  de \textabbrv{CSS}), notions de responsive design
\item
  Contenu d'une page~: éléments multimédia, encodage des caractères
\item
  Sensibilisation aux mentions obligatoires d'un site Web (mentions
  légales, copyright, \ldots)
\end{itemize}
}

% Mots-clés
\ajoutRmotscles{Web, \textabbrv{HTML}, \textabbrv{CSS}, Client/serveur, Codage de l'information.}
