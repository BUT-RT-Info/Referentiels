%%%%%%%%%%%%%%%%%%%%%%%%%%%%%%%%%
% Ressources
%%%%%%%%%%%%%%%%%%%%%%%%%%%%%%%%%

\nouvelleressource{R214}{Analyse mathématique des signaux}

\ajoutRheures{30}{6}

%% Les compétences et les ACs
\ajoutRcompetence{RT1-Administrer}{\niveauA}

\ajoutRac{AC0111}{Maîtriser les lois fondamentales de l'électricité afin d'intervenir sur des équipements de réseaux et télécommunications}

\ajoutRcompetence{RT2-Connecter}{\niveauA}

\ajoutRac{AC0211}{Mesurer et analyser les signaux}
\ajoutRac{AC0212}{Caractériser des systèmes de transmissions élémentaires et découvrir la modélisation mathématique de leur fonctionnement}

\ajoutRcompetence{RT3-Programmer}{\niveauA}


% Les SAE
\ajoutRsae{SAÉ22}{Mesurer et caractériser un signal ou un système}
\ajoutRsae{SAÉ24}{Projet intégratif de S2}

% Les pre-requis
\ajoutRprerequis{R113}{Mathématiques du signal}
\ajoutRprerequis{R114}{Mathématiques des transmissions}

% Le descriptif
\ajoutRancrage{L'étude des signaux de transmission nécessite l'usage d'outils
mathématiques de base, en particulier la dérivation, l'intégration
(calcul de puissance, de valeur efficace, de valeur moyenne). Les
systèmes sont souvent étudiés en régime linéaire, d'où l'intérêt de
définir des équivalents. On veillera à montrer l'intérêt des concepts
présentés pour modéliser les systèmes électroniques et on choisira de
préférence des exercices en lien avec l'électronique et les
télécommunications.}

% Contenus
\ajoutRcontenudetaille{
\begin{itemize}
\item
  Dérivée
  \begin{itemize}
    \item
    définition
  \item
    notation s'(t)=ds/dt
  \item
    équation de la tangente
  \item
    dérivée des fonctions usuelles
  \item
    opérations sur les dérivées (somme, produit, quotient, composition)
  \item
    sens de variation
  \item
    application à la recherche d'optimum local
  \end{itemize}
\item
  Comportement local et asymptotique
  \begin{itemize}
    \item
    limites (opérations, formes indéterminées)
  \item
    fonctions négligeables, équivalents
  \end{itemize}
\item
  Intégration
  \begin{itemize}
    \item
    définition d'une intégrale comme une surface
  \item
    primitive
  \item
    calcul d'une intégrale à l'aide d'une primitive
  \item
    intégration par parties et changement de variable
  \end{itemize}
\end{itemize}
}

% Mots-clés
\ajoutRmotscles{Dérivées, Intégrales, limites}
