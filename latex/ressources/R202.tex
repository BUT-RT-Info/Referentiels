%%%%%%%%%%%%%%%%%%%%%%%%%%%%%%%%%
% Ressources
%%%%%%%%%%%%%%%%%%%%%%%%%%%%%%%%%

\nouvelleressource{R202}{Administration système et fondamentaux de la virtualisation}

\ajoutRheures{30}{20}

%% Les compétences et les ACs
\ajoutRcompetence{RT1-Administrer}{\niveauA}

\ajoutRcoeff{12}

\ajoutRac{AC0113}{Configurer les fonctions de base du réseau local}
\ajoutRac{AC0114}{Maîtriser les rôles et les principes fondamentaux des systèmes d'exploitation afin d'interagir avec ceux-ci pour la configuration et administration des réseaux et services fournis}
\ajoutRac{AC0115}{Identifier les dysfonctionnements du réseau local}
\ajoutRac{AC0116}{Installer un poste client}

\ajoutRcompetence{RT2-Connecter}{\niveauA}





\ajoutRcompetence{RT3-Programmer}{\niveauA}

\ajoutRcoeff{3}

\ajoutRac{AC0311}{Utiliser un système informatique et ses outils}
% Les SAE
\ajoutRsae{SAÉ21}{Construire un réseau informatique pour une petite structure}
\ajoutRsae{SAÉ24}{Projet intégratif}

% Les pre-requis
\ajoutRprerequis{R101}{Initiation aux réseaux informatiques}
\ajoutRprerequis{R107}{Fondamentaux de la programmation}
\ajoutRprerequis{R108}{Bases des systèmes d'exploitation}

% Le descriptif
\ajoutRancrage{Cette ressource apporte le socle de connaissances et savoir-faire pour
les compétences de cœur de métier \og Administrer les réseaux et
l'Internet\fg{} (RT1). Elle donne aux étudiants les compétences pour
effectuer des tâches simples d'administration du système d'information
de l'entreprise (processus, utilisateurs, automatisation) et pour
utiliser des solutions de virtualisation, de conteneurisation.\\
Elle contribue aussi à la compétence \og Créer des outils et
applications informatiques pour les R\&T\fg{} (RT3) à travers la
découverte du poste client et de son environnement logiciel.}

% Contenus
\ajoutRcontenudetaille{
\vspace{-5pt}
\begin{itemize}
\item
  Gestion des systèmes de fichiers (volumes, droits, types de fichiers)
\item
  Gestion de processus et services
\item
  Gestion de ressources utilisateurs (comptes, quotas)
\item
  Scripts pour l'automatisation de séquences de commandes
\item
  Utilisation de fichiers de traces (logs)
\item
  Initiation et mise en oeuvre d'infrastructures de virtualisation et/ou
  de conteneurisation
\end{itemize}
}

% Mots-clés
\ajoutRmotscles{Systèmes d'exploitation, Linux, Windows, Scripts, Virtualisation, Conteneurisation, Cybersécurité.}
