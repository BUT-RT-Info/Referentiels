%%%%%%%%%%%%%%%%%%%%%%%%%%%%%%%%%
% Ressources
%%%%%%%%%%%%%%%%%%%%%%%%%%%%%%%%%

\nouvelleressource{R213}{Mathématiques des systèmes numériques}

\ajoutRheures{30}{6}

%% Les compétences et les ACs
\ajoutRcompetence{RT1-Administrer}{\niveauA}

\ajoutRcoeff{3}

\ajoutRac{AC0112}{Comprendre l'architecture des systèmes numériques et les principes du codage de l'information}

\ajoutRcompetence{RT2-Connecter}{\niveauA}

\ajoutRcoeff{5}

\ajoutRac{AC0212}{Caractériser des systèmes de transmissions élémentaires et découvrir la modélisation mathématique de leur fonctionnement}

\ajoutRcompetence{RT3-Programmer}{\niveauA}

\ajoutRcoeff{5}

\ajoutRac{AC0311}{Utiliser un système informatique et ses outils}
\ajoutRac{AC0313}{Traduire un algorithme, dans un langage et pour un environnement donné}
% Les SAE
\ajoutRsae{SAÉ22}{Mesurer et caractériser un signal ou un système}
\ajoutRsae{SAÉ23}{Mettre en place une solution informatique pour l'entreprise}
\ajoutRsae{SAÉ24}{Projet intégratif}

% Les pre-requis


% Le descriptif
\ajoutRancrage{Les systèmes numériques font intervenir des signaux discrets, qui
peuvent être modélisés sous la forme de vecteurs ou de matrices. Par
ailleurs, certains algorithmes sont itératifs d'où l'importance de la
notion de récurrence. On veillera à illustrer les concepts présentés par
l'exploitation d'algorithmes mis en œuvre via un outil informatique.}

% Contenus
\ajoutRcontenudetaille{
\vspace{-10pt}
\begin{itemize}[topsep=5pt]
\item
  Suites, récurrence, signal numérique
  \begin{itemize}
  \item
    raisonnement par récurrence\,;
  \item
    suites récurrentes\,;
  \item
    signal discret (exemples~: Kronecker, échelon échantillonné)\,;
  \item
    convergence d'une suite (opérations sur les limites).
  \end{itemize}
\item
  Vecteurs en \textabbrv{2D} et \textabbrv{3D}
  \begin{itemize}
  \item
    définitions\,;
  \item
    opérations (addition et multiplication externe)\,;
  \item
    produit scalaire (lien avec la trigonométrie)\,;
  \item
    application au calcul d'une équation de droite.
  \end{itemize}
\item
  Matrices et vecteurs
  \begin{itemize}
  \item
    définitions\,;
  \item
    opérations\,;
  \item
    résolutions de systèmes linéaires (pivot de Gauss).
  \end{itemize}
\end{itemize}
}

% Mots-clés
\ajoutRmotscles{Mathématiques, Suites, Ensembles, Vecteurs, Matrices.}
