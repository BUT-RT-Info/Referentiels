%%%%%%%%%%%%%%%%%%%%%%%%%%%%%%%%%
% Ressources
%%%%%%%%%%%%%%%%%%%%%%%%%%%%%%%%%

\nouvelleressource{R113}{Mathématiques du signal}

\ajoutRheures{30}{6}

%% Les compétences et les ACs
\ajoutRcompetence{RT1-Administrer}{\niveauA}

\ajoutRcoeff{5}

\ajoutRac{AC0111}{Maîtriser les lois fondamentales de l'électricité afin d'intervenir sur des équipements de réseaux et télécommunications}

\ajoutRcompetence{RT2-Connecter}{\niveauA}

\ajoutRcoeff{8}

\ajoutRac{AC0211}{Mesurer et analyser les signaux}
\ajoutRac{AC0212}{Caractériser des systèmes de transmissions élémentaires et découvrir la modélisation mathématique de leur fonctionnement}

\ajoutRcompetence{RT3-Programmer}{\niveauA}




% Les SAE
\ajoutRsae{SAÉ13}{Découvrir un dispositif de transmission}
\ajoutRsae{SAÉ22}{Mesurer et caractériser un signal ou un système}
\ajoutRsae{SAÉ24}{Projet intégratif}

% Les pre-requis


% Le descriptif
\ajoutRancrage{Les systèmes de transmission font intervenir des fonctions sinusoïdales
ainsi que des signaux de base (périodiques ou non) soumis à des
transformations (retard, dilatation, amplification, offset) qui sont
explicitées par cette ressource. On veillera à montrer l'intérêt des
concepts présentés pour modéliser les systèmes électroniques et on
choisira de préférence des exercices en lien avec l'électronique et les
télécommunications.}

% Contenus
\ajoutRcontenudetaille{
\vspace{-5pt}
\begin{itemize}
\item
  Introduction aux signaux
  \begin{itemize}
  \item
    graphe d'un signal
  \item
    symétries : parité, imparité, \ldots{}
  \item
    causalité, support temporel
  \item
    équation de droite
  \item
    fonction définie par morceaux (ex. : valeur absolue)
  \item
    signaux de base (ex. : porte, triangle, échelon, rampe\ldots)
  \item
    opérations sur les signaux : avance, retard, dilatation,
    amplification, offset (interprétation géométrique
    sur le graphe), somme de signaux
  \end{itemize}
\end{itemize}
Cette partie sera l'occasion de réviser les règles de calculs de base
par l'intermédiaire du calcul d'images et d'antécédents.
\begin{itemize}
\item
  Éléments de trigonométrie
  \begin{itemize}
  \item
    définition du radian
  \item
    cercle trigonométrique
  \item
    formules \(\cos(-x)\), \(\cos(\pi\pm x)\) et
    \(\cos(\frac{\pi}{2}\pm x)\) ; les même avec sinus
  \item
    angles remarquables
  \end{itemize}
\item
  Signaux périodiques
  \begin{itemize}
  \item
    période, fréquence, pulsation
  \item
    signaux périodiques de base : créneau, dent de scie, sinus,
    cosinus\ldots{}
  \item
    fréquence/période/pulsation d'un signal dilaté, d'une combinaison
    linéaire de signaux périodiques
  \item
    graphe des signaux avancés, retardés, dilatés\ldots{}
  \item
    graphe de \(A\cos(\omega t + \phi)\), \(A\cos(\omega t + \phi)\)
  \end{itemize}
\end{itemize}
}

% Mots-clés
\ajoutRmotscles{Signaux, signaux périodiques}
