%%%%%%%%%%%%%%%%%%%%%%%%%%%%%%%%%
% Ressources
%%%%%%%%%%%%%%%%%%%%%%%%%%%%%%%%%

\nouvelleressource{R211}{Expression-Culture-Communication Professionnelles 2}

\ajoutRheures{30}{21}

%% Les compétences et les ACs
\ajoutRcompetence{RT1-Administrer}{\niveauA}

\ajoutRcoeff{3}

\ajoutRac{AC0115}{Identifier les dysfonctionnements du réseau local}

\ajoutRcompetence{RT2-Connecter}{\niveauA}

\ajoutRcoeff{5}

\ajoutRac{AC0215}{Communiquer avec un client ou un collaborateur}

\ajoutRcompetence{RT3-Programmer}{\niveauA}

\ajoutRcoeff{4}

\ajoutRac{AC0316}{S'intégrer dans un environnement propice au développement et au travail collaboratif}
% Les SAE
\ajoutRsae{SAÉ21}{Construire un réseau informatique pour une petite structure}
\ajoutRsae{SAÉ22}{Mesurer et caractériser un signal ou un système}
\ajoutRsae{SAÉ23}{Mettre en place une solution informatique pour l’entreprise}
\ajoutRsae{SAÉ24}{Projet intégratif}

% Les pre-requis
\ajoutRprerequis{R111}{Expression-Culture-Communication Professionnelles 1}

% Le descriptif
\ajoutRancrage{}

% Contenus
\ajoutRcontenudetaille{
\vspace{-5pt}
\begin{itemize}
\item
  Utiliser les outils et ressources documentaires de manière
  professionnelle
\item
  Analyser et restituer des informations de façon synthétique
\item
  S'initier au résumé
\item
  Produire des écrits longs et clairs, structurés, adaptés au
  destinataire et répondant aux normes de
  présentation professionnelle et académique (dossier, présentation
  longue, exploitation de la mise en
  forme pour alléger les contenus et guider la lecture\ldots)
\item
  Réécrire et corriger ses documents ;
\item
  Exploiter efficacement des outils de traitement de texte et de partage
  des données
\item
  Renforcer les compétences linguistiques
\item
  Élaborer un discours clair et efficace dans différents contextes
\item
  Adapter sa communication verbale et non-verbale
\item
  Comprendre une situation de communication complexe
\item
  Savoir utiliser à bon escient des outils multimédia pour une
  présentation orale
\item
  Décrire et analyser l'image fixe et mobile
\item
  Produire un document audiovisuel court
\item
  Adopter des savoir-être professionnels essentiels dans le travail en
  équipe (coopération, prise en
  compte de l'opinion d'autrui, adaptation, prise d'initiative\ldots)
\item
  S'initier à la gestion de projet : argumenter, défendre son point de
  vue
\item
  Agir en cohérence avec les objectifs du développement durable
\item
  Comprendre et s'approprier les enjeux du monde contemporain
\end{itemize}
Création de supports vidéo (film, tutoriel, notice) - outils de veille
documentaire - critique des médias sociaux - participation à des actions
culturelles - résumé - synthèse d'un document - débat - revue de presse
}

% Mots-clés
\ajoutRmotscles{Synthèse, résumé, expression écrite, rédaction technique, expression orale, médias, culture générale, esprit critique, développement durable}
