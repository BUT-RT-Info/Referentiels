%%%%%%%%%%%%%%%%%%%%%%%%%%%%%%%%%
% Ressources
%%%%%%%%%%%%%%%%%%%%%%%%%%%%%%%%%

\nouvelleressource{R208}{Analyse et traitement de données structurées}

\ajoutRheures{16}{10}

%% Les compétences et les ACs
\ajoutRcompetence{RT1-Administrer}{\niveauA}





\ajoutRcompetence{RT2-Connecter}{\niveauA}





\ajoutRcompetence{RT3-Programmer}{\niveauA}

\ajoutRcoeff{10}

\ajoutRac{AC0311}{Utiliser un système informatique et ses outils}
\ajoutRac{AC0312}{Lire, exécuter, corriger et modifier un programme}
\ajoutRac{AC0313}{Traduire un algorithme, dans un langage et pour un environnement donné}
\ajoutRac{AC0315}{Choisir les mécanismes de gestion de données adaptés au développement de l'outil}
\ajoutRac{AC0316}{S'intégrer dans un environnement propice au développement et au travail collaboratif}
% Les SAE
\ajoutRsae{SAÉ23}{Mettre en place une solution informatique pour l'entreprise}
\ajoutRsae{SAÉ24}{Projet intégratif}

% Les pre-requis
\ajoutRprerequis{R107}{Fondamentaux de la programmation}

% Le descriptif
\ajoutRancrage{Le professionnel R\&T est amené à développer différents outils
informatiques à usage personnel ou interne à l'équipe (compétence
RT3-Programmer). Ces outils peuvent traiter des données complexes, viser
des fonctionnalités multiples et être développé en équipe~: il est alors
nécessaire - pour le professionnel R\&T - de structurer son travail,
tant sur les variables manipulant les données, les fichiers qui les
sauvegardent ou les restaurent, que sur l'organisation (arborescence) de
son projet. La ressource introduit ses éléments structurels en
contribuant à l'acquisition des apprentissages critiques mentionnés
précédemment.}

% Contenus
\ajoutRcontenudetaille{
\vspace{-10pt}
\begin{itemize}[topsep=5pt]
\item
  Structure d'un programme~: arborescence de fichiers, modules et
  packages.
\item
  Contexte d'exécution~: programme principal vs script.
\item
  Structure complexe de données~:
  \begin{itemize}
  \item
    Listes \textabbrv{2D}, tableaux associatifs/dictionnaires\,;
  \item
    Notion de classes (instance, attributs, méthodes).
  \end{itemize}
\item
  Manipulation de fichiers avancée~:
  \begin{itemize}
  \item
    Fichiers structurés (\textabbrv{XML}, \textabbrv{CSV}, \textabbrv{JSON}, YAML)\,;
  \item
    Gestion de l'arborescence par le code\,;
  \item
    Lecture/écriture de fichiers structurés\,;
  \item
    Notion de sérialisation\,;
  \item
    Notion de persistance des données.
  \end{itemize}
\item
  Initiation aux expressions régulières.
\item
  Introduction au traitement des erreurs.
\end{itemize}
L'utilisation de l'anglais est préconisée dans la documentation du code.
}

% Mots-clés
\ajoutRmotscles{Algorithmes, Langages informatiques, Programmation, Structure de données, Méthodologie de développement, gestion de versions.}
