%%%%%%%%%%%%%%%%%%%%%%%%%%%%%%%%%
% Ressources
%%%%%%%%%%%%%%%%%%%%%%%%%%%%%%%%%

\nouvelleressource{R108}{Bases des systèmes d'exploitation}

\ajoutRheures{27}{21}

%% Les compétences et les ACs
\ajoutRcompetence{RT1-Administrer}{\niveauA}

\ajoutRac{AC0114}{Maîtriser les rôles et les principes fondamentaux des systèmes d'exploitation afin d'interagir avec ceux-ci pour la configuration et administration des réseaux et services fournis}
\ajoutRac{AC0116}{Installer un poste client}

\ajoutRcompetence{RT2-Connecter}{\niveauA}



\ajoutRcompetence{RT3-Programmer}{\niveauA}

\ajoutRac{AC0311}{Utiliser un système informatique et ses outils}
\ajoutRac{AC0312}{Lire, exécuter, corriger et modifier un programme}
% Les SAE
\ajoutRsae{SAÉ15}{Traiter des données}

% Les pre-requis


% Le descriptif
\ajoutRancrage{Cette ressource traite des bases de l'utilisation d'un poste client et
de son système d'exploitation. Elle est essentielle pour la prise en
main pratique d'un système informatique en abordant notamment la gestion
des données dans un espace de stockage (organisation, recherche, droits)
et la maîtrise d'un environnement numérique, deux thèmes attendus par le
référentiel PIX (https://pix.fr/competences).
Cette ressource introduit également un usage avancé du système
d'exploitation nécessaire au besoin d'un professionnel R\&T. Elle vise
la maîtrise de commandes en ligne pour gérer l'arborescence de fichiers,
les programmes et les processus du système d'exploitation, par exemple
pour exécuter un programme ou configurer les éléments d'un site Web
(compétence RT3-Programmer). Elle vise également l'emploi des
principales commandes réseau, dans des scripts simples. Ces commandes
sont les bases d'appui pour administrer - par la suite - un réseau et de
ses services (compétence RT1-Administrer).
Elle contribue donc aux apprentissages critiques mentionnés
précédemment.}

% Contenus
\ajoutRcontenudetaille{
\begin{itemize}
\item
  Systèmes d'exploitations Windows/Linux, Interface-Homme-Machine et
  ligne de commande
\item
  Arborescence des répertoires, déplacement, consultation, chemins
\item
  Manipulation de fichiers avec un éditeur texte
\item
  Permissions, droits
\item
  Gestion des processus et flux (redirection, pipe\ldots)
\item
  Se documenter sur le détail des commandes en français/anglais
  (commande man)
\item
  Consulter et modifier les variables d'environnement
\item
  Commandes réseau (wget, curl, ping, traceroute, netstat, nmap\ldots)
\item
  Initiation aux scripts pour l'automatisation de séquences de
  commandes, aux structures de contrôle
\end{itemize}
}

% Mots-clés
\ajoutRmotscles{Programmation, arborescence, processus, scripts, variables d'environnement, PIX}
