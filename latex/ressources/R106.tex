%%%%%%%%%%%%%%%%%%%%%%%%%%%%%%%%%
% Ressources
%%%%%%%%%%%%%%%%%%%%%%%%%%%%%%%%%

\nouvelleressource{R106}{Architecture des systèmes numériques et informatiques}

\ajoutRheures{24}{12}

%% Les compétences et les ACs
\ajoutRcompetence{RT1-Administrer}{\niveauA}

\ajoutRcoeff{10}

\ajoutRac{AC0112}{Comprendre l'architecture des systèmes numériques et les principes du codage de l'information}

\ajoutRcompetence{RT2-Connecter}{\niveauA}





\ajoutRcompetence{RT3-Programmer}{\niveauA}



\ajoutRac{AC0311}{Utiliser un système informatique et ses outils}
% Les SAE
\ajoutRsae{SAÉ12}{S'initier aux réseaux informatiques}
\ajoutRsae{SAÉ15}{Traiter des données}

% Les pre-requis


% Le descriptif
\ajoutRancrage{Cette ressource apporte le socle de connaissances et savoir-faire pour
les compétences de cœur de métier \og Administrer les réseaux et
l'Internet\fg{} (RT1) et \og  Créer des outils et applications
informatiques pour les R\&T\fg{} (RT3).\\
Les systèmes informatiques et numériques sont au cœur de la spécialité
Réseaux et Télécoms. Cette ressource vise tout d'abord à permettre la
compréhension du codage et du stockage des données. Puis elle permet de
comprendre de façon très fine le comportement interne des systèmes
numériques avec notamment des notions de temps d'exécution. Enfin elle
permettra aux étudiants de relier ces systèmes au monde extérieur.}

% Contenus
\ajoutRcontenudetaille{
\vspace{-5pt}
\begin{itemize}[topsep=5pt]
\item
  Codage des nombres, des caractères, des images.
\item
  Fonctions logiques - Logique combinatoire et séquentielle - Notion
  d'\textabbrv{ALU}.
\item
  Structure d'un processeur - Différents types de mémoires.
\item
  Périphériques et entrées-sorties. Exemples \textabbrv{GPIO}, liaison série.
\end{itemize}
}

% Mots-clés
\ajoutRmotscles{Nombres binaires, Codage, Fonctions logiques, Processeur, \textabbrv{ALU}.}
