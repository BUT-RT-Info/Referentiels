%%%%%%%%%%%%%%%%%%%%%%%%%%%%%%%%%
% Ressources
%%%%%%%%%%%%%%%%%%%%%%%%%%%%%%%%%

\nouvelleressource{R111}{Expression-Culture-Communication Professionnelles 1}

\ajoutRheures{30}{21}

%% Les compétences et les ACs
\ajoutRcompetence{RT1-Administrer}{\niveauA}

\ajoutRcoeff{4}

\ajoutRac{AC0116}{Installer un poste client}

\ajoutRcompetence{RT2-Connecter}{\niveauA}

\ajoutRcoeff{4}

\ajoutRac{AC0215}{Communiquer avec un client ou un collaborateur}

\ajoutRcompetence{RT3-Programmer}{\niveauA}

\ajoutRcoeff{4}

\ajoutRac{AC0316}{S'intégrer dans un environnement propice au développement et au travail collaboratif}
% Les SAE
\ajoutRsae{SAÉ11}{Se sensibiliser à l'hygiène informatique et à la cybersécurité}
\ajoutRsae{SAÉ12}{S'initier aux réseaux informatiques}
\ajoutRsae{SAÉ13}{Découvrir un dispositif de transmission}
\ajoutRsae{SAÉ14}{Se présenter sur Internet}
\ajoutRsae{SAÉ15}{Traiter des données}
\ajoutRsae{SAÉ16}{Portfolio}

% Les pre-requis


% Le descriptif
\ajoutRancrage{A travers différentes activités (ateliers d'écriture, exposés,
dialogues, constitution de dossiers, etc.), les étudiants apprendront à
communiquer de manière claire et professionnelle, en utilisant à bon
escient les techniques et outils à leur disposition, que ce soit pour la
communication écrite ou orale ou interpersonnelle. L'enseignement
s'appuiera sur des exemples de situations professionnelles typiques du
domaine réseaux et télécommunications. Au-delà de la communication
proprement dite, il s'agira aussi de sensibiliser les étudiants à
l'importance des savoir-être et aux enjeux du développement durable.}

% Contenus
\ajoutRcontenudetaille{
\vspace{-10pt}
\begin{itemize}[topsep=5pt]
\item
  Rechercher, sélectionner ses sources et questionner leur fiabilité
\item
  Analyser et restituer des informations
\item
  Produire des écrits courts, clairs, structurés, adaptés et répondant
  aux normes de présentation professionnelle
  et académique (mail, argumentation courte\ldots)
\item
  Réécrire et corriger ses documents
\item
  Découvrir des outils de traitement de texte et de partage des données
\item
  Renforcer les compétences linguistiques selon différents canaux
\item
  Élaborer un discours clair et efficace dans un contexte simple
\item
  Être attentif à ses manières de communiquer (dimensions verbale et
  non-verbale)
\item
  Comprendre une situation de communication simple
\item
  Savoir utiliser des outils multimédia pour une présentation orale
\item
  Décrire et analyser l'image fixe et mobile
\item
  Adopter des savoir-être professionnels essentiels dans le travail en
  équipe (écoute, reformulation,
  transmission des informations, explications\ldots)
\item
  S'initier aux objectifs du développement durable
\item
  Être sensible aux enjeux du monde contemporain
\end{itemize}
}

% Mots-clés
\ajoutRmotscles{Recherche documentaire, Expression écrite, Rédaction technique, Expression orale, Médias, Culture générale, Esprit critique, Développement durable.}
