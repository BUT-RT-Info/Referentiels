%%%%%%%%%%%%%%%%%%%%%%%%%%%%%%%%%
% Ressources
%%%%%%%%%%%%%%%%%%%%%%%%%%%%%%%%%

\nouvelleressource{R101}{Initiation aux réseaux informatiques}

\ajoutRheures{46}{24}

%% Les compétences et les ACs
\ajoutRcompetence{RT1-Administrer}{\niveauA}

\ajoutRcoeff{12}

\ajoutRac{AC0113}{Configurer les fonctions de base du réseau local}
\ajoutRac{AC0115}{Identifier les dysfonctionnements du réseau local}
\ajoutRac{AC0116}{Installer un poste client}

\ajoutRcompetence{RT2-Connecter}{\niveauA}

\ajoutRcoeff{4}

\ajoutRac{AC0213}{Déployer des supports de transmission}

\ajoutRcompetence{RT3-Programmer}{\niveauA}

\ajoutRcoeff{4}

\ajoutRac{AC0311}{Utiliser un système informatique et ses outils}
% Les SAE
\ajoutRsae{SAÉ11}{Se sensibiliser à l'hygiène informatique et à la cybersécurité}
\ajoutRsae{SAÉ12}{S'initier aux réseaux informatiques}

% Les pre-requis


% Le descriptif
\ajoutRancrage{Cette ressource apporte le socle de connaissances et savoirs-faire pour
les compétences de cœur de métier \og Administrer les réseaux et
l'Internet\fg{} (RT1) et \og Connecter les entreprises et les
usagers\fg{} (RT2). Elle contribue aussi à la compétence \og Créer des
outils et applications informatiques pour les R\&T\fg{} (RT3) à travers
la découverte du poste client et de son environnement logiciel.\\
Cette ressource permet à l'étudiant de découvrir et déployer un premier
système d'information au sein d'une entreprise. À travers des exercices
de mise en place progressive de réseaux locaux, d'interconnection
d'équipements et de prise en main des fonctions de base des systèmes
d'exploitation, l'étudiant découvrira les principaux concepts utilisés
dans les réseaux informatiques, et commencera à comprendre le rôle et
les principes des normes et protocoles essentiels des réseaux locaux,
comme Ethernet, \textabbrv{TCP}/\textabbrv{IP}, \textabbrv{DHCP}, \textabbrv{DNS}.\\
On introduira des notions de sécurité informatique (les ressources
associées aux recommandations de l'\textabbrv{ANSSI}, CyberEdu, CyberMalveillance
pourront servir de support).}

% Contenus
\ajoutRcontenudetaille{
\vspace{-5pt}
\begin{itemize}
\item
  Initiation au réseau
  \begin{itemize}
  \item
    Découverte et prise en main du réseau local
  \item
    Adressage \textabbrv{IP}v4~: classes d'adresses, masques naturels, adressage
    statique, adressage dynamique (\textabbrv{DHCP})
  \item
    Notion de routage, de passerelle et de serveur \textabbrv{DNS}
  \end{itemize}
\item
  Bases du système d'exploitation
  \begin{itemize}
  \item
    Architecture d'un système d'exploitation
  \item
    Différents types de systèmes d'exploitation~: les clients, les
    serveurs, les systèmes embarqués
  \item
    Systèmes d'exploitation commerciaux et Open Sources.
  \item
    Administration des systèmes d'exploitation
  \item
    Architectures réseaux et systèmes d'exploitation
  \end{itemize}
\item
  Architecture client-serveur dans un réseau local
  \begin{itemize}
  \item
    Mise en place d'une architecture client/serveur simple (serveur
    d'authentification/de fichiers et
    postes clients associés)
  \end{itemize}
\item
  Introduction à la sécurité informatique
\end{itemize}
}

% Mots-clés
\ajoutRmotscles{Réseau, système d'exploitation, \textabbrv{TCP}/\textabbrv{IP}, \textabbrv{LAN}, hygiène informatique.}
