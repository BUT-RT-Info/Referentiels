%%%%%%%%%%%%%%%%%%%%%%%%%%%%%%%%%
% Ressources
%%%%%%%%%%%%%%%%%%%%%%%%%%%%%%%%%

\nouvelleressource{R201}{Technologie de l'Internet}

\ajoutRheures{60}{30}

%% Les compétences et les ACs
\ajoutRcompetence{RT1-Administrer}{\niveauA}

\ajoutRac{AC0113}{Configurer les fonctions de base du réseau local}
\ajoutRac{AC0115}{Identifier les dysfonctionnements du réseau local}
\ajoutRac{AC0116}{Installer un poste client}

\ajoutRcompetence{RT2-Connecter}{\niveauA}

\ajoutRac{AC0213}{Déployer des supports de transmission}

\ajoutRcompetence{RT3-Programmer}{\niveauA}

\ajoutRac{AC0311}{Utiliser un système informatique et ses outils}
% Les SAE
\ajoutRsae{SAÉ21}{Construire un réseau informatique pour une petite structure}
\ajoutRsae{SAÉ24}{Projet intégratif de S2}

% Les pre-requis
\ajoutRprerequis{R101}{Initiation aux réseaux informatiques}
\ajoutRprerequis{R102}{Principes et architecture des réseaux}
\ajoutRprerequis{R103}{Réseaux locaux et équipements actifs}

% Le descriptif
\ajoutRancrage{Cette ressource apporte le socle de connaissances et savoirs-faire pour
les compétences de cœur de métier ``Administrer les réseaux et
l'Internet'' (RT1) et ``Connecter les entreprises et les usagers''
(RT2). Elle vise à fournir à l'étudiant les connaissances et les
compétences indispensables pour pouvoir concevoir, déployer et maintenir
les infrastructures réseaux grande distance (Internet), plus précisément
l'adressage, le routage et le transport de paquets. Une première
approche du filtrage (sécurité) y est aussi abordée.
Elle contribue aussi à la compétence ``Créer des outils et applications
informatiques pour les R\&T'' (RT3) à travers la découverte du poste
client et de son environnement logiciel.
On introduira des notions de sécurité informatique (les ressources
associées aux recommandations de l'ANSSI, CyberEdu, CyberMalveillance
pourront servir de support)}

% Contenus
\ajoutRcontenudetaille{
\begin{itemize}
\item
  Protocole et adressage IPv4\&6.
\item
  Traduction d'adresses (NAT/PAT).
\item
  Routage statique et routage dynamique (OSPF).
\item
  TCP, UDP.
\item
  Politiques de filtrage ACL.
\end{itemize}
}

% Mots-clés
\ajoutRmotscles{Plan d'adressage, routage état de lien, stratégies de filtrage, , CIDR, VLSM, agrégation de routes, IPv6, NDP}
