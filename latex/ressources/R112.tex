%%%%%%%%%%%%%%%%%%%%%%%%%%%%%%%%%
% Ressources
%%%%%%%%%%%%%%%%%%%%%%%%%%%%%%%%%

\nouvelleressource{R112}{Projet Personnel et Professionnel}

\ajoutRheures{18}{15}

%% Les compétences et les ACs
\ajoutRcompetence{RT1-Administrer}{\niveauA}

\ajoutRac{AC0112}{Comprendre l'architecture des systèmes numériques et les principes du codage de l'information}

\ajoutRcompetence{RT2-Connecter}{\niveauA}

\ajoutRac{AC0212}{Caractériser des systèmes de transmissions élémentaires et découvrir la modélisation mathématique de leur fonctionnement}

\ajoutRcompetence{RT3-Programmer}{\niveauA}

\ajoutRac{AC0311}{Utiliser un système informatique et ses outils}
% Les SAE


% Les pre-requis


% Le descriptif
\ajoutRancrage{}

% Contenus
\ajoutRcontenudetaille{
\begin{itemize}
\item
  s'approprier son champ d'activité;
\item
  se constituer un panorama des métiers dans le domaine des réseaux et
  télécommunications;
\item
  se constituer un panorama des entreprises dans le secteur des réseaux
  et télécommunications;
\item
  se constituer un panorama des différents parcours du BUT R\&T pour
  pouvoir se positionner avec pertinence;
\item
  amener les étudiants à se projeter en tant que professionnels;
\item
  découvrir le portfolio et son utilisation dans la formation.
\end{itemize}
Activités notamment proposées dans cette ressource :
\begin{itemize}
\item
  rencontres d'entrepreneurs, de chefs de service, de techniciens et
  d'enseignants ;
\item
  visites d'entreprises, forums;
\item
  témoignages, relations avec d'anciens diplômés ;
\item
  découverte et compréhension d'un bassin d'emploi particulier ;
\item
  intérêt et prise en main d'un portfolio ;
\item
  déterminer ses atouts personnels.
\end{itemize}
}

% Mots-clés
\ajoutRmotscles{Métiers, entreprises, orientation, parcours, portfolio}
