%%%%%%%%%%%%%%%%%%%%%%%%%%%%%%%%%
% Ressources
%%%%%%%%%%%%%%%%%%%%%%%%%%%%%%%%%

\nouvelleressource{R112}{Projet Personnel et Professionnel}

\ajoutRheures{15}{12}

%% Les compétences et les ACs
\ajoutRcompetence{RT1-Administrer}{\niveauA}

\ajoutRcoeff{3}

\ajoutRac{AC0112}{Comprendre l'architecture des systèmes numériques et les principes du codage de l'information}

\ajoutRcompetence{RT2-Connecter}{\niveauA}

\ajoutRcoeff{3}

\ajoutRac{AC0212}{Caractériser des systèmes de transmissions élémentaires et découvrir la modélisation mathématique de leur fonctionnement}

\ajoutRcompetence{RT3-Programmer}{\niveauA}

\ajoutRcoeff{3}

\ajoutRac{AC0311}{Utiliser un système informatique et ses outils}
% Les SAE
\ajoutRsae{SAÉ14}{Se présenter sur Internet}

% Les pre-requis


% Le descriptif
\ajoutRancrage{}

% Contenus
\ajoutRcontenudetaille{
\vspace{-5pt}
\begin{itemize}
\item
  s'approprier son champ d'activité;
\item
  se constituer un panorama des métiers dans le domaine des réseaux et
  télécommunications;
\item
  se constituer un panorama des entreprises dans le secteur des réseaux
  et télécommunications;
\item
  se constituer un panorama des différents parcours du BUT R\&T pour
  pouvoir se positionner avec pertinence;
\item
  amener les étudiants à se projeter en tant que professionnels;
\item
  découvrir le portfolio et son utilisation dans la formation.
\end{itemize}
PROPOSITION ADIUT :
\begin{itemize}
\item
  d'avoir une compréhension exhaustive du référentiel de compétences de
  la formation et des éléments
  le structurant (composantes essentielles, niveaux, apprentissages
  critiques, famille de situations)
  ;
  \begin{itemize}
  \item
    de faire le lien entre les niveaux de compétences ciblés, les SAÉ et
    les ressources au programme
    de chaque semestre ;
  \item
    de découvrir les métiers associés à la spécialité et les
    environnements professionnels correspondants
    ;
  \item
    de se positionner sur un des parcours de la spécialité lorsque ces
    parcours sont proposés en seconde
    année ;
  \item
    de mobiliser les techniques de recrutement dans le cadre d'une
    recherche de stage ou d'un contrat
    d'alternance.
  \end{itemize}
\end{itemize}
Activités notamment proposées dans cette ressource :
\begin{itemize}
\item
  rencontres d'entrepreneurs, de chefs de service, de techniciens et
  d'enseignants ;
\item
  visites d'entreprises, forums;
\item
  témoignages, relations avec d'anciens diplômés ;
\item
  découverte et compréhension d'un bassin d'emploi particulier ;
\item
  intérêt et prise en main d'un portfolio ;
\item
  déterminer ses atouts personnels.
\end{itemize}
}

% Mots-clés
\ajoutRmotscles{Métiers, entreprises, orientation, parcours, portfolio}
