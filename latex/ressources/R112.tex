%%%%%%%%%%%%%%%%%%%%%%%%%%%%%%%%%
% Ressources
%%%%%%%%%%%%%%%%%%%%%%%%%%%%%%%%%

\nouvelleressource{R112}{Projet Personnel et Professionnel}

\ajoutRheures{15}{12}

%% Les compétences et les ACs
\ajoutRcompetence{RT1-Administrer}{\niveauA}

\ajoutRcoeff{3}

\ajoutRac{AC0112}{Comprendre l'architecture des systèmes numériques et les principes du codage de l'information}

\ajoutRcompetence{RT2-Connecter}{\niveauA}

\ajoutRcoeff{3}

\ajoutRac{AC0212}{Caractériser des systèmes de transmissions élémentaires et découvrir la modélisation mathématique de leur fonctionnement}

\ajoutRcompetence{RT3-Programmer}{\niveauA}

\ajoutRcoeff{3}

\ajoutRac{AC0311}{Utiliser un système informatique et ses outils}
% Les SAE
\ajoutRsae{SAÉ14}{Se présenter sur Internet}

% Les pre-requis


% Le descriptif
\ajoutRancrage{Le Projet Personnel et Professionnel (\textabbrv{PPP}) des semestres 1 et 2 de la
première année de B.U.T. permet à l'étudiant :
\begin{itemize}
\item
  s'approprier son champ d'activité;
\item
  se constituer un panorama des métiers dans le domaine des réseaux et
  télécommunications;
\item
  se constituer un panorama des entreprises dans le secteur des réseaux
  et télécommunications;
\item
  se constituer un panorama des différents parcours du BUT R\&T pour
  pouvoir se positionner avec pertinence;
\item
  amener les étudiants à se projeter en tant que professionnels en
  mobilisant les techniques de recrutement dans le
  cadre d'une recherche de stage ou d'un contrat d'alternance;
\item
  découvrir le portfolio et son utilisation dans la formation;
\item
  de faire le lien entre les niveaux de compétences ciblés, les SAÉ et
  les ressources au programme de chaque semestre ;
\item
  d'avoir une compréhension exhaustive du référentiel de compétences de
  la formation et des éléments le structurant
  (composantes essentielles, niveaux, apprentissages critiques, famille
  de situations).
\end{itemize}}

% Contenus
\ajoutRcontenudetaille{
Les activités pouvant être proposées dans cette ressource sont :
\begin{itemize}
\item
  Rencontres d'entrepreneurs, de chefs de service, de techniciens et
  d'enseignants;
\item
  Visites d'entreprises, forums;
\item
  Témoignages, relations avec d'anciens diplômés;
\item
  Découverte et compréhension d'un bassin d'emploi particulier;
\item
  Intérêt et prise en main d'un portfolio;
\item
  Déterminer ses atouts personnels.
\end{itemize}
}

% Mots-clés
\ajoutRmotscles{Métiers, Entreprises, Orientation, Parcours, Portfolio.}
