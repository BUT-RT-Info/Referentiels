%%%%%%%%%%%%%%%%%%%%%%%%%%%%%%%%%
% Ressources
%%%%%%%%%%%%%%%%%%%%%%%%%%%%%%%%%

\nouvelleressource{R103}{Réseaux locaux et équipements actifs}

\ajoutRheures{27}{16}

%% Les compétences et les ACs
\ajoutRcompetence{RT1-Administrer}{\niveauA}

\ajoutRcoeff{8}

\ajoutRac{AC0113}{Configurer les fonctions de base du réseau local}
\ajoutRac{AC0115}{Identifier les dysfonctionnements du réseau local}
\ajoutRac{AC0116}{Installer un poste client}

\ajoutRcompetence{RT2-Connecter}{\niveauA}

\ajoutRcoeff{4}

\ajoutRac{AC0213}{Déployer des supports de transmission}

\ajoutRcompetence{RT3-Programmer}{\niveauA}

\ajoutRcoeff{0}

\ajoutRac{AC0311}{Utiliser un système informatique et ses outils}
% Les SAE
\ajoutRsae{SAÉ11}{Se sensibiliser à l'hygiène informatique et à la cybersécurité}
\ajoutRsae{SAÉ12}{S'initier aux réseaux informatiques}

% Les pre-requis
\ajoutRprerequis{R101}{Initiation aux réseaux informatiques}

% Le descriptif
\ajoutRancrage{Cette ressource apporte le socle de connaissances et savoirs-faire pour
les compétences de cœur de métier «Administrer les réseaux et
l'Internet» (RT1). Elle vise à fournir à l'étudiant les connaissances et
les compétences indispensables pour pouvoir concevoir, déployer et
maintenir l'infrastructure réseau informatique de l'entreprise
(Ethernet).\\
La compétence RT1 est renforcée à travers la mise en place et la
configuration de matériels actifs comme des commutateurs, la gestion de
la sûreté de fonctionnement du réseau local Ethernet (spanning tree) et
la participation à la sécurisation du système d'information dont il est
le support (segmentation physique et virtuelle, \textabbrv{VLAN}). Ces deux
compétences s'appuient sur la compréhension des mécanismes intrinsèques
aux réseaux locaux Ethernet~: adressage \textabbrv{MAC}, commutation/routage de
niveau 2, \textabbrv{ARP}, passage d'un type de support physique à un autre,
changements de débit.\\
Pour la compétence «Connecter les entreprises et les usagers» (RT2),
elle aborde les notions d'exploitation du câblage (brassage).\\
Elle contribue aussi à la compétence «Créer des outils et applications
informatiques pour les R\&T» (RT3) à travers la découverte du poste
client et de son environnement logiciel.}

% Contenus
\ajoutRcontenudetaille{
\vspace{-10pt}
\begin{itemize}[topsep=5pt]
\item
  Câblage réseaux.
\item
  Différentes topologies physiques et logiques.
\item
  Normalisation Ethernet 802 (802.1, 802.2, 802.3).
\item
  Commutation Ethernet~: apprentissage des adresses \textabbrv{MAC}, diffusion,
  Broadcast.
\item
  Différents équipements actifs~: commutateur, routeur.
\item
  Configuration d'un réseau segmenté en \textabbrv{VLAN}, lien multi-vlan,
  communication inter-vlan.
\item
  Redondance et détection de boucles dans un réseau commuté~: \textabbrv{STP}, \textabbrv{RSTP}.
\end{itemize}
Sur chaque thème, faire le lien avec les notions de cybersécurité
abordées en R101.
}

% Mots-clés
\ajoutRmotscles{Réseaux locaux, Ethernet, commutateurs, routeurs, \textabbrv{VLAN}, 8021Q, 8021P, \textabbrv{STP}, \textabbrv{RSTP}.}
