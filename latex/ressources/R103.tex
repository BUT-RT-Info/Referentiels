%%%%%%%%%%%%%%%%%%%%%%%%%%%%%%%%%
% Ressources
%%%%%%%%%%%%%%%%%%%%%%%%%%%%%%%%%

\nouvelleressource{R103}{Réseaux locaux et équipements actifs}

\ajoutRheures{27}{16}

%% Les compétences et les ACs
\ajoutRcompetence{RT1-Administrer}{\niveauA}

\ajoutRcoeff{8}

\ajoutRac{AC0113}{Configurer les fonctions de base du réseau local}
\ajoutRac{AC0115}{Identifier les dysfonctionnements du réseau local}
\ajoutRac{AC0116}{Installer un poste client}

\ajoutRcompetence{RT2-Connecter}{\niveauA}

\ajoutRcoeff{4}

\ajoutRac{AC0213}{Déployer des supports de transmission}

\ajoutRcompetence{RT3-Programmer}{\niveauA}

\ajoutRcoeff{0}

\ajoutRac{AC0311}{Utiliser un système informatique et ses outils}
% Les SAE
\ajoutRsae{SAÉ11}{Se sensibiliser à l'hygiène informatique et à la cybersécurité}
\ajoutRsae{SAÉ12}{S'initier aux réseaux informatiques}

% Les pre-requis
\ajoutRprerequis{R101}{Initiation aux réseaux informatiques}

% Le descriptif
\ajoutRancrage{Cette ressource apporte le socle de connaissances et savoirs-faire pour
les compétences de cœur de métier ``Administrer les réseaux et
l'Internet'' (RT1). Elle vise à fournir à l'étudiant les connaissances
et les compétences indispensables pour pouvoir concevoir, déployer et
maintenir l'infrastructure réseau informatique de l'entreprise
(Ethernet)\\
Au niveau de la compétence RT2-Connecter, elle aborde les notions
d'exploitation du câblage (brassage). La compétence RT1-Administrer est
quant à elle renforcée à travers la mise en place et la configuration de
matériels actifs comme des commutateurs, la gestion de la sûreté de
fonctionnement du réseau local Ethernet (\emph{spanning tree}) et la
participation à la sécurisation du système d'information dont il est le
support (segmentation physique et virtuelle (\texttt{VLAN})). Ces deux
compétences s'appuient sur la compréhension des mécanismes intrinsèques
aux réseaux locaux Ethernet : adressage \texttt{MAC}, commutation/routage de
niveau 2, \texttt{ARP}, passage d'un type de support physique à un autre,
changements de débit, \ldots{}\\
Elle contribue aussi à la compétence ``Créer des outils et applications
informatiques pour les R\&T'' (RT3) à travers la découverte du poste
client et de son environnement logiciel.}

% Contenus
\ajoutRcontenudetaille{
\vspace{-5pt}
\begin{itemize}
\item
  Câblage réseaux.
\item
  Différentes topologies physiques et logiques.
\item
  Normalisation Ethernet 802 (802.1, 802.2, 802.3).
\item
  Commutation Ethernet : apprentissage des adresses \texttt{MAC}, diffusion,
  Broadcast.
\item
  Différents équipements actifs : commutateur, routeur\ldots{}
\item
  Configuration d'un réseau segmenté en \texttt{VLAN}, lien multi-\texttt{VLAN},
  communication inter-\texttt{VLAN}.
\item
  Redondance et détection de boucles dans un réseau commuté: \texttt{STP},
  \texttt{RSTP}\ldots{}
\end{itemize}
Sur chaque thème, faire le lien avec les notions de cybersécurité
abordées en R101.
}

% Mots-clés
\ajoutRmotscles{Réseaux locaux, Ethernet, Commutateurs, Routeurs, \texttt{VLAN}, 8021Q, 8021P, \texttt{STP}, \texttt{RSTP}}
