%%%%%%%%%%%%%%%%%%%%%%%%%%%%%%%%%
% Ressources
%%%%%%%%%%%%%%%%%%%%%%%%%%%%%%%%%

\nouvelleressource{R102}{Principes et architecture des réseaux}

\ajoutRheures{27}{15}

%% Les compétences et les ACs
\ajoutRcompetence{RT1-Administrer}{\niveauA}

\ajoutRcoeff{12}

\ajoutRac{AC0114}{Maîtriser les rôles et les principes fondamentaux des systèmes d'exploitation afin d'interagir avec ceux-ci pour la configuration et administration des réseaux et services fournis}
\ajoutRac{AC0115}{Identifier les dysfonctionnements du réseau local}

\ajoutRcompetence{RT2-Connecter}{\niveauA}





\ajoutRcompetence{RT3-Programmer}{\niveauA}




% Les SAE
\ajoutRsae{SAÉ11}{Se sensibiliser à l'hygiène informatique et à la cybersécurité}
\ajoutRsae{SAÉ12}{S'initier aux réseaux informatiques}

% Les pre-requis
\ajoutRprerequis{R101}{Initiation aux réseaux informatiques}
\ajoutRprerequis{R106}{Architecture des systèmes numériques et informatiques}

% Le descriptif
\ajoutRancrage{Cette ressource a pour objectif de donner à l'étudiant un cadre commun
et intégratif de l'ensemble des mécanismes nécessaires au fonctionnement
des réseaux informatiques. Ce cadre général est essentiel, et sert de
référence à l'ensemble des autres ressources réseaux.\\
Elle participe principalement à la compétence RT1 \og Administrer les
réseaux et l'Internet\fg{} à travers la compréhension et l'utilisation
de la partie réseau des systèmes d'exploitation, la compréhension de
l'interopérabilité des systèmes via les protocoles réseaux, ainsi que
les notions de services rendus et de performance du réseau.}

% Contenus
\ajoutRcontenudetaille{
\vspace{-5pt}
\begin{itemize}
\item
  Approche en couches et encapsulation.
\item
  Étude détaillée des protocoles Ethernet, \textabbrv{ARP}, \textabbrv{ICMP}.
\item
  Découverte des protocoles \textabbrv{IP}v4, \textabbrv{IP}v6, \textabbrv{ICMP}v6, \textabbrv{TCP}, \textabbrv{UDP} et des
  protocoles applicatifs.
\item
  Topologies de réseaux.
\item
  Principes de normalisation des technologies de l'Internet.
\item
  Notions sur les métriques de performances: débit, fiabilité, gigue,
  taux de pertes.
\end{itemize}
Outils préconisés : logiciels du type Wireshark, GNS3, Packet Tracer,
scapy, Marionnet.\\
Des éléments relatifs à la sécurité et aux risques informatiques et
réseaux seront progressivement introduits au travers des différents
contenus étudiés afin que ces éléments deviennent une préoccupation
routinière. Les éléments de cybersécurité pourront être abordés via des
exemples tels que l'arp-spoofing, la prise d'empreintes via \textabbrv{ICMP}, des
captures, la génération et analyse de trames. Des liens avec les aspects
sécurité informatique et réseaux mentionnés en R101 seront également
faits.
}

% Mots-clés
\ajoutRmotscles{Architecture en couches, topologies, protocoles, modèle \textabbrv{TCP}/\textabbrv{IP}.}
