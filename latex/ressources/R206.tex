%%%%%%%%%%%%%%%%%%%%%%%%%%%%%%%%%
% Ressources
%%%%%%%%%%%%%%%%%%%%%%%%%%%%%%%%%

\nouvelleressource{R206}{Numérisation de l'information}

\ajoutRheures{24}{12}

%% Les compétences et les ACs
\ajoutRcompetence{RT1-Administrer}{\niveauA}



\ajoutRac{AC0112}{Comprendre l'architecture des systèmes numériques et les principes du codage de l'information}

\ajoutRcompetence{RT2-Connecter}{\niveauA}

\ajoutRcoeff{10}

\ajoutRac{AC0211}{Mesurer et analyser les signaux}
\ajoutRac{AC0212}{Caractériser des systèmes de transmissions élémentaires et découvrir la modélisation mathématique de leur fonctionnement}

\ajoutRcompetence{RT3-Programmer}{\niveauA}




% Les SAE
\ajoutRsae{SAÉ22}{Mesurer et caractériser un signal ou un système}

% Les pre-requis
\ajoutRprerequis{R113}{Mathématiques du signal}
\ajoutRprerequis{R114}{Mathématiques des transmissions}
\ajoutRprerequis{R205}{Signaux et Systèmes pour les transmissions}

% Le descriptif
\ajoutRancrage{Cette ressource apporte le socle de connaissances et savoir-faire pour
les compétences de cœur de métier \og Administrer les réseaux et
l'Internet\fg{} (RT1) et \og Connecter les entreprises et les
usagers\fg{} (RT2)\\
Les systèmes de Réseaux et Télécoms véhiculent en permanence de données
numérisées. Ce module vient donc présenter les principes de la
numérisation de l'information, les contraintes de cette numérisation et
les conséquences sur la qualité du signal. Il trouvera des prolongements
en téléphonie, ou en télécommunications numériques.}

% Contenus
\ajoutRcontenudetaille{
\vspace{-5pt}
\begin{itemize}
\item
  Comprendre la notion de signal numérique, et le principe de la
  numérisation et de la restitution de
  signaux analogiques
\item
  Échantillonnage des signaux : choix d'une fréquence adéquate
  d'échantillonnage.
\item
  Quantification des signaux -- Erreur de quantification.
\item
  Filtre Anti-repliement et filtre de restitution.
\end{itemize}
}

% Mots-clés
\ajoutRmotscles{Numérisation, Échantillonnage, Quantification, Acquisition/restitution, \textabbrv{CAN}, \textabbrv{CNA}.}
