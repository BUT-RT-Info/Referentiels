%%%%%%%%%%%%%%%%%%%%%%%%%%%%%%%%%
% Ressources
%%%%%%%%%%%%%%%%%%%%%%%%%%%%%%%%%

\nouvelleressource{R210}{Anglais de communication et développement de l'anglais technique}

\ajoutRheures{45}{30}

%% Les compétences et les ACs
\ajoutRcompetence{RT1-Administrer}{\niveauA}

\ajoutRcoeff{3}

\ajoutRac{AC0112}{Comprendre l'architecture des systèmes numériques et les principes du codage de l'information}
\ajoutRac{AC0114}{Maîtriser les rôles et les principes fondamentaux des systèmes d'exploitation afin d'interagir avec ceux-ci pour la configuration et administration des réseaux et services fournis}

\ajoutRcompetence{RT2-Connecter}{\niveauA}

\ajoutRcoeff{9}

\ajoutRac{AC0215}{Communiquer avec un client ou un collaborateur}

\ajoutRcompetence{RT3-Programmer}{\niveauA}

\ajoutRcoeff{7}

\ajoutRac{AC0316}{S'intégrer dans un environnement propice au développement et au travail collaboratif}
% Les SAE
\ajoutRsae{SAÉ21}{Construire un réseau informatique pour une petite structure}
\ajoutRsae{SAÉ22}{Mesurer et caractériser un signal ou un système}
\ajoutRsae{SAÉ23}{Mettre en place une solution informatique pour l’entreprise}
\ajoutRsae{SAÉ24}{Projet intégratif}

% Les pre-requis
\ajoutRprerequis{R110}{Anglais de communication et initiation au vocabulaire technique}

% Le descriptif
\ajoutRancrage{Cette ressource apporte le socle de connaissances langagières pour les
compétences de cœur de métier ``Administrer les réseaux et l'Internet''
(RT1) et ``Connecter les entreprises et les usagers'' (RT2). Elle
contribue aussi à la compétence ``Créer des outils et applications
informatiques pour les R\&T'' (RT3) à travers des mises en situations,
jeux de rôle, dialogues qui permettent la prise de parole en continu et
en interaction, en développant les compétences de compréhension dans un
contexte professionnel technique.}

% Contenus
\ajoutRcontenudetaille{
\vspace{-5pt}
\begin{itemize}
\item
  Savoir structurer son discours oral et écrit (courriel, conversation
  téléphonique, visioconférence,
  réunion, débat).
\item
  Présenter son parcours professionnel à l'oral et à l'écrit (\textabbrv{CV}, lettre
  de motivation, entretien).
\item
  Analyser des problèmes et proposer des solutions.
\item
  Faire un exposé technique.
\item
  Développer le vocabulaire technique des domaines cibles.
\item
  Extensions possibles : télécollaboration, télétandem, parcours
  international, dispositif \textabbrv{EMILE}.
\end{itemize}
}

% Mots-clés
\ajoutRmotscles{Anglais général et technique, Situations de communication professionnelle, Expression et compréhension.}
