%%%%%%%%%%%%%%%%%%%%%%%%%%%%%%%%%
% Ressources
%%%%%%%%%%%%%%%%%%%%%%%%%%%%%%%%%

\nouvelleressource{R212}{Projet Personnel et Professionnel}

\ajoutRheures{15}{12}

%% Les compétences et les ACs
\ajoutRcompetence{RT1-Administrer}{\niveauA}

\ajoutRcoeff{1}

\ajoutRac{AC0115}{Identifier les dysfonctionnements du réseau local}

\ajoutRcompetence{RT2-Connecter}{\niveauA}

\ajoutRcoeff{3}

\ajoutRac{AC0215}{Communiquer avec un client ou un collaborateur}

\ajoutRcompetence{RT3-Programmer}{\niveauA}

\ajoutRcoeff{2}

\ajoutRac{AC0316}{S'intégrer dans un environnement propice au développement et au travail collaboratif}
% Les SAE
\ajoutRsae{SAÉ24}{Projet intégratif}

% Les pre-requis
\ajoutRprerequis{R112}{Projet Personnel et Professionnel}

% Le descriptif
\ajoutRancrage{Cette ressource permettra à l'étudiant de :\\
-- d'avoir une compréhension exhaustive du référentiel de compétences de
la formation et des éléments le structurant\\
-- de faire le lien entre les niveaux de compétences ciblés, les SAÉ et
les ressources au programme de chaque semestre ;\\
-- de se positionner sur un des parcours de la spécialité lorsque ces
parcours sont proposés en seconde année ;\\
-- de mobiliser les techniques de recrutement dans le cadre d'une
recherche de stage ou d'un contrat d'alternance\\
-- se caractériser pour préparer son stage ou son alternance\\
-- se présenter, se définir;\\
-- exprimer l'intérêt professionnel, valeurs, motivations, traits de
personnalité, expériences professionnelles ou personnelles;\\
-- mettre en valeur et présenter son savoir-être;\\
-- identifier ses compétences;\\
-- préciser et exprimer ses souhaits professionnels.}

% Contenus
\ajoutRcontenudetaille{
Activités notamment proposées dans cette ressource :
\begin{itemize}
\item
  enquête métiers et veille professionnelle ;
\item
  rencontres et entretiens avec des professionnels et anciens étudiants
  ;
\item
  visite d'entreprise ou d'organisation ;
\item
  participation à des conférences métiers ;
\item
  construction d'une identité professionnelle numérique;
\item
  découverte et compréhension d'un bassin d'emploi particulier.
\end{itemize}
}

% Mots-clés
\ajoutRmotscles{\textabbrv{CV}, Lettre de motivation, Entretien de recrutement, Stage, Alternance.}
