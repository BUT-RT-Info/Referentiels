%%%%%%%%%%%%%%%%%%%%%%%%%%%%%%%%%
% Ressources
%%%%%%%%%%%%%%%%%%%%%%%%%%%%%%%%%

\nouvelleressource{R203}{Bases des services réseaux}

\ajoutRheures{30}{18}

%% Les compétences et les ACs
\ajoutRcompetence{RT1-Administrer}{\niveauA}

\ajoutRcoeff{12}

\ajoutRac{AC0113}{Configurer les fonctions de base du réseau local}
\ajoutRac{AC0114}{Maîtriser les rôles et les principes fondamentaux des systèmes d'exploitation afin d'interagir avec ceux-ci pour la configuration et administration des réseaux et services fournis}
\ajoutRac{AC0115}{Identifier les dysfonctionnements du réseau local}

\ajoutRcompetence{RT2-Connecter}{\niveauA}





\ajoutRcompetence{RT3-Programmer}{\niveauA}




% Les SAE
\ajoutRsae{SAÉ21}{Construire un réseau informatique pour une petite structure}
\ajoutRsae{SAÉ24}{Projet intégratif}

% Les pre-requis
\ajoutRprerequis{R101}{Initiation aux réseaux informatiques}
\ajoutRprerequis{R102}{Principes et architecture des réseaux}
\ajoutRprerequis{R108}{Bases des systèmes d'exploitation}

% Le descriptif
\ajoutRancrage{Cette ressource apporte les connaissances et compétences de base
nécessaires à la mise en oeuvre des services réseaux dans un système
d'information\\
Les services abordés sont des services essentiels à tout système
d'information tels que le \textabbrv{DNS}, le \textabbrv{DHCP} ou le transfert de fichiers pour
les configurations d'appareils réseaux pour n'en citer que quelques-uns.
Cette découverte des premiers protocoles applicatifs permettra également
de sensibiliser les étudiants aux risques de sécurité liés à la
configuration de ces services\\
On introduira des notions de sécurité informatique (les ressources
associées aux recommandations de l'\textabbrv{ANSSI}, CyberEdu, CyberMalveillance
pourront servir de support).}

% Contenus
\ajoutRcontenudetaille{
\vspace{-5pt}
\begin{itemize}
\item
  Rappels sur les protocoles de transport (\textabbrv{TCP}, \textabbrv{UDP}).
\item
  Utilisation de ssh pour l'accès distant.
\item
  Principe, installation, configuration et tests des services :
  \begin{itemize}
  \item
    \textabbrv{DHCP};
  \item
    \textabbrv{DNS} (fonctions de base);
  \item
    \textabbrv{HTTP};
  \item
    \textabbrv{TFTP}, \textabbrv{FTP};
  \item
    \textabbrv{NTP}.
  \end{itemize}
\end{itemize}
}

% Mots-clés
\ajoutRmotscles{Protocoles et ports applicatifs, Services, Systèmes d'exploitation.}
