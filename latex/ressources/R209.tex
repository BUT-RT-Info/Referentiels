%%%%%%%%%%%%%%%%%%%%%%%%%%%%%%%%%
% Ressources
%%%%%%%%%%%%%%%%%%%%%%%%%%%%%%%%%

\nouvelleressource{R209}{Initiation au développement Web}

\ajoutRheures{24}{15}

%% Les compétences et les ACs
\ajoutRcompetence{RT1-Administrer}{\niveauA}

\ajoutRcoeff{2}

\ajoutRac{AC0112}{Comprendre l'architecture des systèmes numériques et les principes du codage de l'information}
\ajoutRac{AC0114}{Maîtriser les rôles et les principes fondamentaux des systèmes d'exploitation afin d'interagir avec ceux-ci pour la configuration et administration des réseaux et services fournis}

\ajoutRcompetence{RT2-Connecter}{\niveauA}





\ajoutRcompetence{RT3-Programmer}{\niveauA}

\ajoutRcoeff{10}

\ajoutRac{AC0311}{Utiliser un système informatique et ses outils}
\ajoutRac{AC0312}{Lire, exécuter, corriger et modifier un programme}
\ajoutRac{AC0313}{Traduire un algorithme, dans un langage et pour un environnement donné}
\ajoutRac{AC0314}{Connaître l'architecture et les technologies d'un site Web}
\ajoutRac{AC0315}{Choisir les mécanismes de gestion de données adaptés au développement de l'outil}
\ajoutRac{AC0316}{S'intégrer dans un environnement propice au développement et au travail collaboratif}
% Les SAE
\ajoutRsae{SAÉ23}{Mettre en place une solution informatique pour l’entreprise}
\ajoutRsae{SAÉ24}{Projet intégratif}

% Les pre-requis
\ajoutRprerequis{R107}{Fondamentaux de la programmation}
\ajoutRprerequis{R109}{Introduction aux technologies Web}
\ajoutRprerequis{R207}{Sources de données}
\ajoutRprerequis{R208}{Analyse et traitement de données structurées}

% Le descriptif
\ajoutRancrage{Le professionnel R\&T peut être amené à développer, pour ses besoins
personnels ou pour ses collaborateurs, un site Web, par exemple pour
fournir une interface de présentation du réseau informatique :
compétence \og Créer des outils et applications informatiques pour les
R\&T\fg{} (RT3).\\
Il doit en appréhender tous les éléments : il doit aussi bien connaître
les protocoles de communication du Web que veiller à la sécurité de
ceux-ci. Il doit également pouvoir accéder, traiter et afficher des
informations provenant de différentes sources de données telles que des
\textabbrv{SGBD}, des \textabbrv{API} ou des fichiers structurés. La présente ressource
contribue aux apprentissages critiques mentionnés précédemment.}

% Contenus
\ajoutRcontenudetaille{
\vspace{-5pt}
\begin{itemize}
\item
  Introduction au protocole \textabbrv{HTTP}.
\item
  Mise en forme de pages Web :
  \begin{itemize}
  \item
    balises \textabbrv{HTML} avancées;
  \item
    structure d'une page avec son \textabbrv{DOM};
  \item
    \textabbrv{CSS} avancé ou Framework;
  \item
    initiation au dynamisme côté client (JavaScript, bibliothèques comme
    jQuery)
  \end{itemize}
\item
  Scripts côté serveur.
\item
  Eléments d'interaction client-serveur (requête \textabbrv{HTTP}, \textabbrv{URL}, formulaire).
\item
  Interrogation d'un \textabbrv{SGBD} ou d'une \textabbrv{API}.
\item
  Sensibilisation à la sécurisation de sites : failles \textabbrv{XSS}, \textabbrv{XSS} stockée,
  injections \textabbrv{SQL}.
\end{itemize}
L'utilisation de l'anglais est préconisée dans la documentation du code.
}

% Mots-clés
\ajoutRmotscles{Web, Développement, Algorithmes, \textabbrv{SGBD}, \textabbrv{API}, Sécurité, Client-serveur.}
