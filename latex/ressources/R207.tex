%%%%%%%%%%%%%%%%%%%%%%%%%%%%%%%%%
% Ressources
%%%%%%%%%%%%%%%%%%%%%%%%%%%%%%%%%

\nouvelleressource{R207}{Sources de données}

\ajoutRheures{20}{12}

%% Les compétences et les ACs
\ajoutRcompetence{RT1-Administrer}{\niveauA}

\ajoutRcoeff{0}

\ajoutRac{AC0112}{Comprendre l'architecture des systèmes numériques et les principes du codage de l'information}

\ajoutRcompetence{RT2-Connecter}{\niveauA}





\ajoutRcompetence{RT3-Programmer}{\niveauA}

\ajoutRcoeff{10}

\ajoutRac{AC0311}{Utiliser un système informatique et ses outils}
\ajoutRac{AC0312}{Lire, exécuter, corriger et modifier un programme}
\ajoutRac{AC0313}{Traduire un algorithme, dans un langage et pour un environnement donné}
\ajoutRac{AC0314}{Connaître l'architecture et les technologies d'un site Web}
\ajoutRac{AC0315}{Choisir les mécanismes de gestion de données adaptés au développement de l'outil}
\ajoutRac{AC0316}{S'intégrer dans un environnement propice au développement et au travail collaboratif}
% Les SAE
\ajoutRsae{SAÉ23}{Mettre en place une solution informatique pour l'entreprise}
\ajoutRsae{SAÉ24}{Projet intégratif}

% Les pre-requis
\ajoutRprerequis{R107}{Fondamentaux de la programmation}
\ajoutRprerequis{R109}{Introduction aux technologies Web}

% Le descriptif
\ajoutRancrage{Le professionnel R\&T traite un grand nombre de données, comme
l'annuaire des utilisateurs du réseau ou l'état des équipements
informatiques. Elles peuvent servir à gérer les services réseau d'une
entreprise (RT1) ou à alimenter les pages d'un site Web (RT3). Il est
donc amené à stocker, organiser, gérer, protéger des données provenant
de différentes sources (thématiques du référentiel PIX,
\url{https://pix.fr/competences}), mais aussi à les traiter en
développant différents outils informatiques pour ses besoins personnels
ou pour son équipe (RT3). Plus largement, il contribue activement à
l'exploitation et à la maintenance du système d'information de
l'entreprise. Cette ressource introduit les systèmes de gestion de base
de données. Elle présente différentes alternatives technologiques pour
le stockage et le codage de l'information en fonction des données et de
leur usage. L'accès aux données utilise des langages et des outils
spécifiques qui seront introduits.}

% Contenus
\ajoutRcontenudetaille{
\vspace{-5pt}
\begin{itemize}[topsep=5pt]
\item
  Stockage et accès aux données~:
  \begin{itemize}
  \item
    Système de gestion de données (relationnel/non relationnel)\,;
  \item
    Structuration des données~: fichiers (\textabbrv{CSV}, \textabbrv{JSON}), exemples de sources
    ouvertes (open data), web scraping\,;
  \item
    Sensibilisation à la réglementation française et internationale
    (\textabbrv{CNIL}, \textabbrv{RGPD}).
  \end{itemize}
\item
  Base de données relationnelles~:
  \begin{itemize}
  \item
    Schéma relationnel d'une base de données\,;
  \item
    Sensibilisation aux contraintes d'intégrité\,;
  \item
    Création de tables simples\,;
  \item
    Interrogation de données, ajout et modification de données.
  \end{itemize}
\item
  Lecture d'une documentation technique (\textabbrv{UML}, diagramme de classes).
\end{itemize}
L'utilisation de l'anglais est préconisée dans la documentation du code.
}

% Mots-clés
\ajoutRmotscles{Base de données, Langages informatiques, Programmation, Algorithmes.}
