%%%%%%%%%%%%%%%%%%%%%%%%%%%%%%%%%
% Ressources
%%%%%%%%%%%%%%%%%%%%%%%%%%%%%%%%%

\nouvelleressource{R104}{Fondamentaux des systèmes électroniques}

\ajoutRheures{33}{18}

%% Les compétences et les ACs
\ajoutRcompetence{RT1-Administrer}{\niveauA}

\ajoutRcoeff{9}

\ajoutRac{AC0111}{Maîtriser les lois fondamentales de l'électricité afin d'intervenir sur des équipements de réseaux et télécommunications}

\ajoutRcompetence{RT2-Connecter}{\niveauA}

\ajoutRcoeff{5}

\ajoutRac{AC0211}{Mesurer et analyser les signaux}

\ajoutRcompetence{RT3-Programmer}{\niveauA}




% Les SAE
\ajoutRsae{SAÉ13}{Découvrir un dispositif de transmission}

% Les pre-requis


% Le descriptif
\ajoutRancrage{Cette ressource apporte le socle de connaissances et savoir-faire pour
les compétences de cœur de métier \og Administrer les réseaux et
l'Internet\fg{} (RT1) et \og Connecter les entreprises et les
usagers\fg{} (RT2).\\
La connaissance des phénomènes électriques, la maîtrise des grandeurs
électriques et de leurs mesures, ainsi que la notion de puissance
permettent à l'étudiant de comprendre le fonctionnement des systèmes
télécom et de travailler sur les signaux.\\
Les notions de dimensionnement électrique concourent à la sécurité du
fonctionnement des équipements réseaux et télécoms. La puissance
maximale permet d'aborder les problèmes d'adaptation d'impédance.\\
A travers des exercices de mise en place de circuits simples, les
étudiants seront capables d'implanter des circuits, de placer les
instruments de mesure et d'interpréter les résultats.}

% Contenus
\ajoutRcontenudetaille{
\vspace{-10pt}
\begin{itemize}[topsep=5pt]
\item
  Lois de base de l'électricité, théorèmes fondamentaux, pont diviseur
\item
  Résistance et Condensateur. Savoir réaliser un circuit simple et
  savoir brancher les appareils de
  mesure sur platine d'expérimentation
\item
  Mesure de signaux avec calculs simples (voltmètre, tension moyenne,
  efficace\ldots)
\item
  Représentation temporelle des signaux simples. Utilisation de
  l'oscilloscope (chronogramme).
\item
  Définition de la puissance électrique. Adaptation ``d'impédance'' par
  le calcul de la puissance maximale.
\item
  Dimensionnement des puissances d'une installation télécom ou réseau.
  Sensibilisation à la sécurité
  électrique et au Développement Durable. Coût de fonctionnement des
  équipements.
\item
  Exemples~: dimensionnement d'une alimentation pour des serveurs,
  limite de puissance sur un câble
  (alternatif ou continu).
\end{itemize}
}

% Mots-clés
\ajoutRmotscles{Mesures, Oscilloscope, Voltmètre, Puissance, Dimensionnement, Adaptation d'impédance.}
