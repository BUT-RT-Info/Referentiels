%%%%%%%%%%%%%%%%%%%%%%%%%%%%%%%%%
% Ressources
%%%%%%%%%%%%%%%%%%%%%%%%%%%%%%%%%

\nouvelleressource{R205}{Signaux et Systèmes pour les transmissions}

\ajoutRheures{36}{15}

%% Les compétences et les ACs
\ajoutRcompetence{RT1-Administrer}{\niveauA}

\ajoutRcoeff{2}

\ajoutRac{AC0111}{Maîtriser les lois fondamentales de l'électricité afin d'intervenir sur des équipements de réseaux et télécommunications}

\ajoutRcompetence{RT2-Connecter}{\niveauA}

\ajoutRcoeff{12}

\ajoutRac{AC0211}{Mesurer et analyser les signaux}
\ajoutRac{AC0212}{Caractériser des systèmes de transmissions élémentaires et découvrir la modélisation mathématique de leur fonctionnement}

\ajoutRcompetence{RT3-Programmer}{\niveauA}




% Les SAE
\ajoutRsae{SAÉ22}{Mesurer et caractériser un signal ou un système}

% Les pre-requis
\ajoutRprerequis{R104}{Fondamentaux des systèmes électroniques}
\ajoutRprerequis{R113}{Mathématiques du signal}
\ajoutRprerequis{R114}{Mathématiques des transmissions}

% Le descriptif
\ajoutRancrage{Cette ressource apporte le socle de connaissances et savoir-faire pour
les compétences de cœur de métier «Administrer les réseaux et
l'Internet» (RT1) et «Connecter les entreprises et les usagers» (RT2)\\
La caractérisation du comportement d'un système télécom en fonction de
la fréquence permet au technicien d'appréhender la notion de bande
passante et d'introduire celle de canal de transmission.\\
La représentation spectrale des signaux permet de comprendre quelles
modifications ces signaux vont subir dans un système télécom.}

% Contenus
\ajoutRcontenudetaille{
Étude de la fonction de transfert d'un système linéaire\,; notion de
filtrage\,; réponse fréquentielle d'un support de transmission\,; notion de
bande passante. Atténuation, amplification des systèmes.\\
Représentations temporelles et fréquentielles des signaux\,; analyse
spectrale de signaux réels (exemples~: audio, WiFi, \textabbrv{ADSL}).\\
Influence de la fonction de transfert d'un système sur un signal
(exemples~: audio, numérique).\\
Bilans de liaison de systèmes de transmissions.
}

% Mots-clés
\ajoutRmotscles{Représentations temporelles et fréquentielles des signaux, fonction de transfert, bande passante, analyse spectrale.}
