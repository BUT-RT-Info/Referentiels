%%%%%%%%%%%%%%%%%%%%%%%%%%%%%%%%%
% Ressources
%%%%%%%%%%%%%%%%%%%%%%%%%%%%%%%%%

\nouvelleressource{R105}{Supports de transmission pour les réseaux locaux}

\ajoutRheures{12}{6}

%% Les compétences et les ACs
\ajoutRcompetence{RT1-Administrer}{\niveauA}





\ajoutRcompetence{RT2-Connecter}{\niveauA}

\ajoutRcoeff{5}

\ajoutRac{AC0211}{Mesurer et analyser les signaux}
\ajoutRac{AC0213}{Déployer des supports de transmission}

\ajoutRcompetence{RT3-Programmer}{\niveauA}




% Les SAE
\ajoutRsae{SAÉ13}{Découvrir un dispositif de transmission}

% Les pre-requis


% Le descriptif
\ajoutRancrage{Cette ressource apporte les bases de connaissances et savoir-faire
techniques pour la compétence \og connecter les entreprises et les
usagers\fg{} à travers les apprentissages critiques \og mesurer et
analyser les signaux\fg{} et \og déployer des supports de
transmission\fg{}.\\
Il s'agit d'étudier les concepts fondamentaux des supports de
transmission.}

% Contenus
\ajoutRcontenudetaille{
\vspace{-5pt}
\begin{itemize}
\item
  Types de support de transmission (réseau d'entreprise, réseau
  opérateur)
\item
  Caractéristiques d'un ou plusieurs types de supports (exemples: retard
  de propagation, atténuation,
  continuité, échos, bruit, perturbations, identifier un défaut, bande
  passante ) à partir de mesures
  et d'analyse des signaux
\item
  Prolongement possible : recettage, certification \textabbrv{LAN}.
\end{itemize}
}

% Mots-clés
\ajoutRmotscles{Supports de transmission (fibre optique, cuivre, radio), Mesures.}
