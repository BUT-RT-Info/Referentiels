%%%%%%%%%%%%%%%%%%%%%%%%%%%%%%%%%
% Ressources
%%%%%%%%%%%%%%%%%%%%%%%%%%%%%%%%%

\nouvelleressource{R115}{Gestion de projet}

\ajoutRheures{12}{4}

%% Les compétences et les ACs
\ajoutRcompetence{RT1-Administrer}{\niveauA}





\ajoutRcompetence{RT2-Connecter}{\niveauA}

\ajoutRcoeff{2}

\ajoutRac{AC0215}{Communiquer avec un client ou un collaborateur}

\ajoutRcompetence{RT3-Programmer}{\niveauA}

\ajoutRcoeff{4}

\ajoutRac{AC0316}{S'intégrer dans un environnement propice au développement et au travail collaboratif}
% Les SAE
\ajoutRsae{SAÉ11}{Se sensibiliser à l'hygiène informatique et à la cybersécurité}
\ajoutRsae{SAÉ12}{S'initier aux réseaux informatiques}
\ajoutRsae{SAÉ13}{Découvrir un dispositif de transmission}
\ajoutRsae{SAÉ14}{Se présenter sur Internet}
\ajoutRsae{SAÉ15}{Traiter des données}

% Les pre-requis


% Le descriptif
\ajoutRancrage{Le professionnel R\&T peut être impliqué dans différents projets
l'amenant à travailler en équipe.}

% Contenus
\ajoutRcontenudetaille{
Dans le cadre de cette ressource transversale, l'étudiant devra :
\begin{itemize}
\item
  Partager de façon collective l'information :
  \begin{itemize}
  \item
    Utilisation avancée du courriel : création d'une adresse générique,
    utilisation du \textabbrv{CC} et du \textabbrv{CCi}.
  \item
    Utilisation d'outils collaboratifs adaptés (par ex : Mattermost,
    Slack, MSTeams, Google Drive,
    OnlyOffice)
  \end{itemize}
\item
  Organiser son travail et celui de l'équipe à partir d'outils de
  planification (Gantt, \textabbrv{PERT})
\item
  Prendre sa place dans une équipe en connaissant les différents rôles
  d'une équipe projet
\item
  Conceptualiser les étapes des tâches à réaliser à l'aide d'outils
  adaptés (cartes mentales, infographies,
  etc.)
\item
  Prendre conscience des délais et échéances dans un travail en mode
  projet
\item
  Savoir s'adapter à des profils professionnels différents (manager,
  collaborateur, client) qui interviennent
  dans un projet
\item
  Apprendre à faire un bilan régulier sur l'avancée d'un projet : points
  bloquants, solutions apportées
\item
  Appliquer la critique constructive dans l'intérêt du projet
\item
  Organiser des réunions de projet
\item
  Présenter un projet selon ses spécificités et le public visé.
\end{itemize}
}

% Mots-clés
\ajoutRmotscles{Planification, Partage d'informations, Organisation, Conceptualisation, Réunion.}
