%%%%%%%%%%%%%%%%%%%%%%%%%%%%%%%%%
% Parcours PilPro
% (Parcours D)
%%%%%%%%%%%%%%%%%%%%%%%%%%%%%%%%%

\nouveauparcours{Pilotage de Projets de Réseaux}{PilPro}

%% Métiers accessibles
\ajoutmetier{Responsable d'affaires clients (rattaché/ée aux services techniques)}
\ajoutmetier{\Conducteur/  de travaux cuivre ou fibre ou mobile}
\ajoutmetier{Bureaux d'études déploiement cuivre ou fibre ou radio}
\ajoutmetier{\Technicien/ avant-vente -- \Technicien/ de production}
\ajoutmetier{\Coordinateur/ de projet R\&T}
\ajoutmetier{\Chargé/ d’études télécoms, FTTH, aménagement réseaux, vie du réseaux}
\ajoutmetier{Pilote d’activités réseaux et télécoms -- Pilote de production réseaux}
\ajoutmetier{\Chef/ de projets techniques en R\&T (déploiement réseaux et services)}



%%%%%%%%%%%%%%%%%%%%%%%%%%%%%%%%%%%%%
% COMPETENCE SPECIFIQUE n°1 
% (compétence SA)
%%%%%%%%%%%%%%%%%%%%%%%%%%%%%%%%%%%%%

\ajoutcompetence{PilPro1}{Mettre en \oe{}uvre des projets techniques et réglementaires des R\&T dans son activité
}{%
Un professionnel R\&T doit avoir une bonne compréhension des technologies des réseaux informatiques et de télécommunications. Ces connaissances métiers lui permettent de dialoguer avec les différents acteurs d’un projet ou d’un contrat. Il a la capacité d’analyser, de suivre, coordonner puis de piloter les demandes internes et externes des entreprises, organismes privés ou publics, ainsi que les composantes des contrats, dans le respect des normes techniques et réglementaires de son activité. Il suit les évolutions technologiques dans le souci de l’empreinte environnementale.}


% Composantes essentielles => mot-clé En
\ajoutcompo{maîtrisant les enjeux techniques et réglementaires des nouvelles technologies}
\ajoutcompo{pilotant un projet technique R\&T} 
\ajoutcompo{faisant preuve de vision stratégique en phase avec le marché des réseaux et des télécoms}
\ajoutcompo{collaborant de façon responsable avec des équipes}

% Situations professionnelles
\ajoutsitupro{Adéquation technique des solutions réseaux informatiques et télécoms à la demande client}
\ajoutsitupro{Élaboration de solutions techniques clients adaptées}
\ajoutsitupro{Accompagnement technique de la mise en place des solutions clients}

% Niveaux de développement de la compétence commune

%% Niveau 1
% \ajout_niveau{nom_de_la_competence}{lettre_competence}{titre_court}{description}
\ajoutniveau{}{%
Mettre en œuvre un projet R\&T}

\ajoutapprentissage{}{Compréhension d’un cahier des charges technique R\&T \motscles{avec le détail technique et la compréhension du contenu et de l’articulation de l’ensemble}}
\ajoutapprentissage{}{Planification des étapes d’un projet technique R\&T \motscles{comprendre le projet et savoir le découper en tâches qui pourront se répartir en temps et en personne ; Gestion des ressources techniques}}
\ajoutapprentissage{}{Co-animation d’une équipe technique \motscles{parler et échanger sur des solutions techniques R\&T}}
\ajoutapprentissage{}{Proposition de solutions techniques efficientes}
\ajoutapprentissage{}{Échanges vulgarisés ou techniques avec tous les acteurs d’un projet}


%% Niveau 2
\ajoutniveau{}{%
Opérationnaliser un projet R\&T}

\ajoutapprentissage{}{Rédaction d’un appel d’offres ou d’un cahier des charges technique}
\ajoutapprentissage{}{Animation  technique d’équipes pluridisciplinaires}
\ajoutapprentissage{}{Coordination d’équipes sur une partie de projet ou sa totalité}
\ajoutapprentissage{}{Mise en place de solutions techniques efficientes}
\ajoutapprentissage{}{Livraison et suivi technique de projet}


%%%%%%%%%%%%%%%%%%%%%%%%%%%%%%%%%%%%%
% COMPETENCE SPECIFIQUE n°2 
% (compétence SB)
%%%%%%%%%%%%%%%%%%%%%%%%%%%%%%%%%%%%%
\ajoutcompetence{PilPro2}{Gérer des activités réseaux et télécoms en termes organisationnels, relationnels, financiers et commerciaux}{%
Le monde des réseaux et télécommunications est très évolutif, concurrentiel et complexe. Il nécessite une bonne connaissance des axes de développement des entreprises, de la politique nationale et de la législation qui l’accompagne. 
A différents titres, en tant que responsable d’affaires client, chargé d’études, chef de projet technique, technico-commercial, il analyse une demande dans ses dimensions les plus larges et propose une offre adaptée à un client. Ainsi, il organise et anime des réunions, il participe à la rédaction et aux évolutions de cahiers des charges, il rend compte de l’activité (analyse, évalue, communique) et il veille au respect des coûts, de la qualité et des délais, ainsi qu’à la conformité de la mise en œuvre des décisions, des demandes, ou des engagements contractuels.}


% Composantes essentielles => mot-clé En
\ajoutcompo{pilotant avec agilité des solutions techniques}
\ajoutcompo{sachant communiquer à l’écrit et à l’oral avec tous les acteurs d’un projet}
\ajoutcompo{respectant des contraintes technico-économiques (financières, éthiques, temporelles, contractuelles, qualité)}

% Situations professionnelles
\ajoutsitupro{Communication et stratégie technique en interne et en externe pour des projets R\&T}
\ajoutsitupro{Suivi des objectifs opérationnels de projets R\&T \motscles{animation des réunions de suivi, analyse des résultats obtenus, remontée et suivi des risques, définition des axes d’amélioration}}
\ajoutsitupro{Pilotage de la relation client}


% Niveaux de développement de la compétence commune

%% Niveau 1
\ajoutniveau{}{%
Élaborer un projet technique}

\ajoutapprentissage{}{Prise en compte des contraintes d’un pilotage de projet \motscles{contraintes de temps, financières, du besoin, éthique, qualité, norme, juridique, …}}
\ajoutapprentissage{}{Planification de solutions techniques R\&T efficientes \motscles{savoir répartir des tâches en temps et en personne, les prioriser, utilisation d’outils}}
\ajoutapprentissage{}{Prise de conscience des enjeux de la communication dans les relations interpersonnelles \motscles{savoir écouter, savoir prendre la parole, savoir parler, vulgariser}}
\ajoutapprentissage{}{Établissement d’un relationnel de qualité}


%% Niveau 2
\ajoutniveau{}{%
Mettre en place et suivre un projet technique}

\ajoutapprentissage{}{Rigueur dans le pilotage d’un projet dans sa globalité \motscles{rigueur budgétaire, rigueur des choix de solutions techniques, rigueur dans le planning, rigueur dans la com}}
\ajoutapprentissage{}{Flexibilité dans la gestion des équipes et des tâches \motscles{savoir prioriser les tâches, être agile pour tenir compte des évolutions du client, des évolutions possibles au sein de l’équipe ou de l’entreprise, des évolutions financières ou réglementaires ou contractuelles du marché ou du client, évolution des outils pour piloter les projets}}
\ajoutapprentissage{}{Prise de responsabilité envers les équipes \motscles{savoir coordonner ou savoir s’intégrer en bonne intelligence, savoir animer des réunions, analyse des résultats obtenus, remontée et suivi des risques, définition des axes d’amélioration}} 
\ajoutapprentissage{}{Valorisation de solutions déployées, ou d’offres techniques, ou d’offres commerciales}
\ajoutapprentissage{}{Force de proposition de solutions innovantes \motscles{proposer des évolutions ou de nouvelles solutions et justifier auprès de profils d’interlocuteurs différents ou évaluer les données d’un marché ou savoir collecter et synthétiser un rapport d’audit sur des nouvelles solutions techniques}}

