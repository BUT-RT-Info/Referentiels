%%%%%%%%%%%%%%%%%%%%%%%%%%%%%%%%%
% Parcours CYBER
% (Parcours A)
%%%%%%%%%%%%%%%%%%%%%%%%%%%%%%%%%

\nouveauparcours{Internet des Objets et Mobilité}{IOM}
%{
%Le diplômé Réseaux Mobiles et Internet des Objets maîtrise les technologies de communication entre objets mobiles et communicants : téléphones, véhicules, capteurs de toute nature. Le secteur de l'Internet des objets est en plein développement, avec de nombreux métiers émergents. Les solutions de communication mobiles (3G, 4G, 5G) - plus mûres - continuent à générer de nombreux emplois de techniciens supérieurs et sont de plus en plus utilisées pour l'Internet des objets.
%}

%% Métiers accessibles 
\ajoutmetier{\Technicien/ de maintenance exploitation -- \Technicien/ de maintenance réseaux mobiles}
\ajoutmetier{\Technicien/ télécom radio -- \Technicien/ support réseau mobile}
\ajoutmetier{\Intégrateur/ de réseaux mobiles -- Superviseur/eure de réseaux mobiles}
\ajoutmetier{\Technicien/ support réseau mobile et IoT -- \Technicien/ audit liaison sans fil}
\ajoutmetier{\Administrateur/ -- \Intégrateur/ de système de l’Internet des Objets}
\ajoutmetier{\Administrateur/ de solutions web et mobiles -- \Intégrateur/ d'applications mobiles}
\ajoutmetier{Pilote d’exploitation des réseaux spécifiques comme Wifi, LoRaWAN, Sigfox}


%%%%%%%%%%%%%%%%%%%%%%%%%%%%%%%%%%%%%
% COMPETENCE SPECIFIQUE n°1 
% (compétence SA)
%%%%%%%%%%%%%%%%%%%%%%%%%%%%%%%%%%%%%

\ajoutcompetence{IMO1}{%
Gérer les infrastructures des réseaux mobiles}{%
L'Internet des Objets (IoT) s'appuie sur différents réseaux mobiles : les solutions des opérateurs de télécommunications (3G, 4G, 5G, ...), les réseaux propriétaires (LoRaWan, Sigfox, ...) ou les technologies standardisées (bluetooth, wifi, RFID, ...) que l'entreprise choisit en fonction de ses besoins (industrie, médical, etc.). Le professionnel IOM doit donc comprendre les infrastructures autour desquelles s'articulent ces réseaux. Il doit être capable de les mettre en œuvre, de les d'administrer et d'y connecter différents équipements.
}


% Composantes essentielles => mot-clé En
\ajoutcompo{respectant les normes et protocoles en vigueur}
\ajoutcompo{intégrant les dernières technologies mobiles}


% Situations professionnelles
\ajoutsitupro{Gestion des infrastructures d’un opérateur de réseaux mobiles et d’Internet des Objets}
\ajoutsitupro{Gestion des infrastructures de réseaux mobiles dans le contexte industriel, personnel ou médical}


% Niveaux de développement de la compétence commune

%% Niveau 1
% \ajout_niveau{nom_de_la_competence}{lettre_competence}{titre_court}{description}
\ajoutniveau{}{%
Mettre en œuvre les réseaux pour la mobilité}

\ajoutapprentissage{}{Comprendre les architectures, protocoles et services des réseaux mobiles 4G/5G} 
\ajoutapprentissage{}{Mettre en œuvre des réseaux mobiles personnels ou industriel \motscles{Bluetooth, ANT+, Wifi, RFID, Zigbee, Z-wave}}

%% Niveau 2
\ajoutniveau{}{%
Raccorder des objets connectés aux réseaux mobiles}

\ajoutapprentissage{}{Comprendre les architectures et spécificités des réseaux LPWAN \motscles{LoRaWAN, Sigfox, LTE-M, NB-IoT}}
\ajoutapprentissage{}{Choisir un réseau mobile pour satisfaire les contraintes énergétiques, en bande passante, en débit et en portée d’un objet connecté}
\ajoutapprentissage{}{Mettre en œuvre des systèmes de transmissions pour l’accès à un réseau mobile \motscles{en fonction de l’orientation voulue : administrer des routeurs 4G/5G, des systèmes de transmission par exemple de type radio logicielle, la partie RF d’objets connectés}}


%%%%%%%%%%%%%%%%%%%%%%%%%%%%%%%%%%%%%
% COMPETENCE SPECIFIQUE n°2 
% (compétence SB)
%%%%%%%%%%%%%%%%%%%%%%%%%%%%%%%%%%%%%
\ajoutcompetence{IMO2}{Mettre en œuvre des applications et des protocoles sécurisés pour l'Internet des Objets}{%
Le domaine de l'Internet des Objets (IoT) s'attache autant aux objets physiques connectés (capteurs, systèmes embarqués), aux protocoles de communication, qu’aux applications exploitant les données échangées. Le professionnel IOM doit comprendre les mécanismes d’échange de données de cette chaîne de bout en bout. Il doit savoir mettre en œuvre les technologies standards, en veillant aux problématiques de sécurité, et contribuer - en créant de nouvelles applications - à faire évoluer ce domaine en plein essor.
}



% Composantes essentielles => mot-clé En
\ajoutcompo{travaillant au sein d’une équipe pluridisciplinaire}
\ajoutcompo{respectant les normes et contraintes opérationnelles}


% Situations professionnelles
\ajoutsitupro{Déploiement d’un système IoT de la source capteur aux traitements des données}
\ajoutsitupro{Gestion, administration et sécurisation d’un système IoT}


% Niveaux de développement de la compétence commune

%% Niveau 1
\ajoutniveau{}{%
Mettre en œuvre des solutions pour l'Internet des Objets}

\ajoutapprentissage{}{Intégrer des systèmes électroniques et des systèmes d’exploitation embarqués \motscles{approche système qui permet l’introduction aux capteurs, ou à tous systèmes électroniques pour l’IoT, systèmes d’exploitation pour l’IoT …}}
\ajoutapprentissage{}{Mettre en œuvre des protocoles pour les réseaux de l’IoT \motscles{utilisation et programmation de protocoles, MQTT, …}}
\ajoutapprentissage{}{Mettre en œuvre des applications et traiter des données issues des objets connectés \motscles{applications simples pour l’IoT comme la localisation, capteurs de température… Peut aller de l’intégration à la programmation d’applications pour l’IoT, big data, …}}


%% Niveau 2
\ajoutniveau{}{%
Créer des solutions sécurisées pour l'Internet des Objets
}

\ajoutapprentissage{}{Superviser et analyser le déploiement des réseaux sans-fil}
\ajoutapprentissage{}{Sécuriser les objets connectés}
\ajoutapprentissage{}{Créer et innover pour l’IoT}

