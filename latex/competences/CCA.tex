%%%%%%%%%%%%%%%%%%%%%%%%%%%%%
% Bloc de compétences communes 1 
% (compétence CA)
%%%%%%%%%%%%%%%%%%%%%%%%%%%%%

% Compétences
\ajoutcompetencecommune{RT1}{Administrer les réseaux et l'Internet}{%
%A l’ère du numérique, le bon fonctionnement du réseau est essentiel au bon fonctionnement de l’entreprise. La gestion des réseaux informatiques ou de télécommunications se décline en quatre phases : la conception du réseau, son déploiement, son administration et sa supervision. Le gestionnaire doit être très réactif en cas d’incident et savoir communiquer avec les employés/clients pour comprendre et résoudre leurs problématiques. Il s’agit d’un domaine en constante évolution où le travail d’autoformation est essentiel notamment sur la sécurité. Le travail se fait typiquement au sein d'équipes souvent internationales où chacun a son domaine de spécialisation et il est donc important de savoir s’intégrer dans un groupe, de définir des procédures pour garantir le bon fonctionnement du réseau et de documenter son travail.  L'administrateur réseau exploite des équipements complexes, comme les routeurs, commutateurs et met en œuvre des liaisons sur fibres optiques, câbles, ou sans fils. Il  assure la supervision des systèmes et réseaux pour prévenir tout dysfonctionnement ou cyberattaque.
%}
A l’ère du numérique, la gestion des réseaux informatiques ou de télécommunications, essentielle au bon fonctionnement de l'entreprise, suppose de concevoir le réseau, le déployer, l'administrer et le superviser. Le gestionnaire doit être réactif en cas d’incident et savoir communiquer avec les employés/clients pour comprendre et résoudre leurs problématiques. Il exploite des équipements complexes (routeurs, commutateurs...) et met en œuvre des liaisons (fibres optiques, câbles ou sans fils). Il doit être au fait des évolutions du domaine et de la sécurité en s'autoformant. Travaillant typiquement au sein d'équipes souvent internationales,  il doit savoir s’intégrer dans un groupe, définir des procédures pour garantir le bon fonctionnement du réseau et documenter son travail. Il assure également la supervision des systèmes et réseaux pour prévenir tout dysfonctionnement ou cyberattaque.}

% Composantes essentielles => mot-clé En
\ajoutcompo{choisissant les solutions et technologies réseaux adaptées}
\ajoutcompo{respectant les principes fondamentaux de la sécurité informatique}
\ajoutcompo{utilisant une approche rigoureuse pour la résolution des dysfonctionnements \motscles{troubleshooting}}
\ajoutcompo{respectant les règles métiers \motscles{déontologie, tests, vérifications de conformité}}
\ajoutcompo{assurant une veille technologique \motscles{y compris lecture en anglais}}

% Situations professionnelles
\ajoutsitupro{Conception et administration de l’infrastructure du réseau informatique d’une entreprise}
\ajoutsitupro{Installation et administration des services réseau informatique d’une entreprise }
\ajoutsitupro{Déploiement et administration des solutions fixes pour les clients d’un opérateur de télécommunication}


% Niveaux de développement de la compétence commune
% \ajout_niveau{nom_de_la_competence}{lettre}{titre_court}{description}

%%% Niveau 1
\ajoutniveau{}{Assister l'administrateur du réseau} % ajout niveau A

\ajoutapprentissage{AC0111}{Maîtriser les lois fondamentales de l’électricité afin d'intervenir sur des équipements de réseaux et télécommunications \motscles{habilitation électrique}} % apprentissage critique
\ajoutapprentissage{AC0112}{Comprendre l'architecture des systèmes numériques et les principes du codage de l’information}
\ajoutapprentissage{AC0113}{Configurer les fonctions de base du réseau local \motscles{Ethernet, IPv4, intro IPv6, commutation, VLANs, routage statique}}
\ajoutapprentissage{AC0114}{Maîtriser les rôles et les principes fondamentaux des systèmes d'exploitation afin d'interagir avec ceux-ci pour la configuration et administration des réseaux et services fournis \motscles{DHCP, DNS, ...}}
\ajoutapprentissage{AC0115}{Identifier les dysfonctionnements du réseau local}
\ajoutapprentissage{AC0116}{Installer un poste client \motscles{et le sécuriser: antivirus, parefeu, comptes, ...}}


%%% Niveau 2
\ajoutniveau{}{Administrer un réseau} % niveau B

\ajoutapprentissage{AC0121}{Configurer et dépanner le routage dynamique dans un réseau}
\ajoutapprentissage{AC0122}{Configurer une politique simple de QoS et les fonctions de base de la sécurité d’un réseau}
\ajoutapprentissage{AC0123}{Déployer des postes clients et des solutions virtualisées}
\ajoutapprentissage{AC0124}{Déployer des services réseaux avancés et systèmes de supervision \motscles{gestion des utilisateurs: LDAP, ActiveDirectory}}
\ajoutapprentissage{AC0125}{Identifier les réseaux opérateurs et l’architecture d’Internet}
\ajoutapprentissage{AC0126}{Travailler en équipe} % apprentissage critique

%%% Niveau 3
\ajoutniveau{}{Concevoir un réseau} % niveau C

\ajoutapprentissage{AC0131}{Concevoir un projet de réseau informatique d’une entreprise en intégrant les problématiques de haute disponibilité, de QoS et de sécurité}
\ajoutapprentissage{AC0132}{Réaliser la documentation technique de ce projet \motscles{en français et en anglais}}
\ajoutapprentissage{AC0133}{Réaliser une maquette de démonstration du projet}
\ajoutapprentissage{AC0134}{Défendre/argumenter un projet}
\ajoutapprentissage{AC0135}{Communiquer avec les acteurs du projet}
\ajoutapprentissage{AC0136}{Gérer le projet et les différentes étapes de sa mise en œuvre en respectant les délais}





