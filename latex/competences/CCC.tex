%%%%%%%%%%%%%%%%%%%%%%%%%%%%%
% Bloc de compétences communes 3 
% (compétence CC)
%%%%%%%%%%%%%%%%%%%%%%%%%%%%%

\ajoutcompetencecommune{RT3}{Créer des outils et applications informatiques pour les R\&T}{
L’administrateur de réseaux informatiques ou de télécommunications a un besoin grandissant de compétences en développement informatique. Il peut être amené à développer des applications spécifiques pour son réseau (par exemple, pour remonter et analyser des informations sur ses équipements, développer une sonde pour un logiciel de supervision ou une interface pour les objets connectés). Il peut également intervenir sur les sites Web et sur la gestion des données de l’entreprise. Parce que l’administration des systèmes et des réseaux s’oriente de plus en plus vers la programmation notamment dans le cadre de la virtualisation en général et du SDN (\textit{Software Defined Network}) en particulier, il doit être en capacité de suivre les évolutions de son métier.}

% Composantes essentielles => mot-clé En
\ajoutcompo{étant à l’écoute des besoins du client}
\ajoutcompo{documentant le travail réalisé}
\ajoutcompo{utilisant les outils numériques à bon escient}
\ajoutcompo{choisissant les outils de développement adaptés}
\ajoutcompo{intégrant les problématiques de sécurité}

% Situations professionnelles
\ajoutsitupro{Conception, déploiement et maintenance du système d'information d'une entreprise}
\ajoutsitupro{Automatisation du déploiement et de la maintenance des outils logiciels}
\ajoutsitupro{Développement d'outils informatiques à usage interne d’une équipe}

% Niveaux de développement de la compétence commune
% \ajout_niveau{nom_de_la_competence}{lettre}{titre_court}{description}

%%% Niveau 1
\ajoutniveau{}{S'intégrer dans un service informatique}

\ajoutapprentissage{AC0311}{Utiliser un système informatique et ses outils \motscles{intro système, fichiers, CLI, zips, outils collaboratifs sur Internet, hygiène informatique… : référentiel PIX et CyberEdu}}
\ajoutapprentissage{AC0312}{Lire, exécuter, corriger et modifier un programme \motscles{programmes simples type scripts, algorithme inclus}}
\ajoutapprentissage{AC0313}{Traduire un algorithme, dans un langage et pour un environnement donné \motscles{peut inclure plusieurs, y compris objet}}
\ajoutapprentissage{AC0314}{Connaître l’architecture et les technologies d’un site Web \motscles{Langages et outils : HTML, CSS, CMS, JS, responsive}}
\ajoutapprentissage{AC0315}{Choisir les mécanismes de gestion de données adaptés au développement de l’outil \motscles{BDD, SQL, LDAP, JSON, sérialisation, conception de BDD simple…}}
\ajoutapprentissage{AC0316}{S’intégrer dans un environnement propice au développement et au travail collaboratif \motscles{intro méthodes agiles}}

%%% Niveau 2
\ajoutniveau{}{Développer une application R\&T}

\ajoutapprentissage{AC0321}{Automatiser l’administration système avec des scripts}
\ajoutapprentissage{AC0322}{Développer une application à partir d’un cahier des charges donné, pour le Web ou les périphériques mobiles}
\ajoutapprentissage{AC0323}{Utiliser un protocole réseau pour programmer une application client/serveur \motscles{qq lignes de codes, sans algos distribués}}
\ajoutapprentissage{AC0324}{Installer, administrer un système de gestion de données \motscles{SQL, NoSQL}}
\ajoutapprentissage{AC0325}{Accéder à un ensemble de données depuis une application et/ou un site web \motscles{backend web dynamique, accès BDD, SQL}}

%%% Niveau 3
\ajoutniveau{}{Piloter un projet de développement d’une application R\&T}

\ajoutapprentissage{AC0331}{Élaborer les spécifications techniques et le cahier des charges d'une application informatique}
\ajoutapprentissage{AC0332}{Mettre en place un environnement de travail collaboratif \motscles{outil collaboratif, développement en équipe, documentation à destination des co-développeur}}
\ajoutapprentissage{AC0333}{Participer à la formation des utilisateurs \motscles{formation orale, documentation de logiciel à destination des utilisateurs}}
\ajoutapprentissage{AC0334}{Déployer et maintenir une solution informatique}
\ajoutapprentissage{AC0335}{S'informer sur les évolutions et les nouveautés technologiques \motscles{incluant lecture critique en anglais}}
\ajoutapprentissage{AC0336}{Sécuriser l’environnement numérique d'une application}

