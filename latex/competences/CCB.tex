%%%%%%%%%%%%%%%%%%%%%%%%%%%%%
% Bloc de compétences communes 2 
% (compétence CB)
%%%%%%%%%%%%%%%%%%%%%%%%%%%%%

\ajoutcompetencecommune{RT2}{Connecter les entreprises et les usagers}{
Les réseaux informatiques ou de télécommunications s’appuient sur des infrastructures de transmission pour l’acheminement des données entre les différents équipements du réseau. L’installation et l’administration de ces infrastructures demandent de bonnes connaissances dans certains domaines de la physique, du traitement du signal et des communications numériques. L’installation et la gestion de ces systèmes sont généralement faites par des équipes spécialisées qui doivent donc rendre compte de leur travail aux différents acteurs qui vont utiliser ces systèmes. Les phénomènes rencontrés en cas de problèmes peuvent être très complexes et il est donc essentiel de respecter les règles métiers pour éviter tout dysfonctionnement et de savoir adopter une démarche scientifique pour la gestion des incidents. Une culture scientifique est nécessaire pour comprendre et s'adapter aux rapides évolutions technologiques du secteur.}
% Résumé : Les réseaux informatiques ou de télécommunications s’appuient sur des infrastructures de transmission pour l’acheminement des données entre les différents équipements du réseau. L’installation et l’administration de ces infrastructures demandent de bonnes connaissances en physique, en traitement du signal et en communications numériques. Ces systèmes sont généralement installés et gérés par des équipes spécialisées qui doivent donc rendre compte de leur travail aux différents utilisateurs. Les phénomènes rencontrés en cas de problèmes peuvent être complexes. Il est donc essentiel de respecter les règles métiers pour éviter tout dysfonctionnement et de savoir adopter une démarche scientifique pour la gestion des incidents. Une culture scientifique est nécessaire pour comprendre et s'adapter aux rapides évolutions technologiques du secteur.}

% Composantes essentielles => mot-clé En
\ajoutcompo{communiquant avec le client et les différents acteurs impliqués, parfois en anglais}
\ajoutcompo{faisant preuve d'une démarche scientifique}
\ajoutcompo{choisissant les solutions et technologies adaptées}
\ajoutcompo{proposant des solutions respectueuses de l'environnement}

% Situations professionnelles
\ajoutsitupro{Déploiement des supports et systèmes de transmission}
\ajoutsitupro{Mise en service et administration des équipements d’accès fixe ou mobile d’un opérateur de télécommunication}
\ajoutsitupro{Déploiement et administration des accès sans fil pour l’entreprise}
\ajoutsitupro{Déploiement des systèmes de communications \motscles{ToIP, visio, chat, …}}

% Niveaux de développement de la compétence commune
% \ajout_niveau{nom_de_la_competence}{lettre}{titre_court}{description}

%%% Niveau 1
\ajoutniveau{}{Découvrir les transmissions et la ToIP}

\ajoutapprentissage{AC0211}{Mesurer et analyser les signaux \motscles{audio et vidéo, ordres de grandeurs, comprendre la dualité temps/fréquence et utiliser des équipements d’analyse spectrale, avec projets autour de la chaîne d'acquisition/traitement}}
\ajoutapprentissage{AC0212}{Caractériser des systèmes de transmissions élémentaires et découvrir la modélisation mathématique de leur fonctionnement}
\ajoutapprentissage{AC0213}{Déployer des supports de transmission \motscles{traitement au choix de la propagation, câblage LAN, fibre optique, ou une introduction à l’IoT}}
\ajoutapprentissage{AC0214}{Connecter les systèmes de ToIP \motscles{s'interprétant ici comme la possibilité de se connecter au réseau téléphonique}}
\ajoutapprentissage{AC0215}{Communiquer avec un client ou un collaborateur}

%%% Niveau 2
\ajoutniveau{}{Maîtriser les différentes composantes des solutions de connexion des entreprises et des usagers}

\ajoutapprentissage{AC0221}{Déployer et caractériser des systèmes de transmissions complexes %
\motscles{filaires, radio, les réseaux de multiplexage SDH, OTN et WDM, Comprendre les paramètres radio des systèmes de transmission sans fil}}
\ajoutapprentissage{AC0222}{Mettre en place un accès distant sécurisé \motscles{VPN end user, avec SaE faisant le lien télécoms/réseaux}}
\ajoutapprentissage{AC0223}{Mettre en place une connexion multi-site via un réseau opérateur \motscles{VPN/MPLS opérateur}}
\ajoutapprentissage{AC0224}{Administrer les réseaux d'accès des opérateurs %
\motscles{boucle locale, visions et développements en télécom ou en réseau possibles}}
\ajoutapprentissage{AC0225}{Organiser un projet pour répondre au cahier des charges}

%%% Niveau 3
\ajoutniveau{}{Déployer une solution de connexion ou de communication sur IP}

\ajoutapprentissage{AC0231}{Déployer un système de communication pour l’entreprise}
\ajoutapprentissage{AC0232}{Déployer un réseau d’accès sans fil pour le réseau d’entreprise en intégrant les enjeux de la sécurité}
\ajoutapprentissage{AC0233}{Déployer un réseau d’accès fixes ou mobiles pour un opérateur de télécommunications en intégrant la sécurité}
\ajoutapprentissage{AC0234}{Permettre aux collaborateurs de se connecter de manière sécurisée au système d'information de l'entreprise}
\ajoutapprentissage{AC0235}{Collaborer en mode projet en français et en anglais}
