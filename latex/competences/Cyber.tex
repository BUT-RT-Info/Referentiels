%%%%%%%%%%%%%%%%%%%%%%%%%%%%%%%%%
% Parcours CYBER
% (Parcours A)
%%%%%%%%%%%%%%%%%%%%%%%%%%%%%%%%%

\nouveauparcours{CyberSécurité}{Cyber}

%% Métiers accessibles 
\ajoutmetier{\Technicien/ des réseaux d'entreprises / en cybersécurité / réseaux sécurisés / d'infrastructures sécurisées}
\ajoutmetier{\Coordinateur/ cybersécurité des systèmes d'information}
\ajoutmetier{\Administrateur/ de solutions de sécurité}
\ajoutmetier{\Auditeur/ de sécurité technique}
\ajoutmetier{\Opérateur/ analyste SOC (\textit{Security Operation Center})}
\ajoutmetier{\Intégrateur/ de solutions de sécurité}
\ajoutmetier{\Administrateur/ Data Center}
%\nouveaumetier{\Expert en cybersécurité (bac+2 ou bac+3 avec 10 ans d'expérience)

%%%%%%%%%%%%%%%%%%%%%%%%%%%%%%%%%%%%%
% COMPETENCE SPECIFIQUE n°1 
% (compétence SA)
%%%%%%%%%%%%%%%%%%%%%%%%%%%%%%%%%%%%%

\ajoutcompetence{Cyber1}{Administrer un système d'information sécurisé}{%
Un Système d’Information (SI) peut être défini comme l’ensemble des ressources de l’entreprise qui permettent la gestion de l’information. Le SI est généralement associé aux technologies des infrastructures systèmes, réseaux et informatiques, aux processus qui les accompagnent, et aux personnes qui les supportent. 
La conception d’un SI commence inévitablement par une analyse de l’existant et des risques associés. Cette étape inclut une analyse des besoins de l’entreprise et permet d’identifier les biens essentiels à protéger au mieux et de définir ainsi une politique de sécurité du SI (PSSI). Une architecture du SI pourra ainsi être définie et déployée, répondant aux besoins de sécurité précédemment établis. Le facteur humain est le principal facteur de risque ; la responsabilisation et la formation du personnel sont par conséquent essentiels. En outre, les attaques sont de plus en plus courantes et sophistiquées et il est donc important pour la sécurité de s’appuyer sur les méthodes et les normes existantes, de respecter le cadre juridique et de mettre à jour régulièrement sa politique.}
%Résumé : Un Système d’Information (SI) peut être défini comme l’ensemble des ressources de l’entreprise qui permettent la gestion de l’information, incluant les infrastructures systèmes, réseaux et informatiques, les processus qui les accompagnent, et les personnes qui les supportent. Sa conception commence inévitablement par une analyse des besoins de l'entreprise et des risques associés: il s'agit d'identifier les biens essentiels à protéger au mieux pour définir une politique de sécurité du SI. Le facteur humain étant le principal facteur de risque, la responsabilisation et la formation du personnel sont donc essentiels. En outre, les attaques sont de plus en plus courantes et sophistiquées et il est donc important pour la sécurité de s’appuyer sur les méthodes et les normes existantes, de respecter le cadre juridique et de mettre à jour régulièrement sa politique.}


% Composantes essentielles => mot-clé En
\ajoutcompo{visant un juste compromis entre exigences de sécurité et contraintes d’utilisation}
\ajoutcompo{respectant les normes et le cadre juridique}
\ajoutcompo{intégrant les dernières technologies}
\ajoutcompo{travaillant en équipe}
\ajoutcompo{sensibilisant efficacement des utilisateurs}

% Situations professionnelles
\ajoutsitupro{Analyse de l’existant et étude des besoins de sécurité d’une petite structure}
\ajoutsitupro{Évolution et mise en conformité du système d'information d’une entreprise}


% Niveaux de développement de la compétence commune

%% Niveau 1
% \ajout_niveau{nom_de_la_competence}{lettre_competence}{titre_court}{description}
\ajoutniveau{}{%
Sensibiliser aux vulnérabilités d'un système d'information et aux remédiations possibles}

\ajoutapprentissage{AC0411}{Utiliser les bonnes pratiques et les recommandations de cybersécurité \motscles{cf. bonnes pratiques ANSSI}}
\ajoutapprentissage{AC0412}{Mettre en \oe{}uvre les outils fondamentaux de sécurisation d’une infrastructure du réseau \motscles{segmenter un réseau d’entreprise, identifier et filtrer le trafic réseau, gérer des profils d’utilisateur, gestion des accès WiFi, …}}
\ajoutapprentissage{AC0413}{Sécuriser les systèmes d’exploitation \motscles{mise à jour, correctifs, chiffrement d’un disque , contrôle des accès utilisateurs, disponibilité, intégrité, confidentialité, surveillance des traces, sauvegarde et restauration, durcissement, …}}
\ajoutapprentissage{AC0414}{Choisir les outils cryptographiques adaptés au besoin fonctionnel du système d'information \motscles{i.e. comprendre les propriétés fonctionnelles données par les outils cryptographique, par exemple chiffrement symétrique pour protéger un flux contre les écoutes, chiffrement asymétrique + hashage pour signer des contenus ou effectuer une authentification, etc.}}
\ajoutapprentissage{AC0415}{Connaître les différents types d’attaque \motscles{physique, niveau 2, niveau 3/4, applicative → item à lui tout seul ou à intégrer dans sécu de l’infra et des SE}}
\ajoutapprentissage{AC0416}{Comprendre des documents techniques en anglais}


%% Niveau 2
\ajoutniveau{}{%
Mettre en \oe{}uvre un système d'information sécurisé pour une petite structure}

\ajoutapprentissage{AC0421}{Participer activement à une analyse de risque pour définir une politique de sécurité pour une petite structure %
\motscles{Identifier les besoins de sécurité (niveau 2, niveau 3/4, applicatif}}
\ajoutapprentissage{AC0422}{Mettre en \oe{}uvre des outils avancés de sécurisation d’une infrastructure du réseau \motscles{filtrer le trafic réseau avec des mécanismes avancés jusqu’à l’IDS/IPS, Interconnecter des sites (VPN de différents types : IPSec, openVPN, TLS, VPN MPLS), gérer des profils d’utilisateur, gestion des accès WiFi, …}}
\ajoutapprentissage{AC0423}{Sécuriser les services \motscles{Mail, Web, DNS, LDAP, AD, ToIP, …}}
\ajoutapprentissage{AC0424}{Proposer une architecture sécurisée de système d'information pour une petite structure \motscles{choix des composants réseau et système: switch, routeur, firewall, WiFi, IDS/IPS, accès distant VPN, disposition physique / construction du SI jusqu’au déploiement, sauvegardes, …}}


%%%%%%%%%%%%%%%%%%%%%%%%%%%%%%%%%%%%%
% COMPETENCE SPECIFIQUE n°2 
% (compétence SB)
%%%%%%%%%%%%%%%%%%%%%%%%%%%%%%%%%%%%%
\ajoutcompetence{Cyber2}{Surveiller un système d’information sécurisé}{%
Afin de rester fiable et opérationnel, un système d’information doit être sous surveillance permanente et en perpétuelle évolution et ce, dès sa mise en production. Il doit donc intégrer tous les outils nécessaires à son exploitation, c’est-à-dire permettant d’en assurer sa configuration optimale, de l’administrer, de le surveiller, de le mettre à jour, d’en assurer la continuité de service (notamment le sauvegarder et le restaurer) et d’en gérer les changements.
De même, en cas d’incident constaté (ce qui doit toujours rester une éventualité envisagée), toutes les procédures à suivre doivent être prévues à l’avance et mises en \oe{}uvre afin de minimiser les impacts et conséquences de l’incident, notamment au travers de plans de reprise ou de continuité d’activité.}
% Résumé : Afin de rester fiable et opérationnel, un SI doit être sous surveillance permanente et en perpétuelle évolution et ce, dès sa mise en production. Il doit donc intégrer tous les outils nécessaires à son exploitation, c’est-à-dire permettant d’en assurer sa configuration optimale, de l’administrer, de le surveiller, de le mettre à jour, d’en assurer la continuité de service (notamment le sauvegarder et le restaurer) et d’en gérer les changements. De même, en cas d’incident constaté (ce qui doit toujours rester une éventualité envisagée), toutes les procédures à suivre doivent être prévues à l’avance et mises en œuvre afin de minimiser les impacts et conséquences de l’incident, notamment au travers de plans de reprise ou de continuité d’activité.}



% Composantes essentielles => mot-clé En
\ajoutcompo{assurant une veille permanente \motscles{attaques et défenses: CVE, CVSS, VERIS, CERT, évolutions technologiques, …}}
\ajoutcompo{réalisant les mises à jour critiques}
\ajoutcompo{automatisant des tâches}
\ajoutcompo{s’intégrant dans une équipe}
\ajoutcompo{surveillant le comportement du réseau \motscles{SIEM, …}}
\ajoutcompo{veillant au respect des contrats et à la conformité des obligations du système d'information \motscles{sûreté de fonctionnement}}

% Situations professionnelles
\ajoutsitupro{Surveillance et analyse du système d’information \motscles{collecte et analyse de logs}}
\ajoutsitupro{Audit de sécurité \motscles{analyse de risques, tests de pénétration, cyberdéfense, protection du SI, …}}
\ajoutsitupro{Gestion d'un incident de sécurité \motscles{prise en charge, réponse à incident, …}}


% Niveaux de développement de la compétence commune

%% Niveau 1
\ajoutniveau{}{%
Prendre en main les outils de surveillance et de test du système d'information}

\ajoutapprentissage{}{Administrer les outils de surveillance du système d'information \motscles{SNMP, syslog, netflow, nagios, prtg, centreon, …}}
\ajoutapprentissage{}{Administrer les protections contre les logiciels malveillants \motscles{postes clients, serveurs, proxy mail, …}}
\ajoutapprentissage{}{Automatiser les tâches d’administration \motscles{des serveurs, des services, des équipements réseaux, … avec du scripting (python, …)}}
\ajoutapprentissage{}{Prendre en main des outils de test de pénétration réseau/système \motscles{pentesting (scapy, nmap, metasploit, …, Kali-tools), hack-éthique, scripting, participation à des CTF, …}}


%% Niveau 2
\ajoutniveau{}{%
Mettre en œuvre le système de surveillance d'incidents de sécurité}

\ajoutapprentissage{}{Surveiller l'activité du système d’information \motscles{Installation (build), Administration (run) de SIEM (Security Information and Event Management): ELK, QRadar, Splunk, …, métrologie, identifier et exploiter les événements de sécurité sur des enregistrement d’historiques (remontée de logs), … automatiser les tâches (Ansible, puppet…)}}
\ajoutapprentissage{}{Appliquer une méthodologie de tests de pénétration \motscles{pentesting, utilisation avancée des outils, aller vers la remédiation des failles}}
\ajoutapprentissage{}{Gérer une crise suite à un incident de sécurité \motscles{isolation, préservation de preuves (copies de mémoires, de logs, de supports de stockage, …), application du PRA, résilience, forensic, …, CERT, CSIRT, approche collaborative pour apprendre à mettre les compétences en commun, à savoir s’organiser, à travailler en parallèle pour être plus efficace}}

% Ajout des niveaux à la compétence
% N° compétences, parcours, niveau
%\competenceparcoursniveau{Cyber}{B}{A}{A,B}
